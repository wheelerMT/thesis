Spinor Bose-Einstein condensates (BECs) present an experimentally accessible
quantum emulator of universal phenomena that appear ubiquitously across physics,
some of which are difficult or impossible to study in the laboratory.
In this thesis, we investigate a variety of such phenomena in pseudospin-1/2,
spin-1, and spin-2 BECs, ranging from quantum phase transitions to topological
interfaces.
Our investigations start with the relaxation dynamics of quantum turbulence in a
two-component BEC containing half-quantum vortices.
The temporal scaling of the number of vortices and the correlation lengths are
shown to be, at early times, strongly dependent on the relative strength of the
interspecies interaction.
At later times, the scaling is observed to be universal, independent of the
interspecies interaction, and follows scaling laws observed in the relaxation
dynamics of scalar BECs, despite our system containing topologically distinct
vortices.
A spin-1 BEC is then used as an example system for investigating scaling
behaviour in a discontinuous (first-order) quantum phase transition.
We show how the Kibble-Zurek mechanism can be generalised and applied to our
system, which gives associated scaling laws different from those observed in
continuous quantum phase transitions.
Our predictions are confirmed by mean-field numerical simulations and provide an
experimentally accessible system for investigating properties of the decay of
metastable states.
Spin-2 BECs exhibit multiple ground state phases with continuous or discrete
symmetries, making excellent candidates for studying topological interfaces.
We analytically construct sets of spinor wave functions that continuously
connect two distinct ground state phases, and show how topologically distinct
defects and textures can be created that either terminate at the interface or
continuously penetrate across it, connecting non-trivially to an object
representing a different topology on the other side.
Numerical simulations of select examples reveal a range of dynamics, including
the formation of composite cores and splitting processes.