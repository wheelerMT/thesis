Due to their large family of topological defects and rich phase diagrams
exhibiting vastly different symmetries, spinor Bose-Einstein condensates (BECs)
present an ideal test bed for studying a wide variety of different physics.
In this thesis we analytically and numerically investigate relaxation dynamics,
quantum phase transitions, and topological interfaces in pseudospin-1/2, spin-1,
and spin-2 BECs.
We first introduce the necessary mathematical models to understand the dynamics
of such systems.
Then, we dedicate a chapter to exploring the ground state phases and associated
symmetries that arise in spin-1 and spin-2 systems, and investigate the
topologically stable defects present in each phase.

Our study of such systems begins with investigating the relaxation dynamics of
quantum turbulence in a two-component Bose-Einstein condensate containing
half-quantum vortices.
We investigate both the spatial and temporal aspects of the dynamics, and draw
similarities between our system and similar studies in scalar BECs, despite our
system containing topologically distinct vortices to those in the scalar system.
In particular, investigations into the temporal aspect of the dynamics leads to
a temporal scaling regime for the number of vortices and the correlation
lengths that, at early times, is strongly dependent on the relative strength of
the interspecies interaction. At later times, however, we find that the scaling
becomes universal, independent of the interspecies interaction.

We then use a spin-1 BEC as a test bed for investigating scaling behaviour in a
discontinuous, first-order quantum phase transition (QPT) by quenching the
system from the broken-axisymmetry phase to a phase-separated ferromagnetic
phase.
By generalising the Kibble-Zurek mechanism, we find the critical exponents for
both the onset of the decay of the meta-stable state just after the critical
point is crossed, and the number of resulting phase-separated ferromagnetic
domains.
We then confirm our analytical predictions by performing numerical
simulations of the first-order QPT\ for various quench rates.

Finally, we use a spin-2 BEC as a medium for investigating topological interface
physics by analytically constructing sets of spinor wave functions that
continuously connect two distinct ground state phases.
Using these interpolating spinors, we show how topologically distinct defects
and textures can be created that either terminate at the interface or
continuously penetrate across it.
Such defects include both singular and nonsingular vortices carrying mass or
spin circulation, in addition to more complex objects such as monopoles.
By performing numerical simulations of select examples, we observe the
formation of composite defects and convoluted splitting processes of certain
vortices.
