\chapter{Introduction}


\section{Bose-Einstein condensates}
The first theoretical prediction of Bose-Einstein condensation occurred in 1924,
when Indian physicist Satyendra Nath Bose, by re-deriving Planck's law of
black-body radiation, developed a theory of statistical mechanics of photons
by treating them as a collection of particles~\cite{Bose1924}.
Einstein firstly helped Bose publish his work, before later going on to
generalise the theory by applying it to a system of \(N\)-interacting
bosons~\cite{Einstein1925}.
This then led to the Bose-Einstein distribution, which describes the
statistics of bosons over single-particle energy states:
\begin{equation}\label{eq: Bose-Einstein-distribution}
    f(\epsilon_i) = \frac{1}{e^{(\epsilon_i-\mu)/k_B T} - 1},
\end{equation}
where \(\epsilon_i\) is the energy of level \(i\), \(\mu \) is the chemical
potential, \(k_B\) is the Boltzmann constant, and \(T\) is the temperature.

Since the total number of particles is conserved, the chemical potential enters
the above distribution.
The chemical potential itself is calculated from the total particle number \(N\)
and \(T\) by the condition that the total number of particles be equal to the
sum of the particles in the individual levels.
Mathematically, \(N\) is written as
\begin{equation}
    N = \sum_i N_i = \sum_i g(\epsilon_i)f(\epsilon_i),
\end{equation}
where \(N_i\) gives the mean occupation of level \(i\) and \(g(\epsilon_i)\)
gives the degeneracy of level \(i\) (i.e., the number of distinct states with
energy level \(\epsilon_i\)).

As \(T \rightarrow 0\), Eq.~\eqref{eq: Bose-Einstein-distribution} diverges,
which implies that the total excited state capacity has to decrease to keep
the number of particles fixed.
At the precise point where the total excited states cannot accommodate the total
number of particles, Bose-Einstein condensation occurs.
At \(T=0\), all atoms must occupy the lowest energy level of the system, called
the ground state.

The critical temperature at which Bose-Einstein condensation occurs can be
derived as follows.
Let us consider a system of non-interacting bosons at thermal equilibrium at
temperature \(T\).
According to de Broglie, particles behave like waves and as such have an
associated wavelength termed the de Broglie wavelength.
This wavelength characterises the length scale of the particles localised
wave packet, and is conventionally written as
\begin{equation}
    \lambda_\text{dB} = \frac{h}{\sqrt{2\pi mk_B T}},
\end{equation}
where \(h\) is Planck's constant and \(m\) is the mass of the particle.
Since \(\lambda_\text{dB} \propto 1 / sqrt{T}\), high temperatures (\(T > T_c\))
imply that the de Broglie wavelength is small compared to the average
inter-particle spacing.
In this limit, the system exhibits classical, particle-like behaviour and the
particles closely follow the Boltzmann distribution.
Conversely, as the temperature decreases, the de Broglie wavelength associated
with each particle grows.
At some critical temperature, \(T_c\), the wavelengths of each particle become
comparable to the average inter-particle spacing and as such individual
particles become indistinguishable.
At this point the system exhibits quantum behaviour, and the particles form a
degenerate gas.

\subsection{Experimental realisation}
The first realisation of a Bose-Einstein condensate occurred in 1995, where the
group of E. A. Cornell, C.E. Wieman, and W. Ketterle successfully
\textcolor{red}{wrong! Some were Rb some were Na.} cooled
\( ^{87}\text{Rb}\) atoms to near absolute zero and observed Bose-Einstein
condensation~\cite{Anderson1995}, earning them the 2001 Nobel Prize in Physics.
This pioneering work gave birth to a whole new field of research, one which is
still just as active today.
The Bose-Einstein condensate is formed when Bosons (atoms with integer spin)
are cooled close to absolute zero and thereby condense into the same quantum
mechanical level.
As the atoms are cooled, the wavelength associated with each atom becomes larger
and larger, where eventually, at some critical temperature \(T = T_c\), the
now overlapping waves can be described by a single, macroscopic wave function.
This macroscopic wave function renders BECs a fascinating test bed for
investigating the wave-like nature of matter.
Indeed, various experiments have shown such wave-like behaviour of BECs
    [references].
Today, interest in BECs has only accelerated further, and applications of such
condensates ranges from precision measurements~\cite{Obrecht2007} to
quantum computing~\cite{Byrnes2012}.

\section{Spin degree of freedom: Spinor Bose-Einstein condensates}

\section{Topological defects in spinor Bose-Einstein condensates}

\section{Outline of the thesis}
