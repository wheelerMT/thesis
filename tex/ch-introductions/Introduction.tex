\chapter{Introduction}


\section{Bose-Einstein condensates}
The first theoretical prediction of Bose-Einstein condensation occurred in 1924,
when Indian physicist Satyendra Nath Bose, by re-deriving Planck's law of
black-body radiation, developed a theory of statistical mechanics of photons
by treating them as a collection of particles~\cite{Bose1924}.
Einstein firstly helped Bose publish his work, before later going on to
generalise the theory by applying it to a system of \(N\)-interacting
bosons~\cite{Einstein1925}.

The first realisation of a Bose-Einstein condensate occurred in 1995, where the
group of E. A. Cornell, C.E. Wieman, and W. Ketterle successfully cooled
\( ^{87}\text{Rb}\) atoms to near absolute zero and observed Bose-Einstein
condensation~\cite{Anderson1995}, earning them the 2001 Nobel Prize in Physics.
This pioneering work gave birth to a whole new field of research, one which is
still just as active today.
The Bose-Einstein condensate is formed when Bosons (atoms with integer spin)
are cooled close to absolute zero and thereby condense into the same quantum
mechanical level.
As the atoms are cooled, the wavelength associated with each atom becomes larger
and larger, where eventually, at some critical temperature \(T = T_c\), the
now overlapping waves can be described by a single, macroscopic wave function.
This macroscopic wave function renders BECs a fascinating test bed for
investigating the wave-like nature of matter.
Indeed, various experiments have shown such wave-like behaviour of BECs
    [references].
Today, interest in BECs has only accelerated further, and applications of such
condensates ranges from precision measurements~\cite{Obrecht2007} to
quantum computing~\cite{Byrnes2012}.

\section{Spin degree of freedom: Spinor Bose-Einstein condensates}

\section{Topological defects in spinor Bose-Einstein condensates}

\section{Outline of the thesis}
