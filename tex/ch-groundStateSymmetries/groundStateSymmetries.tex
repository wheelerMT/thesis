\chapter{Ground states, symmetries and defects}
In this chapter we investigate the ground states of spinor BECs obtained through
minimizing the corresponding mean-field energy functionals.
In particular, we investigate the symmetry properties using both Majorana and
spherical harmonic representations.
Furthermore, we construct the phase diagrams for conserved and non-conserved
magnetisation.
Finally, we construct some topological defects present in these systems.

There are numerous references (e.g., see~\cite{Ciobanu2000, Zhang2003,
Kawaguchi2012, StamperKurn2013}) that already provide most of these results,
but we reproduce them here to provide reference for subsequent chapters.
Furthermore, there are subtleties between the phase diagrams of spinor BECs
in the presence of conserved magnetisation that is not widely reported, so we
construct both phase diagrams (conserved and non-conserved) here to clarify
distinctions between the two.


\section{Spin-1}

\subsection{Spherical harmonic representation}

\subsection{Majorana representation}

\subsection{Ground states in a uniform system}

\subsection{Ground states with conserved magnetization}

\subsection{Stationary solutions}

\section{Spin-2}

\subsection{Spherical harmonic representation}

\subsection{Majorana representation}

\subsection{Ground states in a uniform system}

\subsection{Ground states with conserved magnetization}

\subsection{Stationary solutions}

\section{Vortices in spinor BECs}

\subsection{Spin-1}
\subsubsection{The spin-1 half-quantum vortex}

\subsection{Spin-2}

\subsubsection{The ferromagnetic phase}
\subsubsection{The cyclic phase}
\subsubsection{The uniaxial-nematic phase}
\subsubsection{The biaxial-nematic phase}
