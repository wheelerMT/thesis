\chapter{Ground states, symmetries, and defects}
In this chapter we investigate the ground states of spinor BECs obtained through
minimizing the corresponding mean-field energy functional.
In particular, we investigate the symmetry properties using both Majorana and
spherical harmonic representations.
Furthermore, we construct the phase diagrams for conserved and non-conserved
magnetisation.
Finally, we construct some topological defects present in these systems.

There are numerous references (e.g., see~\cite{Ciobanu2000, Zhang2003,
Kawaguchi2012, StamperKurn2013}) that already provide most of these results,
but we reproduce them here to provide reference for subsequent chapters.
Furthermore, there are subtleties between the phase diagrams of spinor BECs
in the presence of conserved magnetisation that is not widely reported, so we
construct both phase diagrams (conserved and non-conserved) here to clarify
distinctions between the two.


\section{Spin-1}

\subsection{Ground states in a uniform system}

For a spin-1 system, the interacting part of the energy functional contains
two independent non-linear interaction terms (\textcolor{red}{See relevant
section constructing interacting Hamiltonian}), given by
\begin{equation}
    E_\mathrm{int} = \frac{1}{2}\int c_0n^2+c_1|\vb{F}|^2 d\vb{r}.
\end{equation}
Since the density term remains fixed for a given normalised ground state, the
spin magnitude is the only relevant term when computing the ground state.
For ferromagnetic interactions (\(c_1 < 0 \)), the energy is minimised when
the spin magnitude is maximised, i.e., \(|\vb{F}| = n\).
A representative spinor for a ferromagnetic ground state is then
\begin{equation}
    \zeta^\mathrm{FM} = \mqty(1 \\ 0 \\ 0).
\end{equation}
In the absence of a magnetic field, the energy of a given spinor is degenerate
with respect to a global \(U(1)\) phase \(e^{i\theta}\) and an \(SO(3)\) spin
rotation, parameterized by three Euler angles \(\alpha, \beta \),
and \(\gamma \).
A general spin rotation represents rotations around the \(z-y-z\) axes:
\(U(\alpha, \beta, \gamma) = e^{-i\alpha F_z}e^{-i\beta F_y}e^{-i\gamma F_z}\).
In explicit matrix form, this becomes
\begin{equation}
    U(\alpha, \beta, \gamma) = \mqty(
        e^{-i(\alpha + \gamma)}\cos^2\frac{\beta}{2} &
        -\frac{e^{-i\alpha}}{\sqrt{2}}\sin\beta &
        e^{-i(\alpha - \gamma)}\sin^2\frac{\beta}{2} \\
        \frac{e^{-i\gamma}}{\sqrt{2}}\sin\beta &
        \cos\beta &
        -\frac{e^{i\gamma}}{\sqrt{2}}\sin\beta \\
        e^{i(\alpha - \gamma)}\cos^2\frac{\beta}{2} &
        \frac{e^{i\alpha}}{\sqrt{2}}\sin\beta &
        e^{i(\alpha + \gamma)}\sin^2\frac{\beta}{2}
    ).
\end{equation}
The general ferromagnetic wave function is constructed by
\begin{equation}
    \psi^\mathrm{FM} = 
    \sqrt{n}e^{i\theta}U(\alpha, \beta, \gamma)\zeta^\mathrm{FM} = 
    \sqrt{n}e^{i(\theta - \gamma)}\mqty(
        e^{-i\alpha}\cos^2\frac{\beta}{2} \\
        \frac{1}{\sqrt{2}}\sin\beta \\
        e^{i\alpha}\sin^2\frac{\beta}{2}
        ).
\end{equation}

For polar interactions (\(c_1 > 0\)), the energy is minimised by having the
spin magnitude vanish \(|F|=0\).
A representative polar spinor is
\begin{equation}
    \zeta^\mathrm{P} = \mqty(0 \\ 1 \\ 0).
    \label{eq: EAP-spinor}
\end{equation}
Similar to the FM case, a general polar wave function is given by
\begin{equation}
    \psi^\mathrm{P} = 
    \sqrt{n}e^{i\theta}U(\alpha, \beta, \gamma)\zeta^\mathrm{P} = 
    \sqrt{n}e^{i\theta}\mqty(
        -\frac{e^{-i\alpha}}{\sqrt{2}}\sin\beta \\
        \cos\beta \\
        \frac{e^{i\alpha}}{\sqrt{2}}\sin\beta
        ).
\end{equation}
Thus, in the absence of a magnetic field, there are two ground states in a
spin-1 system: polar and ferromagnetic, depending on the sign of the
spin-dependent interaction term, \(c_1\).

The presence of an external magnetic field drastically changes the valid ground
states of the spin-1 system.
\begin{figure}[htb]
    \centering
    \begin{tikzpicture}
        \node[anchor=south west, inner sep=0] (sodium) at (0, 0)
        {\includegraphics[width=0.38\textwidth]
        {gfx/ch-groundStateSymmetries/ground_states_polar_int_spin1.pdf}};
        \node[anchor=south west, inner sep=0] (rubidium) at (8, 0)
        {\includegraphics[width=0.38\textwidth]
        {gfx/ch-groundStateSymmetries/ground_states_fm_int_spin1.pdf}};
        
        \begin{scope}[x={($0.1*(sodium.south east)$)},
                      y={($0.1*(sodium.north west)$)}]
            \draw[->, thick] (0,5)--(10.3,5) node[right]{$\frac{q}{c_1n}$};
            \draw[->, thick] (4.97,0)--(4.97,10.3) node[above]{$\frac{p}{c_1n}$};
            \draw[-] (6.1, 4.8) -- (6.1, 5.2);
            \node[anchor = south west] at (6.5, 3)
            {\scriptsize $p^2=2c_1nq$};
            \draw[->, thick] (6.6, 3.1) -- (6.25,2.55);
            \node at (4.8, 7.75) {\scriptsize $1$};
            \node at (4.7, 2.25) {\scriptsize -$1$};
            \node at (6.1, 4.5) {\scriptsize 1/2};
            \node[anchor=south west] at (0.8, 8.2)
            {\textcolor{white}{Ferromagnetic (I)}};
            \node[anchor=south west] at (0.8, 0.5)
            {\textcolor{white}{Ferromagnetic (II)}};
            \node[anchor=south west] at (0.1, 4.9)
            {\textcolor{white}{\small Antiferromagnetic}};
            \node[anchor=south west] at (2.1, 3.8)
            {\textcolor{white}{(III)}};
            \node[anchor=south west] at (6.5, 4.9)
            {\textcolor{white}{Polar (IV)}};
            \node[anchor=south west] at (3.8, -1.5) {\large $c_1 > 0$};
        \end{scope}
        \begin{scope}[x={($0.1*(sodium.south east)$)},
            y={($0.1*(sodium.north west)$)}]
            \draw[->, thick] (18.9,5)--(24.2,5) node[right]{$\frac{q}{|c_1|n}$};
            \draw[->, thick] (18.9,0)--(18.9,10.3) node[above]
            {$\frac{p}{|c_1|n}$};
            \node[anchor=south west] at (18.8, 8.6) 
            {\scriptsize $p^2=q^2-2|c_1|nq$};
            \draw[->, thick] (19.3, 8.8) -- (21.5, 6.2);
            \node[anchor=south west, inner sep=0, rotate=-38] at (19, 4.2)
            {\scriptsize $p=-q$};
            \node[anchor=south west, inner sep=0, rotate=38] at (19.1, 5.3)
            {\scriptsize $p=q$};
            \node[anchor=south west, inner sep=0] at (21.2, 4.5)
            {\scriptsize $2$};
            \node[anchor=south west] at (14.8, 6.5)
            {\textcolor{white}{Ferromagnetic (I)}};
            \node[anchor=south west] at (14.8, 1.5)
            {\textcolor{white}{Ferromagnetic (II)}};
            \node[anchor=south west] at (19.7, 4.9)
            {\textcolor{white}{BA}};
            \node[anchor=south west] at (19.7, 3.8)
            {\textcolor{white}{(V)}};
            \node[anchor=south west] at (21.7, 4.9)
            {\textcolor{white}{Polar}};
            \node[anchor=south west] at (21.7, 3.8)
            {\textcolor{white}{(IV)}};
            \node[anchor=south west] at (18., -1.5) {\large $c_1 < 0$};
        \end{scope}
    \end{tikzpicture}
    \caption{\label{fig: GS-phase-diagram}Ground state phase diagrams of spin-1
    BECs for polar (\(c_1 > 0\)) and ferromagnetic (\(c_1 < 0\)) interactions
    in a parameter space of \((p, q)\).
    Solid or dashed white lines represent discontinuous and continuous phase
    transitions, respectively.}
\end{figure}
Fig.~\ref{fig: GS-phase-diagram} shows the ground state phase diagram for spin-1
BECs with \(c_1 > 0\) (left) and \(c_1 < 0\) (right) in the presence of a
magnetic field.
The full derivation of the ground state phase diagram can be found in recent
reviews~\cite{Kawaguchi2012, StamperKurn2013}.
There are five total ground states shown in Fig.~\ref{fig: GS-phase-diagram},
which are summarised in Table~\ref{tab: spin-1-ground-states}.
\begin{table}
    \centering
    \begin{tabular}{ccc}
        \toprule
        Ground state & Spinor, \(\zeta^T\) & \(F_z\) \\
        \midrule
        Ferromagnetic (I) & \((1, 0, 0)\) & 1\\
        Ferromagnetic (II) & \((0, 0, 1)\) & -1\\
        Antiferromagnetic (III) & \(\left(\sqrt{\frac{1 + p(c_1n)}{2}}, 0,
        \sqrt{\frac{1 - p(c_1n)}{2}}\right)\) & \(\frac{p}{c_1n}\) \\
        Polar (IV) & \((0, 1, 0)\) & 0 \\
        Broken-axisymmetry (V) & Eq.~\eqref{eq: BA-spinor}
        & \(\frac{p(-p^2+q^2+2qc_1n)}{2c_1nq^2}\) \\
        \bottomrule
    \end{tabular}
    \caption{\label{tab: spin-1-ground-states}Summary of the ground state
    phases in a spin-1 BEC with their respective spinors and magnetisation.}
\end{table}
There exists a fully magnetised ferromagnetic state with \(\zeta={(1, 0, 0)}^T\)
and \(F_z=1\) (state I) or \(\zeta={(0, 0, 1)}^T\) and \(F_z=-1\) (state II),
depending on the sign of the linear Zeeman shift \(p\).
For polar interactions \(c_1 > 0\), there exists an antiferromagnetic phase
(state III) with
\begin{equation}
    \zeta^\mathrm{AFM} = {\left(\sqrt{\frac{1 + p/(c_1n)}{2}}, 0,
    \sqrt{\frac{1 - p/(c_1n)}{2}}\right)}^T,
\end{equation}
and \(F_z = p/(c_1n)\), which consists of 
A non-magnetised polar phase (state IV) arises with \(\zeta={(0, 1, 0)}^T\) and
\(F_z = 0\).
Finally, a broken-axisymmetry (BA) phase (state V) occurs in a condensate with
ferromagnetic interactions that has the form
\begin{equation}
    \begin{aligned}
        \zeta_{\pm 1} &= 
                    \frac{q \pm p}{2q}\sqrt{\frac{-p^2+q^2+2c_1nq}{2c_1nq}}, \\
        \zeta_0 &= \sqrt{\frac{(q^2-p^2)(-p^2-q^2+2c_1nq)}{4c_1nq^3}},
    \end{aligned}
    \label{eq: BA-spinor}
\end{equation}
which has a magnetisation that tilts against the quantisation axis
\begin{equation}
    F_z = \frac{p(-p^2 + q^2 + 2qc_1n)}{2c_1nq^2}.
\end{equation}
These five ground states fully encapsulate the phase diagram of spin-1 BECs
in a magnetic field.

\subsection{Ground states with conserved magnetisation}
The above picture computes the ground states without explicitly conserving
magnetisation.
However, if we enforce that magnetisation be conserved, then the ground state
phase diagram changes even further.


\subsection{Spherical harmonic representation}
To visualise the symmetries of ground states it is useful to view the spherical
harmonic representation, given by
\begin{equation}
    \Psi(\hat{s}) = \sum_m\psi_m Y_f^m(\hat{s}),
    \label{eq: spherical-harmonics}
\end{equation}
where \(\hat{s}\) is a unit vector in 3D spin space, and \(Y_f^m \) are the
spherical harmonics for a spin-\(f\) state.
The symmetry can be visualised with a surface plot of \(|\Psi(\hat{s})|^2\),
where the surface colour is represented by the argument of \(\Psi(\hat{s})\).

The orientation of the spherical harmonics corresponds to the condensate spin,
and so as the spin vector rotates, the orientation of spherical harmonics
rotates to match.
In addition, the colour of the spherical harmonics corresponds to the global
phase, \( \theta \).
Therefore, the spherical harmonics give an accurate description of the
physical symmetries of the wave function, along with a pictorial representation
of how the phase is changing.
Throughout this thesis we will use the spherical harmonics to construct a
picture of what is happening to the wave function at different locations in
space, where the symmetry of the wave function can rapidly transform in a
non-trivial manner.

In spin-1, there are three \(f = 1\) spherical harmonics given by
\begin{align}
    Y_1^0(\theta, \phi) &= \frac{1}{2}\sqrt{\frac{3}{\pi}}\cos\theta, \\
    Y_1^{\pm 1}(\theta, \phi) &= 
    \frac{1}{2}\sqrt{\frac{3}{2\pi}}e^{\pm i \phi}\sin\theta.
\end{align}
The spherical harmonic representations of the spin-1 polar and ferromagnetic
ground states are shown in Fig.~\ref{fig: spin-1-spherical-harmonics}.
\begin{figure}
    \begin{subfigure}{0.49\textwidth}
        \includegraphics[width=\textwidth]
        {gfx/ch-groundStateSymmetries/FM-spherical.pdf}
        \caption{\label{subfig: spin-1-FM-spherical}
            \(\zeta^\mathrm{FM}={(1, 0, 0)}^T\)}
    \end{subfigure}
    \begin{subfigure}{0.49\textwidth}
        \includegraphics[width=\textwidth]
        {gfx/ch-groundStateSymmetries/polar-spherical.pdf}
        \caption{\label{subfig: spin-1-polar-spherical}
        \(\zeta^\mathrm{P}={(0, 1, 0)}^T\)}
    \end{subfigure}
    \caption{\label{fig: spin-1-spherical-harmonics}
    Spherical harmonics representation of the ferromagnetic and polar order
    parameters in a spin-1 BEC.\@
    (a): The spin-1 ferromagnetic ground state with
    \(\zeta^\mathrm{FM}={(1, 0, 0)}^T\).
    The black arrow represents the direction of the condensate magnetisation.
    (b): The spin-1 polar ground state with
    \(\zeta^\mathrm{P}={(0, 1, 0)}^T\) where the nematic director
    \(\hat{\vb{d}}\) is aligned with the \(z\)-axis.
    The order parameter remains unchanged about \(\pi/2\) rotations about the
    \(C_2\) axis.}
\end{figure}
We see that the ferromagnetic order parameter has an \(SO(2)\) symmetry about
the \(z\)-axis.
The polar state has two nematic lobes which have a \(\pi \) phase difference.
These lobes are aligned along an axis of symmetry given by the nematic director,
\(\hat{\vb{d}}\).
There is a further axis of symmetry about the \(C_2\) axis, about which \(\pi \)
rotations preserve the symmetry.

\subsection{Majorana representation}
An alternative description to visualising the symmetries of spinor BECs is
through the use of the Majorana representation~\cite{Majorana1932,Bloch1945},
where a spin-\(f\) system can be represented as \(2f\) points on the Bloch
sphere.
The points on the sphere are numerically calculated as the \(2f\) roots
\(z_j\) of the polynomial equation
\begin{equation}
    P^{(f)}(z) = \sum_{\alpha = 0}^{2f}
    \sqrt{\mqty(2f \\ \alpha)}\zeta_{f-\alpha}^*z^\alpha=0,
\end{equation}
where each root represents a stereographic mapping
\(z_j=\tan(\theta/2)e^{i\phi}\) of the spherical coordinates \((\theta, \phi)\).
For spin-1, the polynomial becomes
\begin{equation}
    P^{(1)}(z) = \zeta_1^*z^2+\sqrt{2}\zeta_0^*z+\zeta_{-1}^*.
\end{equation}
The disadvantage of this representation is that one is not able to see the
condensate phase.

The Majorana representations for the states shown in
Fig.~\ref{fig: spin-1-spherical-harmonics} are shown in
Fig.~\ref{fig: spin-1-majorana}.
\begin{figure}
    \centering
    \includegraphics[scale=0.5]{example-image}
    \caption{\label{fig: spin-1-majorana}Majorana representations go here.}
\end{figure}

\section{Spin-2}
The interacting Hamiltonian for the spin-2 system is given by
\begin{equation}
    E_\mathrm{int} = \int c_0n^2 + c_1|\vb{F}|^2+c_2|A_{20}|^2.
\end{equation}
As before, the density remains fixed for any normalised ground state and so
different ground states arise from the competition between the spin-
and singlet-dependent interaction strengths.

If we first consider \(c_1 < 0\) and \(c_2 > 0\), then the energy functional is
minimised when  the spin density is maximised \(|F|=2n\)
and the singlet-duo amplitude is minimised \(|A_{20}|=0\).
Such a state is ferromagnetic with spin \(|F|=2n\), which we call the FM-2
state.
There also exists a ferromagnetic state with spin \(|F|=n\), denoted as the FM-1
state.
This state is not the ground state since the FM-2 state has lower
energy.
However, it should be noted that the FM-1 state can remain stable in
certain situations, such as in the cores of vortices (see
Chapter~\ref{chap: spin-2}).
The representative spinors for the spin-2 ferromagnetic states have the form
\begin{equation}
    \zeta^\mathrm{FM-2} = \mqty(1 \\ 0 \\ 0 \\ 0 \\ 0), \qquad
    \zeta^\mathrm{FM-1} = \mqty(0 \\ 1 \\ 0 \\ 0 \\ 0).
\end{equation}

As in the spin-1 case, the energy of a given spinor in the absence of a magnetic
field is degenerate following the application of a global \(U(1)\) phase and an
\(SO(3)\) spin rotation.
In a spin-2 system, a general spin rotation is instead represented as a
\(5\times 5\) matrix of the form
\begin{multline}   
    U(\alpha, \beta, \gamma) \\
     = \mqty(
        e^{-2i(\alpha + \gamma)}C^4 & -2e^{-i(2\alpha+\gamma)}C^3S 
        & \sqrt{6}e^{-2i\alpha}C^2S^2 & -2e^{-i(2\alpha-\gamma)}CS^3
        & e^{-2i(\alpha + \gamma)}S^4
        \\
        2e^{-i(\alpha+2\gamma)}C^3S & e^{-i(\alpha+\gamma)}C^2(C^2-3S^2)
        & -\sqrt{\frac{3}{8}}e^{-i\alpha}\sin 2\beta
        & -e^{-i(\alpha-\gamma)}S^2(S^2-3C^2) & -2e^{-i(\alpha-2\gamma)}CS^3
        \\
        \sqrt{6}e^{-2i\gamma}C^2S^2 & \sqrt{\frac{3}{8}}e^{-i\gamma}\sin 2\beta
        & \frac{1}{4}(1+3\cos 2\beta)
        & -\sqrt{\frac{3}{8}}e^{-i\gamma}\sin 2\beta
        & \sqrt{6}e^{2i\gamma}C^2S^2
        \\
        2e^{i(\alpha-2\gamma)}CS^3 & -e^{i(\alpha-\gamma)}S^2(S^2-3C^2)
        & \sqrt{\frac{3}{8}}e^{i\alpha}\sin 2\beta
        & e^{i(\alpha-\gamma)}C^2(C^2-3S^2) & -2e^{i(\alpha+2\gamma)}C^3S
        \\
        e^{2i(\alpha - \gamma)}C^4 & 2e^{i(2\alpha-\gamma)}CS^3 
        & \sqrt{6}e^{2i\alpha}C^2S^2 & 2e^{i(2\alpha+\gamma)}C^3S
        & e^{2i(\alpha + \gamma)}C^4
    ),
\end{multline}
where \(S \equiv \sin(\beta/2)\) and \(C \equiv \cos(\beta/2)\).
\textcolor{red}{Need to fix size of matrix so no overflow.}

Following the same procedure as the spin-1 case and applying the above spin
rotation with a global phase \(\theta \) and condensate density \(n\) yields the
general FM-2 spinor
\begin{equation}
    \psi^\mathrm{FM} = \sqrt{n}e^{i\theta'}\mqty(
        e^{-2i\alpha} \cos^4\frac{\beta}{2} \\
        2e^{-i\alpha}\cos^3\frac{\beta}{2}\sin \frac{\beta}{2} \\
        \sqrt{6} \cos^2\frac{\beta}{2} \sin^2\frac{\beta}{2} \\
        2e^{i\alpha}\cos\frac{\beta}{2} \sin^3\frac{\beta}{2} \\
        e^{2i\alpha} \sin^4\frac{\beta}{2}
    ),
\end{equation}
where \(\theta'=\theta-2\gamma \).

Instead, let us now consider the case of \(c_1 > 0\) and \(c_2 < 0\).
We see that the energy functional is minimised when the spin is minimised
\(|\vb{F}| = 0\) but the singlet-duo amplitude is maximised
\(|A_{20}|^2 = n/5\).
Such a state is called nematic, and takes two forms: the uniaxial nematic (UN)
or biaxial nematic (BN), described by the spinors
\begin{equation}
    \zeta^\mathrm{UN} = \mqty(0 \\ 0 \\ 1 \\ 0 \\ 0), \qquad
    \zeta^\mathrm{BN} = \frac{1}{\sqrt{2}}\mqty(1 \\ 0 \\0 \\ 0 \\ 1).
\end{equation}
In the absence of a magnetic field, these two states are degenerate.
The general wave function for the UN state is
\begin{equation}
    \psi^\mathrm{UN} = \frac{\sqrt{6n}}{4}e^{i\theta}\mqty(
        e^{-2i\alpha} \sin^2\beta \\
        -2e^{-i\alpha} \sin\beta \cos\beta \\
        \sqrt{\frac{2}{3}}(3\cos^2\beta - 1) \\
        2e^{i\alpha} \sin\beta \cos\beta \\
        e^{2i\alpha} \sin^2\beta
    ),
\end{equation}
and for the BN
\begin{equation}
    \psi^\mathrm{BN} = \sqrt{\frac{n}{2}}e^{i\theta} \mqty(
        e^{-2i\alpha}\left[\left(1 - \frac{1}{2}\sin^2\beta\right)\cos 2\gamma
                            - i\cos\beta\sin 2\gamma\right] \\
        e^{-i\alpha}\sin\beta(\cos\beta\cos 2\beta - i\sin 2\gamma) \\
        \sqrt{\frac{3}{2}}\sin^2\beta \cos 2\gamma \\
        -e^{i\alpha}\sin\beta(\cos\beta\cos 2\gamma + i\sin 2\gamma) \\
        e^{2i\alpha}\left[\left(1 - \frac{1}{2}\sin^2\beta\right)\cos 2\gamma
                            + i\cos\beta\sin 2\gamma\right]
    ).
\end{equation}

Now consider \(c_1, c_2 > 0\).
The energy functional is minimised when both the spin magnitude and singlet-duo
amplitude is minimised: \(|\vb{F}| = 0, |A_{20}|^2=0\).
Such a state is referred to as the cyclic state and has the representative
spinor
\begin{equation}
    \zeta^\mathrm{C-1} = \frac{1}{2}\mqty(1 \\ 0 \\ i\sqrt{2} \\ 0 \\ 1).
    \label{eq: C-1-spinor}
\end{equation}
The general wave function is given as
\begin{equation}
    \psi^\mathrm{C} = \frac{\sqrt{n}}{2}e^{i\theta} \mqty(
        e^{-2i(\alpha+\gamma)}C^4 + 2i\sqrt{3}e^{-2i\alpha}C^2S^2
        + e^{-2i(\alpha-\gamma)}S^4
        \\
        2e^{-i(\alpha+2\gamma)}C^3S - \frac{\sqrt{3}}{2}ie^{-i\alpha}\sin 2\beta
        - 2e^{-i(\alpha-2\gamma)}CS^3
        \\
        \sqrt{6}e^{-2i\gamma}C^2S^2 + i\frac{\sqrt{2}}{4}(1+3\cos 2\beta)
        + \sqrt{6}e^{2i\gamma}C^2S^2
        \\
        2e^{i(\alpha-2\gamma)}CS^3 + \frac{\sqrt{3}}{2}ie^{i\alpha}\sin 2\beta
        - 2e^{i(\alpha+2\gamma)}C^3S
        \\
        e^{2i(\alpha-\gamma)}S^4 + 2i\sqrt{3}e^{2i\alpha}C^2S^2
        + e^{2i(\alpha+\gamma)}C^4
    ).
\end{equation}

In addition to the three-component cyclic state, there is also a two-component
cyclic state that is useful for understanding the general cyclic state:
\begin{equation}
    \psi^\mathrm{C-2} = \sqrt{\frac{n}{3}}\mqty(1  \\ 0 \\ 0 \\ \sqrt{2} \\ 0),
\end{equation}
which is obtained from Eq.~\eqref{eq: C-1-spinor} via the spin rotation
\begin{equation}
    \psi^{C-2} = e^{-i\pi}e^{i\frac{\pi}{4}F_z}
    \exp\left[-i\frac{F_x-F_y}{\sqrt{2}}
    \arccos{\left(\frac{1}{\sqrt{3}}\right)}\right]\psi^\mathrm{C-1}.
\end{equation}

Finally, for the case of \(c_1, c_2 < 0\), there is a competition between
the ferromagnetic and nematic phases.
For this case the energy functional is minimised by either having maximal spin
density and \(|A_{20}|^2 = 0\) as in the ferromagnetic phase, or by having
minimal spin density and \(|A_{20}|^2 = n/5\) as in the nematic phase.
This leads to a phase boundary at \(c_2n=20c_1n\).
The ground states of the spin-2 system in a parameter space of \((c_1, c_2)\)
are summarised in Fig.~\ref{fig: spin-2-ground-states}.
\begin{figure}
    \centering
    \begin{tikzpicture}
        \node[anchor=south west, inner sep=0] (diagram) at (0, 0)
        {\includegraphics[width=0.5\textwidth]
        {gfx/ch-groundStateSymmetries/ground_states_spin2.pdf}};
        \draw[->, thick, anchor=south west] (7.4, 2.86) -- (7.7, 2.86)
        node[right] {$c_1n$};
        \draw[->, thick, anchor=south west] (3.785, 5.57) -- (3.785, 5.87)
        node[above] {$c_2n$};
        \draw[-, thick, anchor=south west] (0, 2.86) -- (3.785, 2.86);
        \node at (5.5, 4) {\textcolor{white}{Cyclic}};
        \node at (2, 4) {\textcolor{white}{Ferromagnetic}};
        \node at (5.2, 1.4) {\textcolor{white}{Nematic}};
        \node[anchor=south east] at (3.75, 2.86) {\textcolor{white}{0}};
        \node[anchor=south west, rotate=52] at (2.1, 0.5)
        {\textcolor{white}{\(c_2n=20c_1n\)}};
    \end{tikzpicture}
    \caption{\label{fig: spin-2-ground-states}Ground state phase diagram for
    spin-2 BECs in a parameter space of \((c_1, c_2)\) in the absence of a
    magnetic field.
    White dashed lines indicate a first-order phase transition region between
    the phases.}
\end{figure}

\subsection{Spherical harmonic representation}
The mapping of the order parameter onto spherical harmonics in the spin-2 case
follows the same equation as in the spin-1 case, i.e.,
Eq.~\eqref{eq: spherical-harmonics}.
For the spin-2 system, however, we have five \(f=2\) spherical harmonics given
by
\begin{align}
    Y_2^0(\theta, \phi) &= \frac{1}{4}\sqrt{\frac{5}{\pi}}(3\cos^2\theta - 1),\\
    Y_2^{\pm 1}(\theta, \phi) &= 
    \mp \frac{1}{2}\sqrt{\frac{15}{2\pi}}e^{\pm i\phi}\sin\theta\cos\theta, \\
    Y_2^{\pm 2}(\theta, \phi) &=
    \frac{1}{4}\sqrt{\frac{15}{2\pi}}e^{\pm 2i\phi}\sin^2\theta.
\end{align}
The spherical harmonics for the ferromagnetic, UN, BN, and cyclic states
are shown in Fig.~\ref{fig: spin-2-spherical-harmonics}.
\begin{figure}
    \centering
    \vspace{-2cm}
    \begin{subfigure}{0.49\textwidth}
        \includegraphics[width=\textwidth]
        {gfx/ch-groundStateSymmetries/FM-2-spherical.pdf}
        \caption{\label{subfig: FM-2-spherical}
        FM-2: \(\zeta={(1, 0, 0, 0, 0)}^T\)
        }
    \end{subfigure}
    \begin{subfigure}{0.49\textwidth}
        \includegraphics[width=\textwidth]
        {gfx/ch-groundStateSymmetries/FM-1-spherical.pdf}
        \caption{\label{subfig: FM-1-spherical}
        FM-1: \(\zeta={(0, 1, 0, 0, 0)}^T\)}
    \end{subfigure}\\
    \begin{subfigure}{0.49\textwidth}
        \includegraphics[width=\textwidth]
        {gfx/ch-groundStateSymmetries/UN-spherical.pdf}
        \caption{\label{subfig: UN-spherical}
        UN:\@ \(\zeta={(0, 0, 1, 0, 0)}^T\)}
    \end{subfigure}
    \begin{subfigure}{0.49\textwidth}
        \includegraphics[width=\textwidth]
        {gfx/ch-groundStateSymmetries/BN-spherical.pdf}
        \caption{\label{subfig: BN-spherical}
            BN:\@ \(\zeta={(1, 0, 0, 0, 1)}^T/\sqrt{2}\)}
    \end{subfigure}\\
    \begin{subfigure}{0.49\textwidth}
        \includegraphics[width=\textwidth]
        {gfx/ch-groundStateSymmetries/C1-spherical.pdf}
        \caption{\label{subfig: C-1-spherical}
        Cyclic-1:\@ \(\zeta={(1, 0, i\sqrt{2}, 0, 1)}^T/2\)}
    \end{subfigure}
    \begin{subfigure}{0.49\textwidth}
        \includegraphics[width=\textwidth]
        {gfx/ch-groundStateSymmetries/C2-spherical.pdf}
        \caption{\label{subfig: C-2-spherical}
        Cyclic-2:\@ \(\zeta={(1, 0, 0, \sqrt{2}, 1)}^T/\sqrt{3}\)}
    \end{subfigure}
    \caption{\label{fig: spin-2-spherical-harmonics}Spherical harmonic
    representations for different ground states in a spin-2 system.
    (a) and (b): Ferromagnetic states with a spin of \(|F|=2n\) and \(|F|=n\),
    respectively, where the arrows indicate the direction of magnetisation.
    (c): the uniaxial nematic state where \(\hat{\vb{d}}\) is the nematic
    director. THe order parameter remains unchanged about \(\pi \) rotations
    about the \(C_2\) axis.
    (d): the biaxial nematic state. The order parameter is symmetric under
    \(\pi/2\) rotations about the \(C_4\) axis.
    Additionally, there is a two-fold symmetry about the \(C_2, C_2'\) axes.
    (e) and (f): the cyclic states which have a two- and three-fold symmetry
    about the \(C_2, C_3\) axes, respectively.
    \textcolor{red}{Colour bars.}}
\end{figure}

It is clear from Figs~\ref{subfig: FM-2-spherical}
and~\ref{subfig: FM-1-spherical} that the ferromagnetic order parameters have
an \(SO(2)\) symmetry about the \(z\)-axis.
The UN phase as shown in Fig.~\ref{subfig: UN-spherical} differs slightly from
the polar phase of spin-1 (see Fig.~\ref{subfig: spin-1-polar-spherical}) in
that the nematic lobes have the same phase.
This implies that a \(\pi \) spin rotation about any axis in the \(xy\)-plane
leaves the order parameter unchanged.
In addition, like the ferromagnetic states, this order parameter also has an
\(SO(2)\) symmetry about the \(z\)-axis.
The BN phase, shown in Fig.~\ref{subfig: BN-spherical}, breaks the \(SO(2)\)
symmetry due to the perpendicular nematic lobes, which have a \(\pi \) phase
difference.
The symmetry of the order parameter is preserved under \(\pi/4\) rotations
abotu the \(C_4\) axis.
In addition, the BN order parameter is invariant under \(\pi \) rotations about
both the \(C_2\) and \(C_2'\) axes.
Finally, the cyclic order parameter, shown in Figs.~\ref{subfig: C-1-spherical}
and~\ref{subfig: C-2-spherical}, have the symmetry of a tetrahedron.
Each nematic lobe has a two-fold symmetry about the \(C_2\) axis.
Furthermore, the order parameter has a three-fold symmetry about the \(C_3\)
axis.
In Fig.~\ref{subfig: C-1-spherical}, this axis is the \((1, 1, 1)\)-axis,
whereas in Fig.~\ref{subfig: C-2-spherical}, it is aligned along the \(z\)-axis.
Rotations of \(2\pi/3\) about this axis preserve the symmetry of the order
parameter.

\subsection{Majorana representation}
In the spin-2 system, we compute the \(2f=4\) roots of the complex polynomial
\begin{equation}
    P^{(2)}(z) = \zeta_2^*z^4 + 2\zeta_1^*z^3 + \sqrt{6}\zeta_0^*z^2
    + 2\zeta_{-1}^*z + \zeta_{-2}^*.
\end{equation}
The Majorana representation of the ferromagnetic, UN, BN, and cyclic states
are shown in Fig.~\ref{fig: spin-2-Majorana}.
\begin{figure}
    \centering
    \includegraphics[width=\textwidth]{example-image}
    \caption{\label{fig: spin-2-Majorana}}
\end{figure}

\section{Vortices in spinor BECs}

\subsection{Spin-1}
\subsubsection{The spin-1 half-quantum vortex}

\subsection{Spin-2}

\subsubsection{The ferromagnetic phase}
\subsubsection{The cyclic phase}
\subsubsection{The uniaxial-nematic phase}
\subsubsection{The biaxial-nematic phase}
