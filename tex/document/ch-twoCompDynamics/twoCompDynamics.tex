\chapter{Relaxation dynamics in a two-component system}

\section{Introduction}
As we have seen \textcolor{red}{IN RELEVANT PART}, the two-component BEC can be 
treated as a pseudospin-1/2 system. 
This new system gives rise to novel defects such as a half-quantum vortex 
otherwise unseen in a scalar condensate. 
In this chapter we investigate the relaxation dynamics of half-quantum vortices 
(HQVs) in a two-dimensional, two-component condensate. 
Our interest is studying the scaling laws that govern the decay rate of the 
vortices, and consequently the growth of the length scales associated with 
domains in the system, whilst varying the ratio of inter- to intra-species 
interactions. 
We study these scales by starting from an initially turbulent state full of 
HQVs and subsequently letting the system relax in time. 
Upon the relaxation, vortices will annihilate leading to domain growth within 
the system.
To extract the appropriate length scales of these domains, we construct 
correlation functions originally defined for an antiferromagnetic spin-1 
system~\cite{Symes2017}.

These correlation functions then allow us to extract relevant length scales 
associated with spin and mass order. 
By investigating these length scales temporally, we reveal interesting, 
novel dynamics occurring at early times for a sufficiently high ratio of 
inter- to intra-species interactions. 
This result is then confirmed by considering the total vortex number of the 
system. 
\par
Recent work has alluded to fascinating inter-vortex forces arising between HQVs 
of opposite winding within the same condensate 
component~\cite{Eto2011,Kasamatsu2016}.
We relate this discovery to the interesting 
dynamics we observe at early time.

\section{The spin-1 easy-plane polar phase}
As we saw in \textcolor{red}{relevant part}, the spin-1 condensate with polar
interactions supports a polar ground state.
This state can be categorized in two different ways:
The first consists of the nematic director being aligned with the $z$-axis,
which is referred to as the easy-axis polar (EAP) phase.
This phase is stable for $p=0$ and $q>0$.
The second case consists of the director being in the $xy$-plane, which is
referred to as the easy-plane polar (EPP) phase.
This phase is stable for $p=0$ and $q<0$.
The representative wavefunction for the EPP phase can be written as
\begin{equation}
    \psi_\mathrm{EPP} = \frac{1}{\sqrt{2}}\begin{pmatrix}
        1 \\ 0 \\ 1
    \end{pmatrix}.
    \label{eq:EPP_wavefunction}
\end{equation}
This state has only the outer two components occupied, requiring an empty
middle component.
Due to this fact, this representative wavefunction can be mapped to a
two-component system.

\subsection{Mapping the EPP phase onto a two-component system}
To begin the mapping procedure, we first construct the time-independent
GPEs for the spin-1 system with a wavefunction that assumes an empty middle
component:
\begin{equation}
    \begin{aligned}
        \left[-\frac{\hbar^2\nabla^2}{2M}
        + (c_0 + c_2)|\psi_1|^2 + (c_0 - c_2)|\psi_{-1}|^2 
        + q - \mu\right]\psi_1 &= 0, \\
        \left[-\frac{\hbar^2\nabla^2}{2M}
        + (c_0 + c_2)|\psi_{-1}|^2 + (c_0 - c_2)|\psi_1|^2 
        + q - \mu\right]\psi_{-1} &= 0,
    \end{aligned}
    \label{eq:EPP-time-independent-GPEs}
\end{equation}
where $\mu$ is the chemical potential of the system and we have taken $p=0$.

Similarly, the time-independent GPEs for the two component system are
\begin{equation}
    \begin{aligned}
        \left(-\frac{\hbar^2\nabla^2}{2m_1} + g_1|\psi_1|^2
        +g_{12}|\psi_2|^2 - \mu_1\right)\psi_1 &= 0, \\
        \left(-\frac{\hbar^2\nabla^2}{2m_2} + g_2|\psi_2|^2
        +g_{12}|\psi_1|^2 - \mu_2\right)\psi_2 &= 0.
    \end{aligned}
    \label{eq:two-comp-time-independent-gpes}
\end{equation}
Using these time-independent equations, we can map the two-component system
to that of the spin-1 by comparing the coefficients with
Eq.~\eqref{eq:EPP-time-independent-GPEs}.
Doing this we find
\begin{equation}
    g_1=g_2=c_0+c_2, \enskip g_{12} = c_0-c_2, \enskip \mu_1=\mu_2=\tilde{\mu}, 
    \enskip m_1=m_2=M,
\end{equation}
where $\tilde{\mu} = \mu - q$.
Therefore, for the specific cases defined above, the two-component system maps
directly to that of the spin-1 system.

\section{Half-quantum vortices in spin-1/2}
Scalar BECs only support one type of vortex, namely, the singly-quantized
vortex.
Two-component BECs allow a novel vortex configuration otherwise unseen in
scalar BECs: The two-component half-quantum 
vortex (HQV).
The half-quantum vortex is unique such that the circulation of the vortex is
quantized in units of $\kappa / 2$ where $\kappa=h/m$ [REFS].
To construct such a vortex, we start by defining the pseudospin-1/2
wavefunction:
\begin{equation}
    \twovec{\psi_1}{\psi_2} 
    = \twovec{|\psi_1|e^{i\theta_1}}{|\psi_2|e^{i\theta_2}}
    = e^{\Theta}\twovec{|\psi_1|e^{i\Phi}}{|\psi_2|e^{i\Phi}},
    \label{eq:pseudospin-1/2-wavefunction}
\end{equation}
where $\theta_j=\mathrm{Arg}(\psi_j)$ for $j=1,2$ and
\begin{equation}
    \Theta = \frac{\theta_1 + \theta_2}{2}, \enskip 
    \Phi = \frac{\theta_1 - \theta_2}{2}.
\end{equation}
Gradients in $\Theta$ are associated with a total, superfluid mass current
whereas gradients in $\Phi$ are associated with a pseudospin currents.
\textcolor{red}{Do I want to elaborate on this further? Potentially by looking
at actual expressions for mass and pseudospin current in two-component BECs.} \\
Now consider a vortex state which consists of a phase singularity in the $\psi_1$
component, where about the singularity $\theta_1$ winds by $2\pi$ and $\theta_2$
remains unchanged, i.e. it is a smooth phase field.
Such a state can be written as
\begin{equation}
    \twovec{\psi_1}{\psi_2} 
    = \twovec{|\psi_1|e^{i\varphi}}{|\psi_2|}
    = e^{\varphi/2}\twovec{|\psi_1|e^{i\varphi/2}}{|\psi_2|e^{-i\varphi/2}},
\end{equation}
where $\varphi$ is the azimuthal angle around the vortex core.
By comparing the above to~\eqref{eq:pseudospin-1/2-wavefunction} we see that
the vortex is described by a $\pi$ change in $\Theta$ with a simultaneous $\pi$
change in $\Phi$ along a closed path about the vortex.
As $\Theta$ is associated with a total mass current, such a vortex state is
referred to as a half-quantum vortex\footnote{This vortex is topologically
distinct from HQVs that arise in the A and polar phases of superfluid $^3$He
[REFS] and the polar phase of spin-1 BECs [REFS].}.
\par

\section{Half-quantum vortex relaxation dynamics}
To begin studying the relaxation dynamics of HQVs in a turbulent system, we
numerically solve the two-component Gross-Pitaevskii equations in dimensionless
form
(\textcolor{red}{see Appendix for details of dimensionless form}):
\begin{equation}
    \begin{aligned}
        i\frac{\partial \psi_1}{\partial t} &= (-\nabla^2 + g|\psi_1|^2
        + \gamma|\psi_2|^2)\psi_1, \\
        i\frac{\partial \psi_2}{\partial t} &= (-\nabla^2 + g|\psi_2|^2
        + \gamma|\psi_1|^2)\psi_2.
    \end{aligned}
    \label{eq:dimensionless-two-comp-GPEs}
\end{equation}
where we have assumed each component has the same atomic mass and inter-species
interaction strength.
The key parameter is the ratio of inter- to intra-species interaction
\begin{equation}
    \gamma = \frac{g_{12}}{g}.
\end{equation}
We consider the case $0 < \gamma < 1$ with all interactions repulsive such that
the condensate is stable against the separation of the components. \\
The inter- and intra-component interactions give rise to two important length
scales within the system.
These are, respectively, associated with variations in the total superfluid
density and the difference in density between the components.
The density and spin healing lengths are then defined as~\cite{Eto2011}
\begin{equation}
    \xi_d = \frac{\hbar}{\sqrt{2mgn_0}}, \qquad 
    \xi_s = \xi_d \sqrt{\frac{1 + \gamma}{1 - \gamma}},
    \label{eq:healing-lengths}
\end{equation}
where $n_0$ is the background number density of each component in a uniform
system.
The size of the HQV core can be understood from the energetic hierarchy of
these healing lengths.
Since a HQV consists of a phase singularity in one component and not the other,
the core is free to fill with the atoms of the other component.
This corresponds to spatial variations in the $z$-component of the pseudospin,
the size of which is set by the spin healing length, $\xi_s$.
The vortex core is able to expand when $\xi_s \lesssim \xi_d$, which lowers
the total energy.
We see from Eq.~\eqref{eq:healing-lengths} that $\gamma$ directly determines
the core sizes within our systems.
Similar energetic hierarchies exist in spinor BECs, which can facilitate
deformations of vortex cores such as the splitting of singly quantized vortices
into fractional vortices [REFS].

\subsection{Numerical setup}
We solve Eq.~\eqref{eq:dimensionless-two-comp-GPEs} on a periodic domain with 
$N_s^2=1024^2$ grid points which has dimensionless area $L^2=N_s^2$ with side
length $L=N_s$.
We take $N=3.2\times10^9$ atoms per component and fix dimensionless $g=L^2/4N$.
The dimensionless density healing length is then fixed at
$\xi_d=N_s/\sqrt{gN}=2$ in our system.
We wish to explore the effect of the inter-component interaction by varying
$\gamma$ within the range $0 < \gamma < 1$.\par

The initial state is constructed as follows.
We wish to construct $48^2$ vortices in each component.
To do this we start with a grid of vortex positions such that the 
$i$-th vortex is to be imprinted at position $(x_i, y_i)$.
We repeat the same procedure for the other component, however to avoid
overlapping vortex positions, we offset this grid in both $x$ and $y$ whilst
preserving the grid structure.
To facilitate a random distribution of vortices and subsequently turbulent
dynamics, we then displace each position by some small amount 
$(x_i, y_i) \rightarrow (x_i \delta + x_i, y_i + \delta y_i)$, where 
$\delta x_i, \delta y_i < 3\xi_d$.
The vortices are then constructed as an alternating $2\pi$ winding of the
component phase about each vortex position using the method described
in~\cite{Billam2014}.
We imprint the vortices using a short imaginary time propagation of
Eq.~\eqref{eq:dimensionless-two-comp-GPEs} whilst keeping the phase profile
fixed so as to not alter the positions of the vortices.
This results in an initially turbulent system of HQVs.
The HQV cores correspond to a density depletion in one component at the position
of the phase singularity with a corresponding density peak at the same position
in the other component (see Fig.~\ref{fig:initial-vortex-state}).
\begin{figure}
    \centering
    \includegraphics[width=0.5\textwidth]{example-image-a}
    \caption{Density of $\psi_1$ component in a $100\xi_d\times100\xi_d$
    subregion of the initial state after imaginary time propagation.
    We see the HQVs in this component by the density depletion.
    The density peaks correspond to the location of HQV cores in the other
    component, which have been filled by atoms in this component.}
    \label{fig:initial-vortex-state}
\end{figure}
Previous work has shown that clustering of vortices can lead to anomalously
slow coarsening dynamics~\cite{Karl2017} and thus constructing the initial
state this way ensures that there is no clustering of like-signed vortices.
From this initial state, the system evolves according to
Eq.~\eqref{eq:dimensionless-two-comp-GPEs}.
Two HQVs in the same component with opposite winding can annihilate, leading to
a decay of the total vortex number within the system.

\subsection{Investigating the kinetic energy spectrum}
We begin by investing spatial aspects of the relaxation dynamics.

\subsection{Temporal aspects of decay}

\subsection{Inter-vortex interaction between HQVs}
