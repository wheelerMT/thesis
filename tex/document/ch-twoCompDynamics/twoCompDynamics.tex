\chapter{Relaxation dynamics in a two-component system}

\section{Introduction}
Since the realisation of superfluidity, quantum turbulence (QT) has been studied
 in systems ranging from superfluid liquid helium [1,2] to quasi-particle 
 condensates in solid-state systems [3]. 
Due to their unprecedented experimental accessibility, QT in Bose-Einstein 
condensates (BECs) in dilute, ultracold atomic gases has attracted considerable 
theoretical [4–9] and experimental [10–15] interest in both 2D and 3D 
configurations.
In a scalar BEC, the QT state is made up of many vortices with quantised 
circulation. 
The collective behaviour of the vortices plays a key role in the hydrodynamics, 
recovering features of classical turbulence that can exhibit the characteristic 
Kolmogorov power-law spectrum [16].

As we have seen \textcolor{red}{IN RELEVANT PART}, the two-component BEC can be 
treated as a pseudospin-1/2 system. 
This new system gives rise to novel defects such as a half-quantum vortex 
otherwise unseen in a scalar condensate. 
In this chapter we investigate the relaxation dynamics of half-quantum vortices 
(HQVs) in a two-dimensional, two-component condensate. 
Our interest is studying the scaling laws that govern the decay rate of the 
vortices, and consequently the growth of the length scales associated with 
domains in the system, whilst varying the ratio of inter- to intra-species 
interactions. 
We study these scales by starting from an initially turbulent state full of 
HQVs and subsequently letting the system relax in time. 
Upon the relaxation, vortices will annihilate leading to domain growth within 
the system.
To extract the appropriate length scales of these domains, we construct 
correlation functions originally defined for an antiferromagnetic spin-1 
system~\cite{Symes2017}.

These correlation functions then allow us to extract relevant length scales 
associated with spin and mass order. 
By investigating these length scales temporally, we reveal interesting, 
novel dynamics occurring at early times for a sufficiently high ratio of 
inter- to intra-species interactions. 
This result is then confirmed by considering the total vortex number of the 
system. 
\par
Recent work has alluded to fascinating inter-vortex forces arising between HQVs 
of opposite winding within the same condensate component 
\textcolor{red}{[ETO PAPERS]}. 
We relate this discovery to the interesting 
dynamics we observe at early time.
