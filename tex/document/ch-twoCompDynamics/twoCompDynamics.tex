\chapter{Relaxation dynamics in a two-component system}

\section{Introduction}
As we have seen \textcolor{red}{IN RELEVANT PART}, the two-component BEC can be 
treated as a pseudospin-1/2 system. 
This new system gives rise to novel defects such as a half-quantum vortex 
otherwise unseen in a scalar condensate. 
In this chapter we investigate the relaxation dynamics of half-quantum vortices 
(HQVs) in a two-dimensional, two-component condensate. 
Our interest is studying the scaling laws that govern the decay rate of the 
vortices, and consequently the growth of the length scales associated with 
domains in the system, whilst varying the ratio of inter- to intra-species 
interactions. 
We study these scales by starting from an initially turbulent state full of 
HQVs and subsequently letting the system relax in time. 
Upon the relaxation, vortices will annihilate leading to domain growth within 
the system.
To extract the appropriate length scales of these domains, we construct 
correlation functions originally defined for an antiferromagnetic spin-1 
system~\cite{Symes2017}.

These correlation functions then allow us to extract relevant length scales 
associated with spin and mass order. 
By investigating these length scales temporally, we reveal interesting, 
novel dynamics occurring at early times for a sufficiently high ratio of 
inter- to intra-species interactions. 
This result is then confirmed by considering the total vortex number of the 
system. 
\par
Recent work has alluded to fascinating inter-vortex forces arising between HQVs 
of opposite winding within the same condensate 
component~\cite{Eto2011,Kasamatsu2016}.
We relate this discovery to the interesting 
dynamics we observe at early time.

\section{Mapping of Spin-1 to Spin-1/2}

\section{Half-quantum vortices in spin-1/2}

\section{Numerical setup}

\section{Investigating the kinetic energy spectrum}

\section{Temporal aspects of decay}

\section{Inter-vortex interaction between HQVs}
