\chapter{Spinor Bose-Einstein condensates}

In this chapter we provide the theoretical background necessary to understand
the dynamics of ultracold atomic gases.
In particular, we introduce the Hamiltonians of both scalar and spinor BEC
systems.
We start with the simplest case of the scalar BEC, which has one non-linear
interaction term.
Then, by introducing the spin degrees of freedom, we introduce the extra
interactions present in spinor BECs.

\section{Single-particle Hamiltonian}
Spinor systems comprise particles with total spin \(f\).
This implies there are \(2f + 1\) possible spin states along a given spin
quantisation axis.
Such a state is denoted \(\ket{f, m} \), where
\(m \in \{-f, -f+1, \ldots, 0, \ldots, f - 1, f\} \) denotes the magnetic
sublevel for total spin \( f\).

A general spin-\(f\) state is described by the (\(2f+1\))-component wave
function \(\psi \), defined as \(\ket{\psi} = \sum_m\psi_m\ket{f, m}\).
In the spin-1 case, we have \(\psi = {(\psi_1, \psi_0, \psi_{-1})}^T\), and for
the spin-2 case \(\psi = {(\psi_2, \psi_1, \psi_0, \psi_{-1}, \psi_{-2})}^T\).

We start by first introducing the single-particle Hamiltonian for spinor BECs.
We assume a system of identical bosons of mass \(M\) and spin \(f\).
When a magnetic field is applied to a spinor system, the field causes energy
shifts in spin components.
When this field is aligned along the spin quantisation axis, linear, \(p\), and
quadratic, \(q\), Zeeman shifts arise.
In such a case, the single-particle (non-interacting) Hamiltonian is given
by~\cite{Kawaguchi2012}
\begin{equation}\label{eq: single-particle-Hamiltonian}
    H_0 = \int d\vb{r}\sum_{m,m'=-f}^{f} \psi_m^\dagger \left[
        -\frac{\hbar^2}{2M}\nabla^2 + V(\vb{r}) - pm + qm^2\right]\psi_{m'},
\end{equation}
where \(V(\vb{r})\) is a trapping potential.
Throughout most of this thesis we assume that the magnetic field applied is
uniform, such that the Zeeman shifts are constant.
In Chapter~\ref{chap: spin-1}, however, we introduce a time-dependent field
such that the quadratic Zeeman shift has a time-dependence.
Despite the time-dependence, the quadratic Zeeman shift is still spatially
uniform.

It becomes apparent from Eq.~\eqref{eq: single-particle-Hamiltonian} that the
quadratic shift uniformly alters the energetic separation between the
\(\psi_{\pm 1}\) and \(\psi_0\) components.
Conversely, the linear shift maintains the separation.

\section{Scalar BEC interactions}
The many-body nature of BEC systems leads to interactions that arise between
particles.
We begin by first discussing the interactions arising in the simplest BEC\@: the
scalar BEC\@.
We consider a gas of identical bosons, described by the complex order parameter,
\(\psi \).
Collisions of two particles at location \(\vb{r}_1, \vb{r}_2\) are modelled by
\begin{equation}\label{eq: scalar-interactions}
    E_\mathrm{int} = \frac{1}{2}\int \int d\vb{r}_1d\vb{r}_2\,
    \psi^\dagger(\vb{r}_1)\psi^\dagger(\vb{r}_2)
    V_\mathrm{scat}(\vb{r}_1-\vb{r}_2)\psi(\vb{r}_1)\psi(\vb{r}_2),
\end{equation}
where \(V_\mathrm{scat}\) describes the two-body interaction potential.

To describe the low-energy dynamics of the dilute Bose gas, with most atoms
occupying the ground state, we are free to neglect particles with high momenta
and instead replace the two-body potential with an effective contact interaction
between particles:
\begin{equation}
    V_\mathrm{scat}(\vb{r}_1 - \vb{r}_2) = g_s\delta(\vb{r}_1 - \vb{r}_2),
\end{equation}
where \(g_s=4\pi\hbar^2a_s/M\) is the interaction strength for \(s\)-wave
scattering length \(a_s\) and atomic mass \(M\).
Substituting the above contact potential into
Eq.~\eqref{eq: scalar-interactions} results in the scalar interaction
\begin{equation}
    E_\mathrm{scalar} = \frac{g_s}{2}\int d\vb{r} n^2,
\end{equation}
where we have used the relation \(n = \psi^\dagger \psi \) for condensate
density \(n\).

\section{Spinor BEC interaction Hamiltonian}

\subsection{Spin-1 interaction Hamiltonian}

\subsection{Spin-2 interaction Hamiltonian}
