\chapter{Spinor Bose-Einstein condensates}

In this chapter we provide the theoretical background necessary to understand
the dynamics of ultracold atomic gases.
In particular, we introduce the Hamiltonians of both scalar and spinor BEC
systems.
We start with the simplest case of the scalar BEC, which has one non-linear
interaction term.
Then, by introducing the spin degrees of freedom, we introduce the extra
interactions present in spinor BECs.

\section{Scalar BEC interactions}
The many-body nature of BEC systems leads to interactions that arise between
particles.
We begin by first discussing the interactions arising in the simplest BEC\@: the
scalar BEC\@.
We consider a gas of identical bosons, described by the complex order parameter,
\(\psi \).
Collisions of two particles at location \(\vb{r}_1, \vb{r}_2\) are modelled by
\begin{equation}\label{eq: scalar-interactions}
    E_\mathrm{int} = \frac{1}{2}\int \int d\vb{r}_1d\vb{r}_2\,
    \psi^\dagger(\vb{r}_1)\psi^\dagger(\vb{r}_2)
    V_\mathrm{scat}(\vb{r}_1-\vb{r}_2)\psi(\vb{r}_1)\psi(\vb{r}_2),
\end{equation}
where \(V_\mathrm{scat}\) describes the two-body interaction potential.

In the s-wave scattering limit, which is applicable in a dilute gas at low
energies, we are free to neglect particles with high momenta
and instead replace the two-body potential with an effective contact interaction
between particles:
\begin{equation}
    V_\mathrm{scat}(\vb{r}_1 - \vb{r}_2) = g\delta(\vb{r}_1 - \vb{r}_2),
\end{equation}
where \(g=4\pi\hbar^2a_s/M\) is the interaction strength for s-wave
scattering length \(a_s\) and atomic mass \(M\).
Substituting the above contact potential into
Eq.~\eqref{eq: scalar-interactions} results in the scalar interaction
\begin{equation}
    E_\mathrm{scalar} = \frac{g_s}{2}\int d\vb{r} n^2,
\end{equation}
where we have used the relation \(n = \psi^\dagger \psi \) for condensate
density \(n\).

\section{Spinor BEC interactions}
Spinor systems comprise particles with total spin \(f\).
This implies there are \(2f + 1\) possible spin states along a given spin
quantisation axis.
Such a state is denoted \(\ket{f, m} \), where
\(m \in \{-f, -f+1, \ldots, 0, \ldots, f - 1, f\} \) denotes the magnetic
sublevel for total spin \( f\).

A general spin-\(f\) state is described by the (\(2f+1\))-component wave
function \(\psi \), defined as \(\ket{\psi} = \sum_m\psi_m\ket{f, m}\).
In the spin-1 case, we have \(\psi = {(\psi_1, \psi_0, \psi_{-1})}^T\), and for
the spin-2 case \(\psi = {(\psi_2, \psi_1, \psi_0, \psi_{-1}, \psi_{-2})}^T\).

Two identical spin-\(f\) bosons (atoms with integer spin) can collide to form a
total spin of \(\EuScript{F} = 0, 2, \ldots, 2f\) depending on the orientation
of the particles, with \(\EuScript{M} \equiv m+m' \in
\{-\EuScript{F}, \ldots, \EuScript{F}\}\).
In the s-wave scattering limit, where the orbital angular momentum is zero,
the total spin of two interacting particles must be even.
Due to the rotational symmetry, the scattering lengths depend only on the total
spin \(\EuScript{F}\), implying there are \(f + 1\) different scattering lengths
\(a_0, a_2, \ldots, a_{2f}\).

In a spinor BEC, the effective contact interaction is generalised to include
contributions from all spin channels as
\begin{equation}\label{eq: spin-f-interaction-potential}
    V_\mathrm{int} = \delta(\vb{r_1} - \vb{r_2})
    \sum_{\EuScript{F}=0}^{2f}g_\EuScript{F}\vb{P}_\EuScript{F},
\end{equation}
where \(g_\EuScript{F}=4\pi\hbar^2a_\EuScript{F}/M\) and
\(\vb{P}_\EuScript{F}\) is the projection operator which projects a pair
of atoms into a total spin \(\EuScript{F}\) state, defined as
\begin{equation}
    \vb{P}_\EuScript{F} = \sum_{m=-\EuScript{F}}^{\EuScript{F}}
    \ket{\EuScript{F}, \EuScript{M}}\bra{\EuScript{F}, \EuScript{M}}.
\end{equation}
For a system of identical bosons, the sum over all projection operators gives
\begin{equation}\label{eq: completeness-relation}
    \sum_{\EuScript{F}=0}^{2f}\vb{P}_\EuScript{F} = 1.
\end{equation}


\subsection{Spin-1 interaction Hamiltonian}
To calculate the spin-dependent mean-field interaction potential, it is common
to relate the projection operators to products of the single-particle spin
operators.
For a spin-1 condensate, the composition law of the angular momentum
gives~\cite{Kawaguchi2012, StamperKurn2013}
\begin{equation}\label{eq: composition-law}
    \vb{F}_1 \cdot \vb{F}_2 = \frac{1}{2}\left[(\vb{F}_1 + \vb{F}_2)^2
    - \vb{F}_1^2 - \vb{F}_2^2\right] = \frac{1}{2}\EuScript{F}(\EuScript{F} + 1)
    -f(f+1),
\end{equation}
where \(\vb{F}_i\) is the spin operator for atom \(i\).
From here we can construct the relation \(\vb{F}_1 \cdot \vb{F}_2 = 
\sum_{\EuScript{F} = 0}^{2f}\lambda_\EuScript{F}\vb{P}_\EuScript{F}\), where
\(\lambda_\EuScript{F} = \frac{1}{2}\EuScript{F}(\EuScript{F} + 1)
-f(f+1)\).
In an \(f=1\) spinor condensate, only \(\EuScript{F} = 0\) or \(2\) channels
are allowed.
Therefore, we have
\begin{equation}\label{eq: spin-1-spin-relation}
    \vb{F}_1 \cdot \vb{F}_2 = \sum_{\EuScript{F} = 0}^{2f}
    \lambda_\EuScript{F}\vb{P}_\EuScript{F} = \vb{P}_2 - 2\vb{P}_0,
\end{equation}
and
\begin{equation}\label{eq: spin-1-completeness-relation}
    \sum_{\EuScript{F} = 0}^{2f} = 1 = \vb{P}_0 + \vb{P}_2.
\end{equation}
Using both Eq.~\eqref{eq: spin-1-spin-relation} and
Eq.~\eqref{eq: spin-1-completeness-relation}, one can re-write
Eq.~\eqref{eq: spin-f-interaction-potential} as
\begin{equation}
    V_\mathrm{int} = \delta(\vb{r}_1 - \vb{r}_2)
    (c_0 + c_1 \vb{F}_1 \cdot \vb{F}_2),
\end{equation}
where \(c_0 = (g_0+2g_2)/3\) and \(c_1=(g_2-g_0) / 3\).
Finally, the spin-1 interacting Hamiltonian is then given as
\begin{equation}\label{eq: spin-1-interacting-Hamiltonian}
    H_\mathrm{int} = \frac{1}{2}\int d\vb{r} (c_0n^2 + c_1|\vb{F}|^2).
\end{equation}

Here \(c_0\) is interpreted as the spin-independent interaction strength.
It becomes apparent from Eq.~\ref{eq: spin-1-interacting-Hamiltonian} that
\(c_0 > 0\) (repulsive interactions) is required for stability.
In the \(c_0 < 0\) limit (i.e., attractive interactions), the condensate would
favour becoming infinitely dense, and in the process would destroy itself.
Therefore, throughout this thesis we always consider \(c_0 > 0\).

The \(c_1\) term describes is the spin-dependent interaction strength.
Since \(c_0\) is constrained to be positive, \(c_1\) can take positive or
negative values.
When \(c_1 > 0\), the system energetically favours minimising the condensate
spin \(\vb{F}\), which are typically referred to as antiferromagnetic or polar
interactions.
Conversely, for \(c_1 < 0\) the system favours maximising the spin vector,
which are referred to as ferromagnetic interactions.

Finally, we note that since \(c_1\) is the difference of two s-wave scattering
lengths which are comparable in magnitude experimentally, the spin-dependent
interaction strength is usually much smaller than the spin-independent strength.

\subsection{Spin-2 interaction Hamiltonian}
For an \(f=2\) spinor condensate, we now have \(\EuScript{F}=0,2,4\) spin
channels.
For a spin-2 condensate, Eq.~\eqref{eq: composition-law} now gives
\begin{equation}
    \vb{F}_1 \cdot \vb{F}_2 = \sum_{\EuScript{F} = 0}^{2f}
    \lambda_\EuScript{F}\vb{P}_\EuScript{F} = -6\vb{P}_0-3\vb{P}_2+4\vb{P}_4.
\end{equation}
Using the above equation and the completeness relation in
Eq.~\eqref{eq: completeness-relation}, one can write \(\vb{P}_2\) and
\(\vb{P}_4\) in terms of \(\vb{P}_0\) and \(\vb{F}_1 \cdot \vb{F}_2\):
\begin{equation}
    \vb{P}_2 = \frac{1}{7}(4 - \vb{F}_1 \cdot \vb{F}_2 - 10\vb{P}_0), \qquad
    \vb{P}_4 = \frac{1}{7}(3 + \vb{F}_1 \cdot \vb{F}_2 + 3\vb{P}_0).
\end{equation}
From Eq.~\eqref{eq: spin-f-interaction-potential}, the spin-2 interaction
potential takes the form
\begin{equation}
    V_\mathrm{int}\delta(\vb{r}_1-\vb{r}_2)(c_0+c_1\vb{F}_1 \cdot \vb{F}_2 
    +c_2\vb{P}_0),
\end{equation}
where \(c_0=(4g_2+3g_4)/7\), \(c_1=(g_4-g_2)/7\), and
\(c_2=(7g_0-10g_2+3g_4)/7\).
Both \(c_0\) and \(c_1\) are the spin-independent and -dependent interaction
strengths, respectively, analogous to the spin-1 system.
Due to the extra degree of freedom in the spin-2 system, a new interaction term,
\(c_2\), describing the spin-singlet interaction arises.

The spin-2 interaction Hamiltonian is given as
\begin{equation}
    H_\mathrm{int} = \frac{1}{2}\int d\vb{r}\left(c_0n^2+c_1|\vb{F}|^2
    +c_2|A_{20}|^2\right),
\end{equation}
where \(|A_{20}|^2\) is the spin-singlet duo amplitude, defined in terms of
the condensate wave function as
\begin{equation}
    A_{20} = \frac{1}{\sqrt{5}}\left(\psi_0^2-2\psi_1\psi_{-1}
    +2\psi_2\psi_{-2}\right).
\end{equation}
For \(c_2 > 0\), it becomes energetically favourable to minimise the
spin-singlet amplitude, i.e., \(|A_{20}|^2 = 0\).
Conversely, for interactions where \(c_2 < 0\), the energy is minimised by
maximising the amplitude \(|A_{20}|^2=n/5\).

\subsection{Single-particle Hamiltonian}
We start by first introducing the single-particle Hamiltonian for spinor BECs.
We assume a system of identical bosons of mass \(M\) and spin \(f\).
When a magnetic field is applied to a spinor system, the field causes energy
shifts in spin components.
When this field is aligned along the spin quantisation axis, linear, \(p\), and
quadratic, \(q\), Zeeman shifts arise.
In such a case, the single-particle (non-interacting) Hamiltonian is given
by~\cite{Kawaguchi2012}
\begin{equation}\label{eq: single-particle-Hamiltonian}
    H_0 = \int d\vb{r}\sum_{m,m'=-f}^{f} \psi_m^\dagger \left[
        -\frac{\hbar^2}{2M}\nabla^2 + V(\vb{r}) - pm + qm^2\right]\psi_{m'},
\end{equation}
where \(V(\vb{r})\) is a trapping potential.
Throughout most of this thesis we assume that the magnetic field applied is
uniform, such that the Zeeman shifts are constant.
In Chapter~\ref{chap: spin-1}, however, we introduce a time-dependent field
such that the quadratic Zeeman shift has a time-dependence.
Despite the time-dependence, the quadratic Zeeman shift is still spatially
uniform.

It becomes apparent from Eq.~\eqref{eq: single-particle-Hamiltonian} that the
quadratic shift uniformly alters the energetic separation between the
\(\psi_{\pm 1}\) and \(\psi_0\) components.
Conversely, the linear shift maintains the separation.

\section{Spinor Gross-Pitaevskii equations}

\section{Conserved quantities}

\section{Need to include healing lengths}