\chapter{Spinor Bose-Einstein condensates}

In this chapter we provide the theoretical background necessary to understand
the dynamics of ultracold atomic gases.
In particular, we introduce the Hamiltonians of both scalar and spinor BEC
systems.
We start with the simplest case of the scalar BEC, which has one non-linear
interaction term.
Then, by introducing the spin degrees of freedom, we introduce the extra
interactions present in spinor BECs.

\section{Scalar BEC interactions}
The many-body nature of BEC systems leads to interactions that arise between
particles.
We begin by first discussing the interactions arising in the simplest BEC\@: the
scalar BEC\@.
We consider a gas of identical bosons, described by the complex order parameter,
\(\psi \).
Collisions of two particles at location \(\vb{r}_1, \vb{r}_2\) are modelled by
\begin{equation}\label{eq: scalar-interactions}
    E_\mathrm{int} = \frac{1}{2}\int \int d\vb{r}_1d\vb{r}_2\,
    \psi^\dagger(\vb{r}_1)\psi^\dagger(\vb{r}_2)
    V_\mathrm{scat}(\vb{r}_1-\vb{r}_2)\psi(\vb{r}_1)\psi(\vb{r}_2),
\end{equation}
where \(V_\mathrm{scat}\) describes the two-body interaction potential.

In the s-wave scattering limit, which is applicable in a dilute gas at low
energies, we are free to neglect particles with high momenta
and instead replace the two-body potential with an effective contact interaction
between particles:
\begin{equation}
    V_\mathrm{scat}(\vb{r}_1 - \vb{r}_2) = g\delta(\vb{r}_1 - \vb{r}_2),
\end{equation}
where \(g=4\pi\hbar^2a_s/M\) is the interaction strength for s-wave
scattering length \(a_s\) and atomic mass \(M\).
Substituting the above contact potential into
Eq.~\eqref{eq: scalar-interactions} results in the scalar interaction
\begin{equation}
    E_\mathrm{scalar} = \frac{g_s}{2}\int d\vb{r} n^2,
\end{equation}
where we have used the relation \(n = \psi^\dagger \psi \) for condensate
density \(n\).

\section{Spinor BEC interactions}
Spinor systems comprise particles with total spin \(f\).
This implies there are \(2f + 1\) possible spin states along a given spin
quantisation axis.
Such a state is denoted \(\ket{f, m} \), where
\(m \in \{-f, -f+1, \ldots, 0, \ldots, f - 1, f\} \) denotes the magnetic
sublevel for total spin \( f\).

A general spin-\(f\) state is described by the (\(2f+1\))-component wave
function \(\psi \), defined as \(\ket{\psi} = \sum_m\psi_m\ket{f, m}\).
In the spin-1 case, we have \(\psi = {(\psi_1, \psi_0, \psi_{-1})}^T\), and for
the spin-2 case \(\psi = {(\psi_2, \psi_1, \psi_0, \psi_{-1}, \psi_{-2})}^T\).

Two identical spin-\(f\) bosons (integer spin) can collide to form a total spin
of \(\EuScript{F} = 0, 2, \ldots, 2f\) depending on the orientation of the
particles, with \(\EuScript{M} \equiv m+m' \in
\{-\EuScript{F}, \ldots, \EuScript{F}\}\).
In the s-wave scattering limit, where angular momentum is strictly conserved,
the total spin of two interacting particles must be even.
Due to the rotational symmetry, the scattering lengths depend only on the total
spin \(\EuScript{F}\), implying there are \(f + 1\) different scattering lengths
\(a_0, a_2, \ldots, a_{2f}\).

In a spinor BEC, the effective contact interaction is generalised to include
contributions from all spin channels as
\begin{equation}
    V_\mathrm{int} = \delta(\vb{r_1} - \vb{r_2})\frac{4\pi\hbar^2}{2m}
    \sum_{\EuScript{F}=0, 2, \ldots, 2f}a_\EuScript{F}\vb{P}_\EuScript{F},
\end{equation}
where \(\vb{P}_\EuScript{F}\) is the projection operator which projects a pair
of atoms into a total spin \(\EuScript{F}\) state, defined as
\begin{equation}
    \vb{P}_\EuScript{F} = \sum_{m=-\EuScript{F}}^{\EuScript{F}}
    \ket{\EuScript{F}, \EuScript{M}}\bra{\EuScript{F}, \EuScript{M}}.
\end{equation}
For identical bosons, \(\sum_\EuScript{F=0}^{2f}\vb{P}_\EuScript{F} = 1\).

\subsection{Spin-1 interaction Hamiltonian}
To calculate the spin-dependent mean-field interaction potential, it is common
to re-write the projection operators as products of the single-particle spin
operators.
For a spin-1 condensate, the decompositions are given
as~\cite{Kawaguchi2012, StamperKurn2013}
\begin{equation}
    \vb{P}_0 = \frac{1}{3}(1 - \vb{F}_1 \cdot \vb{F}_2), \qquad
    \vb{P}_2 = \frac{1}{3}(2 + \vb{F_1 \cdot \vb{F}_2}).
\end{equation}

\subsection{Spin-2 interaction Hamiltonian}

\subsection{Single-particle Hamiltonian}
We start by first introducing the single-particle Hamiltonian for spinor BECs.
We assume a system of identical bosons of mass \(M\) and spin \(f\).
When a magnetic field is applied to a spinor system, the field causes energy
shifts in spin components.
When this field is aligned along the spin quantisation axis, linear, \(p\), and
quadratic, \(q\), Zeeman shifts arise.
In such a case, the single-particle (non-interacting) Hamiltonian is given
by~\cite{Kawaguchi2012}
\begin{equation}\label{eq: single-particle-Hamiltonian}
    H_0 = \int d\vb{r}\sum_{m,m'=-f}^{f} \psi_m^\dagger \left[
        -\frac{\hbar^2}{2M}\nabla^2 + V(\vb{r}) - pm + qm^2\right]\psi_{m'},
\end{equation}
where \(V(\vb{r})\) is a trapping potential.
Throughout most of this thesis we assume that the magnetic field applied is
uniform, such that the Zeeman shifts are constant.
In Chapter~\ref{chap: spin-1}, however, we introduce a time-dependent field
such that the quadratic Zeeman shift has a time-dependence.
Despite the time-dependence, the quadratic Zeeman shift is still spatially
uniform.

It becomes apparent from Eq.~\eqref{eq: single-particle-Hamiltonian} that the
quadratic shift uniformly alters the energetic separation between the
\(\psi_{\pm 1}\) and \(\psi_0\) components.
Conversely, the linear shift maintains the separation.

\section{Spinor Gross-Pitaevskii equations}

\section{Conserved quantities}

\section{Need to include healing lengths}