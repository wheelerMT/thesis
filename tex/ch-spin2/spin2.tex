\chapter{Vortex connections across topological interfaces}\label{chap: spin-2}
\section{To do}
\begin{itemize}
    \item 
\end{itemize}


Topological interfaces may form at the phase boundary between topologically
distinct phases that are described by different order parameters.
In such an interface, the phases can exist in a coherent and order medium, in
which the different symmetries connect smoothly across the boundary.
Such interfaces arise in many areas of physics, such as the context of domain
walls in the early universe~\cite{Zeldovich1975,Kibble1976,Kibble1980}, where
they may form the termination points of cosmic strings~\cite{Vilenkin1985},
brane models in superstring theory~\cite{Dvali1999,Sarangi2002,Gudnason2015},
the \(A\)--\(B\) phase boundary in superfluid liquid \(^3\)He~\cite{
    Osheroff1977,Yip1986,Salomaa1987,Finne2006,Bradley2007,Volovik2009}, via
atomic Bose-Einstein condensates (BECs)~\cite{Takeuchi2006,Kasamatsu2010,
    Borgh2012,Borgh2013, Borgh2014,Kaneda2014}, and even quantum
chromodynamics~\cite{Alford2001,Cipriani2012,Eto2014}.
The differing order parameter symmetry on each side of the interface implies
that families of vastly different defects and textures may exist, and therefore
cannot cross the interface unchanged.
Instead, the defect must either terminate at the boundary, or continuously
and non-trivially connect to an object representing a different topology on the
other side of the interface.

Since Spinor BECs carry a rich phase diagram which support a wide family of
topological defects exhibiting different order parameter symmetries, they become
an ideal test bed for investigating interface physics~\cite{Borgh2012,Borgh2013,
    Borgh2014}.
Such defects include singular vortices carrying both integer~\cite{Yip1999,
    Isoshima2002,Mizushima2002a,Zhou2003,Sadler2006,Semenoff2007,Kobayashi2009,
    Lovegrove2012,Lovegrove2016,Borgh2016a,Weiss2019,Xiao2021,Xiao2022} and
fractional~\cite{Leonhardt2000,Ji2008,Seo2015,Semenoff2007,Lovegrove2012,
    Lovegrove2016,Borgh2016a,Borgh2017,Xiao2021,Xiao2022} charges, as well as
non-singular vortices (2D Skyrmions)~\cite{Ohmi1998, Ho1998, Mizushima2002,
    Martikainen2002, Leanhardt2003, Mizushima2004, Choi2012, Choi2012a,
    Lovegrove2014,Weiss2019}, wall-vortex complexes~\cite{Kang2019,
    Takeuchi2021}, and monopoles~\cite{Stoof2001,
    Savage2003,Ruostekoski2003,Pietila2009,Ray2014,Ray2015,Ollikainen2017,
    Mithun2022}.
In addition, non-Abelian defects, whose charges depend on other defects within
the system, may arise when the medium on either or both sides of the interface
exhibits point-group order parameter symmetry~\cite{Xiao2022}.

In this chapter, we both analytically and numerically investigate the physics
of topological defects when connected across topological interfaces in spin-2
Bose-Einstein condensates.
We demonstrate that a particularly rich phenomenology of topological defects
at the coherent
interface between regions of different broken symmetries can be realised in
spin-2 Bose-Einstein condensates.
In particular, we propose that an interface between uniaxial and biaxial nematic
phases exhibiting continuous and discrete order-parameter symmetry,
respectively, can be realised using existing experimental techniques.
We construct spinor wave functions that represent topological defects connecting
across the interface.
By numerical energy relaxation as well as simulations of dynamics, we
characterise the emergence of non-trivial defect core structures.
We further demonstrate the emergence of composite vortex-core structures and
continuous connection of fractional vortices representing non-Abelian charges in
interfaces involving the cyclic and ferromagnetic phases, which could be
realised through manipulation of the inter-atomic interactions.
Our results suggest the spin-2 Bose-Einstein condensates as experimentally
accessible test beds for interface physics with all combinations of continuous
and discrete symmetries, as well as phases supporting non-Abelian defects.

\section{Interface crossing solutions in a spin-2 BEC}
\textcolor{red}{Move this paragraph later on, focus on interfaces in the BEC
itself first.}
In a continuous spinor superfluid, it is possible for distinct phases with
varying order parameter symmetry to coexist.
This situation can arise, for example, due to energy relaxation in vortex core
structures~\cite{Ruostekoski2003,Lovegrove2012,Lovegrove2016,Borgh2016,
    Borgh2016a, Weiss2019,Xiao2021,Xiao2022}.
In a spin-2 BEC, the three contributions to the interaction energy each give
rise to a healing length, which describe, respectively, the distance over which
perturbations of the superfluid density, condensate spin, and singlet duo
amplitude heal to the bulk value:
\begin{equation}
    \xi_d=\ell \sqrt{\frac{\hbar\omega}{2nc_0}},
    \enskip \xi_F=\ell \sqrt{\frac{\hbar\omega}{2n|c_1|}},
    \enskip \xi_a=\ell \sqrt{\frac{\hbar\omega}{2n|c_2|}},
    \label{eq: spin-2-healing-lengths}
\end{equation}
where \(\ell = {(\hbar/M\omega)}^{1/2}\) is the harmonic oscillator length.
Typically, \(\xi_F,\xi_a > \xi_d\), allowing the core of singular vortices
to reduce its energy by expanding and filling with a different superfluid
phase~\cite{Ruostekoski2003}.
The condensate wave function then smoothly interpolates between the coexisting
phases in the bulk superfluid and the defect core, establishing a topological
interface between them.

A spinor BEC may be engineered such that separate regions of the condensate
have distinct ground state phases, e.g., nematic or cyclic, and hence have
different order parameter symmetries, creating a topological interface within
the condensate itself.
This can be achieved through variation of parameters in the Hamiltonian (see
\textcolor{red}{Link to spin-2 Hamiltonian}), such as the linear and quadratic
Zeeman shifts, which locally stabilise the different regions.
If the parameter fluctuation is sufficiently acute, then the wave function will
interpolate between the bulk phases across a distance given by an appropriate
healing length.
In addition, as proposed for the spin-1 system~\cite{Borgh2014}, linear \(p\)
and quadratic \(q\) Zeeman shifts form the basis for engineering interpolating
solutions that model the topological interface.
Such steady-state solutions are found from the mean-field Hamiltonian density
functional in Eq.~(\textcolor{red}{No full spin-2 Hamiltonian in paper, either
    put here or back in Chapter 2}) for a uniform system in the presence of an
external magnetic field using the relation \(\delta \mathcal{H}
/ \delta \zeta_m^*=0\)~\cite{Kawaguchi2012}:
\begin{equation}\label{eq: spin-2-GPEs-time-reversal}
    \left[-p \hat{F}_z + q \hat{F}^2_z + c_0n \, \zeta^\dagger \zeta
        + c_1n \, \langle\hat{\vb{F}}\rangle\cdot\hat{\vb{F}}
        + \frac{c_2n}{5}{(\hat{\mathcal{T}}\zeta)}^\dagger \,
        \zeta\hat{\mathcal{T}} -\mu \right] \zeta = 0,
\end{equation}
where \(\mu \) is the chemical potential and \(\hat{\mathcal{T}}\) is a
time-reversal operator defined as \({(\hat{\mathcal{T}}\zeta)}_m =
{(-1)}^m\zeta_{-m}^*\).
Eq.~\eqref{eq: spin-2-GPEs-time-reversal} presents a non-linear system of
equations that can be solved for the unknown
\(\zeta_m\)~\cite{Ciobanu2000,Kawaguchi2012}.

Using the constructed interpolating stationary states, defect states for a given
spin-2 phase can, in principle, be constructed from
the representative spinor by defining an appropriate azimuthal phase winding,
\(\chi_m \), in each component, i.e., \(\text{Arg}(\zeta_m) =\chi_m =
k_m\varphi \), where \(k_m\) is an integer and \(\varphi \) is the azimuthal
angle around the vortex core.
In addition, we also create a unitary framework which is applicable for
constructing other types of defect states across topological interfaces in a
spin-2 BEC, such as monopoles or textures.
A given defect state can be constructed from the uniform state by applying a
global phase winding, \(\tau \), coupled with a spin rotation defined by three
Euler angles \((\alpha, \beta, \gamma)\) as in
Eq.~\eqref{eq: spin-2-rotation-matrix}.
When the same transformation is supported by two phases, A and B, of a spin-2
condensate, a general defect connection across an interface between the two
phases is given as
\begin{align}\label{eq: general-defect-interface}
    \zeta = e^{i\tau}\,\hat{U}(\alpha, \beta, \gamma)\,\zeta^\text{A-B}.
\end{align}
Through the subsequent sections we provide various steady-state solutions that
interpolate between the different phases of a spin-2 BEC.\@
Using these solutions, we then construct interesting defect states that connect
across the topological interface.
Detailed derivations of each interpolating solution used in the subsequent
sections can be found in Appendix~\ref{appendix: stationary}.

\subsection{Uniaxial nematic to biaxial nematic}\label{subsec: UN-BN-defects}
We first focus on a family of solutions interpolating between the UN and BN
phases (see Sec~\ref{sec: ground-states-spin-2} for details on these ground
states).
Such an interpolating solution is given as
\begin{equation}\label{eq: UN-BN-interpolating-spinor}
    \zeta^\text{UN-BN} = \frac{1}{2}\mqty(
    e^{i\chi_2}\sqrt{1 - \eta} \\
    0 \\
    e^{i\chi_0}\sqrt{2(1+\eta)} \\
    0 \\
    e^{i\chi_{-2}}\sqrt{1 - \eta}
    ),
\end{equation}
where \(\eta = 10q /|c_2|n \in [-1, 1]\) and \(\chi_m = \text{Arg}(\zeta_m)\)
for component \(m\).
Here, \(\chi_m\) are arbitrary phase coefficients that can either take fixed
values, or be spatially wound in order to produce different vortex states.
This solution depends only on the quadratic Zeeman shift, which can alter the
spinor between the different phases: When \(q = c_2n / 10\) the system is in the
UN phase with only the \(\zeta_0=1\) component occupied, where the nematic
director is aligned with the \(z\)-axis.
In the opposite limit, when \(q = -c_2n/10\), the system is in the
BN phase with the \(\zeta_{\pm 2} = 1/\sqrt{2}\) components occupied.
This spinor therefore provides interpolating solutions between the UN and BN
phases, engineered through manipulation of the quadratic Zeeman shift.

To determine the energetic stability of this interpolating spinor, we compare
the energy per particle given by~\cite{Kawaguchi2012}
\begin{equation}
    E^\text{UN-BN} = \mathcal{H}[\Psi^\text{UN-BN}] - \frac{c_0n}{2},
\end{equation}
where \(\mathcal{H} = \mathcal{H}_0 + \mathcal{H}_\text{int}\) is the spin-2
Hamiltonian density where \(\mathcal{H}_0\) and \(\mathcal{H}_\text{int}\)
are defined in Eq.~\eqref{eq: single-particle-Hamiltonian} and
Eq.~\eqref{eq: spin-2-interacting-Hamiltonian}, respectively.
The energy of the interpolating spinor given in
Eq.~\eqref{eq: UN-BN-interpolating-spinor} reads
\(E^\text{UN-BN} = \frac{c_2n}{10} + 2q(1 - \eta)\)
Comparing this energy with that of the UN phase
\(E^\text{UN} = c_2n/10\) and the BN phase
\(E^\text{BN} = c_2n/10 + 4q\) reveals that the ground state is UN for
\(q \geq 0\) and BN for \(q \leq 0\).
Therefore, the UN-BN interface can be stabilised through careful choice of
a longitudinal quadratic Zeeman shift \(q(z)\) such that \(q(z)\) changes
sign at some transverse plane, which we typically take to be \(z=0\).

A consequence of a spatially-dependent \(\eta \) is revealed from the spin
singlet-duo and -trio amplitudes, given in terms of the spinor \(\zeta_m\),
respectively, as
\begin{gather}\label{eq: a20definition}
    |A_{20}|^2 = \frac{1}{5}\left|2\zeta_2\zeta_{-2}-2\zeta_1\zeta_{-1}
    + \zeta_0^2\right|^2, \\
    |A_{30}|^2 = \left|\frac{3\sqrt{6}}{2}\left(\zeta_1^2\zeta_{-2}
    + \zeta_{-1}^2\zeta_2\right)
    + \zeta_0\left(\zeta_0^2-3\zeta_1\zeta_{-1}
    - 6\zeta_2\zeta_{-2}\right)\right|^2.
\end{gather}
Upon substitution of Eq.~\eqref{eq: UN-BN-interpolating-spinor} into the above,
we get
\begin{equation}
    \begin{aligned}
        |A_{20}|^2 & = \frac{1}{10} \left[(\eta^2-1)\cos\theta
        + \eta^2 + 1\right],                                          \\
        |A_{30}|^2 & = \frac{1+\eta}{4} \left[3\left(\eta ^2-1\right)
            \cos\theta
            + \eta(5 \eta -8) + 5\right],
    \end{aligned}
\end{equation}
where \(\chi = \chi_2 + \chi_{-2} - 2\chi_0\) is the relative phase difference
between the components.
We plot both the singlet-duo and -trio amplitudes in
Fig.~\ref{fig: UN-BN-duo-trio} in a parameter space of \((\eta, \theta)\).
\begin{figure}
    \centering
    \includegraphics[width=0.8\textwidth]{gfx/ch-spin2/a20-a30-varying.png}
    \caption[Spin-singlet duo and trio amplitudes in a parameter space of
        \(\chi \) and \(\eta \)]
    {\label{fig: UN-BN-duo-trio} Spin singlet-duo (left) and -trio
        (right) amplitudes for the interpolating spinor in
        Eq.~\eqref{eq: UN-BN-interpolating-spinor}.
        Due to the spatially-dependent \(\eta \), this spinor continuously
        interpolates between the UN, BN, and cyclic phases depending on the
        relative phase difference \(\chi=\chi_2+\chi_{-2}-2\chi_0\).}
\end{figure}
Upon investigation, we see interesting behaviour arise in both quantities.
In particular, we see that this spinor can interpolate between the different
phases of UN, BN and cyclic depending on both \(\eta \) and the relative phase
difference \(\chi \).
For example, if one were to maintain \(\theta=0\) and interpolate \(\eta \),
then there would be multiple transitions between the UN (\(|A_{20}|^2 = 1\)) and
BN (\(|A_{20}|^2 = 0\)) phases.
In addition, maintaining a relative phase difference of \(\chi=\pi \), the
singlet-trio amplitude reveals that a cyclic phase is present when
\(-0.5 \lesssim \eta \lesssim 0.5\).
As we shall see in our numerical investigations, this has profound effects on
the structure of topological defects connecting across such an interface.

To begin our investigations of defects connecting across topological interfaces,
we first consider vortex connections.
A summary of different vortex connections possible across a UN-BN interface are
listed in Table~\ref{tab: UN-BN-vortices}.
\begin{table}
    \centering
    \begin{tabular}{cccccc}
        \toprule
        \multicolumn{5}{c}{Uniaxial to Biaxial Nematic (UN-BN) --- Vortices} \\
        \midrule
        UN limit & BN limit &  \(\chi_2/\varphi \) & \(\chi_0/\varphi \) &
        \(\chi_{-2}/\varphi \)  \\
        \midrule
         Singular phase vortex & Singular phase vortex & \(k\) & \(k\) &
         \(k\) \\ 
         Vortex-free & Singular phase vortex & \(k\) & 0 & \(k\) \\
         Singular phase vortex & Vortex-free & 0 & \(k\) & 0\\
         Vortex-free & Singular spin vortex  & \(-k\) & 0 & \(k\) \\
         Singular phase vortex & Singular spin vortex  & \(-k\) & \(k\) &
         \(k\) \\
         Vortex-free & Half-quantum vortex  & \(k\) & 0 & 0\\
         Singular phase vortex & Half-quantum vortex  & \(k\) & \(k\) & 0 \\
        \bottomrule
    \end{tabular}
    \caption{\label{tab: UN-BN-vortices}
    Representative examples of different vortex connections possible across a
    UN-BN interface, constructed from the winding of the phase coefficients
    \(\chi_m\) in Eq.~\eqref{eq: UN-BN-interpolating-spinor}.
    Additionally, generalisations to multiply quantised vortices are given by
    \(k \in \mathbb{Z}\).}
\end{table}
We start by constructing a connection of singly quantised vortices (SQVs) on
either side of the interface, which can be achieved by allowing
\(\chi_m=\varphi \) in Eq.~\eqref{eq: UN-BN-interpolating-spinor}, which results
in a spatially overlapping \(2\pi \) phase winding in each component.
An alternative construction is achieved using
Eq~\eqref{eq: general-defect-interface}, where instead \(\tau=\varphi \) and the
Euler angles are kept constant.
\begin{align}\label{eq: UN-BN-SQV-SQV}
    \zeta_\text{sqv}^\text{UN-BN} & =
    \frac{1}{\sqrt{2}}\left( D_+\zeta^\text{UN}_\text{sqv}
    + D_-\zeta^\text{BN}_\text{sqv}\right),                       \\
    \zeta^\text{UN}_\text{sqv}    & =
    \sqrt{\frac38}e^{i\varphi}
    \begin{pmatrix}
        \sin^2\beta_0                                   \\
        \sin (2 \beta _0)                               \\
        \frac{1}{\sqrt{6}}\left(1+3\cos^2\beta_0\right) \\
        - \sin (2 \beta _0)                             \\
        \sin^2\beta_0
    \end{pmatrix}, \\
    \zeta^\text{BN}_\text{sqv}    & =
    \frac{e^{i\varphi}}{\sqrt{8}}
    \begin{pmatrix}
        \cos^2\beta_0 + 1     \\
        - \sin (2 \beta _0)   \\
        \sqrt{6}\sin^2\beta_0 \\
        \sin (2 \beta _0)     \\
        \cos^2\beta_0 + 1
    \end{pmatrix},
\end{align}
where \(D_{\pm} = \sqrt{1 \pm \eta}\) and \(\beta=\beta_0\) is an arbitrary
constant.
One can see that upon the substitution of \(\beta_0 = 0\) in the above spinors,
we recover the SQV case in both the UN
\(\zeta^\mathrm{SQV} = {(0,0,e^{i\varphi},0,0)}^T\) and BN
\(\zeta^\mathrm{SQV} = {(e^{i\varphi},0,0,0,e^{i\varphi})}^T/\sqrt{2}\) limits.
It is important to note that, despite being characterised by the same phase
winding, the SQVs on either side of the interface represent entirely different
objects due to the differing topologies of the UN and BN phases.

One can, instead, construct a defect crossing solution that connects a vortex
on one side of the interface to a vortex-free region on the other side.
By selectively removing the winding (equivalent to setting \(\varphi=0\)) from,
say, \(\zeta^\text{UN}_\text{sqv}\), we are left with an SQV on the BN side
which smoothly interpolates to a vortex-free region in the UN phase.
Similarly, removing the winding from \(\zeta^\text{BN}_\text{sqv}\) connects an
SQV on the UN side to a vortex-free region on the BN side.
These interpolating solutions parameterise the filling of vortex cores when
\(\xi_a > \xi_d\).
In such a case, we now have a radial dependence on the interpolating parameters
\(D_{\pm}\).
For example, for an SQV in the BN phase can be predicted to fill with atoms in
the UN phase when \(D_-(0) = 0, D_+(0) = \sqrt{2}\) and change back to the BN
phase away from the vortex core \(D_-(\vb{r}_\perp) = \sqrt{2}, D_+(0) = 0\).
\textcolor{red}{Need to get my head around this as it is not immediately clear
    to me.}

Spin vortices, which carry a spin circulation only
(see Sec.~\ref{sec: vortices-spin-2}), can be connected across the interface by
choosing \(\alpha=\varphi, \beta=\beta_0 \) and \(\tau = \gamma = 0\) in
Eq.~\eqref{eq: general-defect-interface}, resulting in the spinor
\begin{align}\label{eq: UN-BN-SV-SV-spinor}
     & \zeta_\text{sv} =  \frac{1}{\sqrt{2}}\left( D_+\zeta^\text{UN}_\text{sv}
    + D_-\zeta^\text{BN}_\text{sv}\right),
    \\
     & \zeta^\text{UN}_\text{sv} =
    \sqrt{\frac38}
    \begin{pmatrix}
        e^{-2i\varphi}\sin^2\beta_0                     \\
        e^{-i\varphi}\sin 2 \beta _0                    \\
        \frac{1}{\sqrt{6}}\left(1+3\cos 2\beta_0\right) \\
        -e^{i\varphi} \sin 2 \beta _0                   \\
        e^{2i\varphi}\sin^2\beta_0
    \end{pmatrix}, \label{eq: UN-SV-spinor}               \\
     & \zeta^\text{BN}_\text{sv} =
    \frac{1}{\sqrt{8}}
    \begin{pmatrix}
        e^{-2i\varphi}\left(\cos^2\beta_0 + 1\right) \\
        - e^{-i\varphi}\sin 2 \beta _0               \\
        \sqrt{6}\sin^2\beta_0                        \\
        e^{i\varphi}\sin 2 \beta _0                  \\
        e^{2i\varphi}\left(\cos^2\beta_0 + 1\right)
    \end{pmatrix},
\end{align}
In particular, spin vortices arise in both components for
\(\beta_0 \neq 0, \pi \), where the spin vortex in each phase carries opposite
the spin winding of the other spin vortex.
For the specific cases of \(\beta_0 = 0, \pi \),
Eq.~\eqref{eq: UN-BN-SV-SV-spinor} instead results in a spin vortex in the BN
phase that continuously connects to a vortex-free state in the UN limit.

In addition to line defects, i.e., vortices, it is also possible to construct
point defects that terminate at the topological interface.
The possibility in the UN-BN interface is that of a monopole, which is
characterised by a spherically symmetric dependence of the nematic axis
\(\vb{\hat{d}} = (\cos\varphi \sin\theta, \sin\varphi \sin\theta, \cos\theta)\),
where \(\theta \) is the polar angle in spherical coordinates.

A monopole solution in the UN phase is derived from Eq.~\eqref{eq: UN-SV-spinor}
using the substitution \(\beta = \theta \) and \(\alpha=\varphi \).
Applying this same rotation to the interpolating spinor in
Eq.~\eqref{eq: UN-BN-SV-SV-spinor} results in a point monopole in the UN limit
which terminates as a hedgehog structure of the nematic axis surrounding a
singular string in the BN phase.
\textcolor{red}{Would quite like an image of the initial state somehow.}

\subsection{Cyclic to nematic}
A steady-state solution that interpolates between the nematic and cyclic phases
can be found by choosing \(\chi = \chi_2 + \chi_{-2} + 2\chi_0 = \pi \) in
Eq.~\eqref{eq: UN-BN-interpolating-spinor}, namely
\begin{equation}\label{eq: C-N-interpolating-spinor}
    \zeta^\mathrm{C-N} = \mqty(
    e^{i\chi_2}D_- \\
    0 \\
    ie^{i\chi_0}\sqrt{2}D_+ \\
    0 \\
    e^{i\chi_{-2}}D_-),
    \quad D_\pm = \sqrt{1\pm\eta},
\end{equation}
where \(\eta = 10q/(|c_2|n)\).
The quadratic Zeeman shift can be used to interpolate the above solution between
the different phases.
At \(q = 0\) the above solution becomes the three component cyclic
\(\zeta_\mathrm{C} = {(1, 0, i\sqrt{2}, 0, 1)}^T/2\).
The sign of the quadratic Zeeman shift determines which nematic state is chosen.
If \(q = -c_2n/10\) then the solution becomes biaxial nematic
\(\zeta_\mathrm{BN} = {(1, 0, 0, 0, 1)}^T/\sqrt{2}\).
In the opposite limit of \(q = c_2n/10\) then the system is in the uniaxial
nematic state \(\zeta_\mathrm{UN} = {(0, 0, 1, 0, 0)}^T\).

The uniform mean-field energy of Eq.~\eqref{eq: C-N-interpolating-spinor} reads
\begin{equation}
    E^\text{C-N} = \frac{c_2n}{10}\eta^2 + 2q(1 - \eta).
\end{equation}
The energy becomes minimised at \(\eta = 10q/c_2n\) for \(c_2 > 0\) and fixed
\(q\).
Since the interpolating spinor \(\zeta^\text{C-N}\) is a function of \(q\) only
the stability of the interface is guaranteed since \(E^\text{C-N} =
2q-10q^2/c_2 \leq E^\text{C}, E^\text{UN},
E^\text{BN}\) for \(|q| < c_2n/10\).

Here, for simplicity, we focus on an interface between the cyclic and BN phases
which is achieved from a longitudinally-dependent quadratic Zeeman shift
\(q(z)\) which varies from \(q = 0\) at the cyclic phase to \(q = -c_2n/10\)
at the BN side.

Firstly, one can construct the SQV-SQV and SV-SV connections considered in the
UN-BN interface for this interface upon the substitution
\(D_+ \rightarrow iD_+\) in Eq.~\eqref{eq: UN-BN-SQV-SQV} and identifying the
cyclic limit as \(D_+ = 1\) and \(D_- = 0\).

In addition to these connections, one can also construct an alternative
connection between spin vortices that arises due to a shared symmetry between
the cyclic and BN phases.
In particular, the common axis of symmetry is given by \(\hat{\vb{e}}_{(1,1,0)}=
(\hat{\vb{x}} + \hat{\vb{y}}) / \sqrt{2}\) (see \(C_2'\) axis in
Fig.~\ref{fig: spin-2-spherical-harmonics}d).
A spin rotation around this axis is given by
\begin{equation}\label{eq: C-BN-sv-rot}
    \hat{U}\left(\hat{\vb{e}}_{(1,1,0)}, \delta\right) =
    \exp{-i\frac{\hat{F}_x + \hat{F}_y}{\sqrt{2}}\delta},
\end{equation}
where \(\delta \) is an angle.
This spin rotation with \(\delta = \varphi \), along with \(\tau=0\), can then
be applied to the interpolating in Eq.~\eqref{eq: C-N-interpolating-spinor} to
produce
\begin{gather}
    \zeta_\mathrm{sv} = \frac{1}{\sqrt{2}}\left(
    iD_+\zeta^\mathrm{UN}_\mathrm{sv} +
    D_-\zeta^\mathrm{BN}_\mathrm{sv}
    \right), \label{eq: C-BN-SV-SV-spinor} \\
    \zeta^\mathrm{UN}_\mathrm{sv} = \sqrt{\frac{3}{8}}
    \mqty(
    \sin^2 \varphi \\
    e^{\frac{i\pi }{4}} \sin 2\varphi \\
    \frac{i}{\sqrt{6}}\left(1 + 3\cos 2\varphi \right)\\
    e^{-\frac{i \pi }{4}} \sin2 \varphi \\
    -\sin^2 \varphi
    ), \\
    \zeta^\mathrm{BN}_\mathrm{sv} = \frac{1}{\sqrt{2}}
    \mqty(
    \cos \varphi \\
    e^{-\frac{3i\pi }{4}} \sin \varphi \\
    0\\
    e^{-\frac{i \pi }{4}} \sin \varphi \\
    \cos \varphi
    ),
\end{gather}
which yields integer spin vortices in the cyclic and both nematic limits.
The same spinor can be constructed from Eq.~\eqref{eq: spin-rotation-action}
with \(\alpha=-\gamma=\pi/4\) and \(\beta = \varphi \).

\subsection{Cyclic to ferromagnetic}
An additional stationary solution can be found by considering \(D_1=D_2=0\).
Eq.~\eqref{eq: D2} and Eq.~\eqref{eq: D1} can then be solved exactly, to yield
\begin{equation}
    \tilde{\mu} = \pm \sqrt{4q^2 + |\alpha|^2}.
\end{equation}
If \(q \neq 0\), then \(\tilde{\mu} \neq \alpha \) which implies \(\zeta_0=0\)
from Eq.~\eqref{eq: spin-2-stationary-zeta0-recast}.
Zero transverse magnetisation then implies we have
\(0 = 2c_1n(\zeta_2^*\zeta_{1} + \zeta_{-1}^*\zeta_{-2})\).
\textcolor{red}{Rest of derivation here.}

The final solution reads
\begin{equation}\label{eq: C-FM-interpolating-spinor}
    \zeta^\text{C-FM} = \frac{1}{\sqrt{3}}\mqty(
    e^{i\chi_2}D_2 \\
    0 \\
    0 \\
    e^{i\chi_{-1}}D_{-1} \\
    0
    ), \qquad
    \mqty{
        D_2 = \sqrt{1 + f_z} \\
        D_{-1} = \sqrt{2 - f_z}
    }
    ,
\end{equation}
where \(f_z\) gives the \(z\)-component of the spin vector.
At zero magnetisation, \(f_z = 0\), this solution is precisely the two-component
cyclic state.
The solution also interpolates between different ferromagnetic states depending
on the value of the magnetisation.
If \(f_z = -1\) one recovers an FM-1 solution
\(\zeta_\text{FM-1} = {(0, 0, 0, 1, 0)}^T\).
Conversely, \(f_z = 2\) yields the FM-2 solution
\(\zeta_\text{FM-2} = {(1, 0, 0, 0, 0)}^T\).
This spinor therefore provides a family of interpolating solutions between the
cyclic and both ferromagnetic phases.

The uniform mean-field energy of the above spinor reads
\begin{equation}
    E^\text{C-FM} = \frac{c_1 n}{2} f_z^2 - (p-q)f_z +2q,
\end{equation}
which becomes minimised upon the choice \(f_z = (p-q)/c_1n\) for \(c_1 > 0\) and
fixed \(p, q\).
Therefore, assuming, e.g., \(q=0\), we can stabilise the interface by an
appropriate choice of a longitudinally-dependent linear Zeeman shift \(p(z)\).
The interface is expected to be stable within the range
\(-c_1n \leq p \leq 2c_1n\) since \(E^\text{C-FM} \leq
E^\text{C}=0, E^{f_z=1} = c_1n/2 + p\).

Applying a phase and a general spin rotation to the spinor in
Eq.~\eqref{eq: C-FM-interpolating-spinor} yields (\(\beta = 0\))
\begin{equation}\label{eq: C-FM-general-spinor}
    \zeta^\text{C-FM} = \frac{e^{i\varphi}}{\sqrt{3}}\mqty(
    e^{-2i(\alpha+\gamma)}D_2 \\
    0 \\
    0 \\
    e^{i(\alpha+\gamma)}D_{-1} \\
    0
    ),
\end{equation}
where \(D_2 = \sqrt{1 + f_z}\) and
\(D_{-1} = \sqrt{2 - f_z}\).
Now, combinations of \(\tau \) and \(\alpha+\gamma \) yields different phase
vortices.
Since the cyclic phase supports fractional vortices
(see Sec. \textcolor{red}{Section}), we are able to construct a myriad of
different connections from this interpolating spinor.
For example, the choice \(\alpha+\gamma=-\varphi/3\) and \(\tau=\varphi/3\)
results in a singular SQV in the FM-2 limit (\(D_2 = 1, D_{-1} = 0\)),
a third vortex in the cyclic limit (\(D_2 = \sqrt{1/3}, D_{-1} = \sqrt{2/3}\)),
and a vortex-free state in the FM-1 limit (\(D_2 = 0, D_{-1} = 1\)).
Instead, by choosing \(\alpha + \gamma = \varphi/3\) and \(\tau = 2\varphi/3\)
we have a vortex-free state in the FM-2 limit, a two-third vortex in the cyclic
limit, and a singular SQV in the FM-1 limit.
A case constructing SQVs in all three limits is given by \(\alpha+\gamma = 0\)
and \(\tau = \varphi \).

One can use the general spinor in Eq.~\eqref{eq: C-FM-general-spinor} to
analytically examine the vortex core structures when \(D_{2,{-1}}\) are
functions of the transverse radius \(\rho = \sqrt{x^2 + y^2}\).
The states can be described by choosing an appropriate function for
\(f_z(\rho)\) that interpolates between all three phases.
We choose \(f_z(\rho) = 3\tanh(\rho/2) - 1\) which becomes \(f_z=-1\) at
\(\rho=0\), \(f_z=0\) and hence cyclic at \(\rho=\tanh^{-1}(1/3)\), and
finally \(f_z=2\) at large \(\rho \).
The resulting spin magnitude and singlet-trio amplitude using the above
substitution in Eq.~\eqref{eq: C-FM-general-spinor} are plotted in
Fig.~\ref{fig: C-FM-analytical-spin-singlet}.
\begin{figure}
    \centering
    \includegraphics[width=0.5\textwidth]
    {gfx/ch-spin2/spin_singlet_radius_analytical.pdf}
    \caption[Analytically calculated spin magnitude and spin-singlet trio
        amplitude for a cyclic to ferromagnetic interface]
    {\label{fig: C-FM-analytical-spin-singlet}
        Analytical calculation of the spin magnitude (solid line) and
        spin-singlet trio amplitude (dashed line) using the initial state in
        Eq.~\eqref{eq: C-FM-interpolating-spinor}.
        We choose \(f_z(\rho) = 3\tanh(\rho/2) - 1\) to mode all three phases of
        this interpolating spinor.
        \textcolor{red}{Plot takes up a lot of space. Can I pair it with
            something?}
    }
\end{figure}
\textcolor{red}{Need to expand this discussion and really get my head around
    what this plot shows in relation to a vortex.}

In addition to phase vortices, we can construct a coreless vortex in the FM
phase that connects to a vortex state in the cyclic phase.
To construct the characteristic fountain-like spin texture, we choose a
monotonically increasing function of \(\beta=\beta(\rho)\).
The order-parameter is then kept single-valued by a combined winding of the
condensate phase coupled to a winding of the spin vector, achieved by
\(\tau'=2\alpha=2\varphi \), where \(\tau'=\tau-2\gamma \).
Applying such a rotation to the general spinor in
Eq.~\eqref{eq: C-FM-interpolating-spinor} yields
\begin{equation}\label{eq: C-FM-coreless-general}
    \zeta_\text{cl} = \frac{1}{\sqrt{3}}\left(D_2\zeta^\text{|F|=2}_\text{cl}
    + D_{-1}\zeta^\text{|F|=1}_\text{cl}\right),
\end{equation}
where
\begin{align}\label{eq: C-FM-coreless-FM-limits}
    \zeta^{|F|=2}_\text{cl} =
    \mqty(
    \cos^4\frac{\beta(\rho)}{2}                                        \\
    2e^{i\varphi}\cos^3\frac{\beta(\rho)}{2}\sin\frac{\beta(\rho)}{2}  \\
    \sqrt{6}e^{2i\varphi}\cos^2\frac{\beta(\rho)}{2}\sin^2
    \frac{\beta(\rho)}{2}                                              \\
    2e^{3i\varphi}\cos\frac{\beta(\rho)}{2}\sin^3\frac{\beta(\rho)}{2} \\
    e^{4i\varphi}\sin^4\frac{\beta(\rho)}{2}
    ),
\end{align}
\begin{align}
    \zeta^{|F|=1}_\text{cl} =
    \mqty(
    -2\cos\frac{\beta(\rho)}{2}\sin^3\frac{\beta(\rho)}{2}         \\
    e^{i\varphi}\sin^2\frac{\beta(\rho)}{2}\left(3\cos^2
    \frac{\beta(\rho)}{2}-\sin^2\frac{\beta(\rho)}{2}\right)       \\
    \sqrt{6}e^{2i\varphi}\left(\cos\frac{\beta(\rho)}{2}\sin^3
    \frac{\beta(\rho)}{2}
    - \cos^3\frac{\beta(\rho)}{2}\sin\frac{\beta(\rho)}{2} \right) \\
    e^{3i\varphi}\cos^2\frac{\beta(\rho)}{2}\left(\cos^2
    \frac{\beta(\rho)}{2}-3\sin^2\frac{\beta(\rho)}{2}\right)      \\
    2e^{4i\varphi}\cos^3\frac{\beta(\rho)}{2}\sin\frac{\beta(\rho)}{2}
    )
\end{align}
represent coreless vortices in both the FM-2 and FM-1 limits, respectively.
The spherical harmonic representation of these coreless vortices are shown
in Fig.~\ref{fig: C-FM-coreless-initial-states}a,
~\ref{fig: C-FM-coreless-initial-states}b.
\begin{figure}
    \centering
    \begin{tikzpicture}
        \node at (0, 0) {\includegraphics[width=0.45\textwidth]
            {gfx/ch-spin2/C-FM=2_coreless_FM_init_spherical.pdf}};
        \node at (7.5, 0) {\includegraphics[width=0.45\textwidth]
            {gfx/ch-spin2/FM-1_coreless_init_state.pdf}};
        \node at (3.75, -4) {\includegraphics[width=0.45\textwidth]
            {gfx/ch-spin2/C-FM=2_coreless_cyclic_init_spherical.pdf}};

        % Colour bar
        \node at (3.25, 2.3){\includegraphics{gfx/colourbars/compiled_hsv.pdf}};

        % Labels
        \node at (0, -2.2) {(a)};
        \node at (7.5, -2.2) {(b)};
        \node at (3.75, -6.2) {(c)};
    \end{tikzpicture}
    \caption[Spherical harmonic representation of a coreless vortex connection
        across a cyclic to ferromagnetic interface]
    {\label{fig: C-FM-coreless-initial-states}Spherical harmonic
        representation of the topological defects obtained in the cyclic, FM-1,
        and FM-2 limits in Eq.~\eqref{eq: C-FM-coreless-general}.
        (a) and (b): Coreless vortex in the FM-2 and FM-1 limit, respectively,
        given by Eq.~\eqref{eq: C-FM-coreless-FM-limits}.
        (c): Doubly quantised vortex in the cyclic limit, given by
        Eq.~\eqref{eq: C-FM-DQV-cyclic-limit}.}
\end{figure}
The resulting spinor in the cyclic limit is
\begin{equation}\label{eq: C-FM-DQV-cyclic-limit}
    \zeta_\text{dqv}
    = \left(\zeta^\text{|F|=2}_\text{cl}
    + \sqrt{2}\zeta^\text{|F|=1}_\text{cl}\right) / 3.
\end{equation}
The spherical harmonic representation of this cyclic state is plotted in
Fig.~\ref{fig: C-FM-coreless-initial-states}c.
By tracing a point about the vortex core, the condensate phase winds by a total
of \(4\pi \), indicating that this is a doubly-quantised vortex.

\subsection{Ferromagnetic to biaxial nematic}
Consider the case \(D_1 \neq 0\) and \(D_2 = 0\).
If \(D_1 \neq 0\), then Eq.~\eqref{eq: D1} implies \(\zeta_1=\zeta_{-1} = 0\).
Additionally, if \(\alpha \neq \tilde{\mu}\),
then Eq.~\eqref{eq: spin-2-stationary-zeta0-recast} implies \(\zeta_0 = 0\).
\textcolor{red}{Rest of derivation here.}

The resulting stationary solution has the form
\begin{equation}\label{eq: FM-BN-interpolating-spinor}
    \zeta^\text{FM-BN} = \frac{1}{\sqrt{2}}\mqty(
    e^{i\chi_2} D_+ \\
    0 \\
    0 \\
    0 \\
    e^{i\chi_{-2}}D_-),
    \quad D_\pm =\sqrt{1 \pm \frac{f_z}{2}},
\end{equation}
where \(f_z\) gives the longitudinal magnetisation.
The energy of this interpolating spinor reads
\begin{equation}\label{eq: FM-BN-energy}
    E^\text{FM-BN} = \frac{1}{2}\left(c_1-\frac{c_2}{20}\right)f_z^2
    - p f_z + 4q + \frac{c_2}{10},
\end{equation}
which is minimised precisely when \(f_z = p / [(c_1-c_2/20)n]\) provided that
\(c_1 \geq c_2/20\) for fixed \(p\) and \(q\).
Substitution of \(f_z = p / [(c_1-c_2/20)n]\) into Eq.~\eqref{eq: FM-BN-energy}
reveals that \(E^\text{FM-BN} \leq E^\text{BN},
E^{f_z=\pm 2}\) for \(|p| < (2c_1-c_2/10)n\).

One can see that at \(p = \pm 2(c_1-c_2/20)n\) the above solution becomes
the ferromagnetic state with spin \(f_z = \pm 2\).
Alternatively, the solution becomes the BN state when \(p=0\).
Therefore, this spinor provides an interpolating solution between the
ferromagnetic and BN phases, which is controlled by a longitudinal dependence of
the linear Zeeman shift, \(p(z)\).

We start from the interpolating spinor between the FM and BN phases in
Eq.~\eqref{eq: FM-BN-interpolating-spinor} and construct various defect states.
The simplest vortex case to consider is that of a singular SQVs on both sides of
the interface.
This winding is achieved by letting \(\tau = \varphi \), where \(\varphi \) is
the azimuthal angle about the core and letting the Euler angles be constant.
This resulting interpolating spinor then reads
\begin{equation}
    \zeta^\mathrm{FM-BN}_\mathrm{sqv} =
    \frac{1}{\sqrt{2}}\mqty(e^{i\varphi}D_+ \\ 0 \\ 0 \\ 0 \\
    e^{i\varphi}D_-),
\end{equation}
where \(D_\pm = \sqrt{1 \pm f_z / 2}\), and we have taken
\(\alpha=\beta=\gamma=0\) for simplicity.
Despite being characterised by the same phase winding on both sides of the
interface, each SQV represents an entirely different object due to the
differing topologies on either side.

Another solution involving fractional vortices can be obtained from the
above by removing the winding from one of the \(\zeta_{\pm 2}\) components.
For example, if \(f_z\) interpolates from \(f_z=0\) to \(f_z=2\) and the winding
was removed from the \(\zeta_{-2}\) component, this solution would then connect
an SQV on the FM side to a half-quantum vortex (HQV) on the BN side.
For that same state, if the winding was instead removed from the \(\zeta_{2}\)
component, then the HQV on the BN side would continuously connect to a
vortex-free state on the FM side.
Such a solution for the latter connection is achieved by
\(\alpha + \gamma = \varphi / 4\) and \(\tau=\varphi / 2\) which yields
\begin{equation}
    \zeta^\mathrm{FM-BN}_\mathrm{vf-sqv} =
    \frac{1}{\sqrt{2}}\mqty(D_+ \\ 0 \\ 0 \\ 0 \\
    e^{i\varphi}D_-).
\end{equation}

The interface between FM and BN phases allows for the connection of SQVs and
spin vortices.
Such an interpolating solution can be constructed through the
choice of \(\varphi=\gamma=0\) and \(\alpha=\varphi/2\) which results in a
singular SQV on the FM side which connects to a BN spin vortex.
The resulting spinor reads
\begin{equation}
    \zeta^\text{FM-BN}_\text{sqv-sv} =
    \mqty(e^{-i\varphi}D_+ \\ 0 \\ 0 \\ 0 \\
    e^{i\varphi}D_-),
\end{equation}
where we have chosen \(\beta = 0\).
This results in a singular SQV in both FM limits \(D_+ = 0 \vee D_- = 0\) and
a spin vortex in the BN limit \(D_+ = D_- = 1/\sqrt{2}\)

We can also construct defect crossing solutions consisting of a spin-2 FM
(Dirac) monopole.
Such a state is realised from the choice \(\tau'=2\alpha=2\varphi \) and
choosing \(\beta = \theta \) where \(\theta \) is the polar angle.
The resulting spinor reads
\begin{align}\label{eq: C-FM-Dirac}
    \zeta_\text{D}          & = \frac{1}{\sqrt{2}}
    \left( D_+\zeta^{F_z=2}_\text{D}+ D_-\zeta^{F_z=-2}_\text{D}\right), \\
    \zeta^{f_z=2}_\text{D}  & =
    \mqty(
    \exp(-4i\varphi) \cos^4 \frac{\theta}{2}                             \\
    2\exp(-3i\varphi) \cos^3 \frac{\theta}{2} \sin \frac{\theta}{2}      \\
    \sqrt{6}e^{2i\varphi}\cos^2\frac{\theta}{2}\sin^2\frac{\theta}{2}    \\
    2\exp(-i\varphi) \sin^3 \frac{\theta}{2} \cos \frac{\theta}{2}       \\
    \sin^4\frac{\theta}{2}
    \label{eq: C-BN-dirac2}
    ),                                                                   \\
    \zeta^{f_z=-2}_\text{D} & =
    \mqty(
    \exp(-4i\varphi) \sin^4 \frac{\theta}{2}                             \\
    2\exp(-3i\varphi) \sin^3 \frac{\theta}{2} \cos \frac{\theta}{2}      \\
    \sqrt{6}e^{2i\varphi}\cos^2\frac{\theta}{2}\sin^2\frac{\theta}{2}    \\
    2\exp(-i\varphi) \cos^3 \frac{\theta}{2} \sin \frac{\theta}{2}       \\
    \cos^4\frac{\theta}{2}
    \label{eq: C-BN-dirac-2}
    ),
\end{align}
where \(\zeta_D^{f_z = \pm 2}\) give the spin-2 generalisation of the
predicted spin-1 Dirac monopole~\cite{Savage2003}.
Such a monopole consists of a radial hedgehog texture of the condensate spin
that terminates at the end of a doubly quantised vortex line.
In the BN limit, one can show the resulting defect deforms into the BN hedgehog
that was introduced in the monopole connection of
Sec.~\ref{subsec: UN-BN-defects}.

As seen in Sec.~\ref{sec: vortices-spin-1}, the FM phase of a spin-1 BEC can
host a non-singular, coreless vortex that has a characteristic fountain-like
spin texture.
A coreless vortex can also be constructed in the spin-2 system following a
similar procedure and having \(\beta=\beta(\rho)\) be a monotonically increasing
function of the radial coordinate \(\rho = \sqrt{x^2 + y^2}\).
This, in addition to choosing the Euler angles
\(\tau - 2\gamma = 2\alpha = 2\varphi \) leads to the interpolating spinor
(\(\gamma=0\))
\begin{equation}\label{eq: FM-BN-coreless-DQV}
    \zeta^\text{FM-BN}_\text{cl} = \mqty(
    C^4D_+ + S^4D_- \\
    2e^{i\varphi}\left[C^3SD_+
        - CS^3D_-\right] \\
    e^{2i\varphi}\sqrt{6}C^2S^2\left[D_+
        + D_- \right] \\
    2e^{3i\varphi}\left[CS^3D_+
        - C^3SD_-\right] \\
    e^{4i\varphi}\left[S^4D_+ + C^4D_- \right]
    ),
\end{equation}
where \(C \equiv \cos(\beta(\rho)/2)\) and \(S \equiv \sin(\beta(\rho)/2)\).
In the FM limit (\(f_z=2\)), one recovers the spin-2 coreless vortex given as
\begin{equation}\label{eq: FM-BN-coreless}
    \zeta^\mathrm{coreless} = \sqrt{2}\mqty(C^4 \\ 2e^{i\varphi}C^3S \\
    e^{2i\varphi}\sqrt{6}C^2S^2 \\ 2e^{3i\varphi}CS^3 \\ e^{4i\varphi}S^4
    ).
\end{equation}
The spherical harmonic representation of this vortex is plotted in
Fig.~\ref{fig: C-FM-coreless-initial-states}, where the characteristic
fountain-like texture becomes apparent.
The spinor in the BN limit (\(f_z=0\)) reads
\begin{equation}\label{eq: FM-BN-DQV}
    \zeta_\mathrm{dqv} = \mqty(
    C^4 + S^4 \\
    2e^{i\varphi}CS\left[C^2-S^2\right] \\
    2e^{2i\varphi}\sqrt{6}C^2S^2 \\
    2e^{3i\varphi}CS\left[S^2 - C^2\right] \\
    e^{4i\varphi}\left[C^4 + S^4\right]).
\end{equation}
It is not immediately obvious from the form of the spinor the type of vortex
present on the BN side of the interface.
However, the spherical harmonics reveal that this is a doubly quantised vortex
in the BN limit (see Fig.~\ref{fig: BN-DQV}).
\begin{figure}
    \centering
    \begin{tikzpicture}
        \node at (0, 0) {\includegraphics[width=0.5\textwidth]
            {gfx/ch-spin2/BN_DQV.pdf}};
        \node[cylinder, draw, fill=black, opacity=0.5, minimum height=4cm,
            rotate=90] (c_dqv) at (0, 0) {};

        % Colour bar
        \node[rotate=90] at (-4.15, 0.4)
        {\includegraphics[width=0.15\textwidth, height=0.015\textheight]
            {gfx/colourbars/hsv_colourbar.pdf}};
        \node[rotate=90] at (-4.15, -1.6) {Arg(\(Z\)) = 0};
        \node[rotate=90] at (-4.15, 1.9) {\(2\pi \)};
    \end{tikzpicture}
    \caption[Spherical harmonic representation of a biaxial nematic doubly
        quantised vortex]
    {\label{fig: BN-DQV} Spherical harmonic
        representation of the BN vortex configuration given by
        Eq.~\eqref{eq: FM-BN-DQV}.
        A \(4\pi \) winding of the condensate phase about the core indicates
        this is a doubly quantised vortex, coupled with a non-trivial rotation
        of the condensate spin.}
\end{figure}
By tracing a point about the vortex line, the phase changes by a total of
\(4\pi \) indicating a doubly quantised vortex.

\section{Numerical investigations of defect crossing physics}
In this section we numerically investigate some topological interfaces defined
in the preceding section, along with a subset of the possible defect
connections.

Our numerical setup is as follows.
We numerically evolve the spin-2 GPEs defined in
Eqs.~\eqref{eq: spin-2-GPEs-pm2} -~\eqref{eq: spin-2-GPEs-0} using a symplectic
integrator~\cite{Symes2017} using a purely isotropic trapping potential
\(V=M\omega^2r^2/2\).
We simulate the energy loss during experiments by introducing a phenomenological
damping coefficient, \(\gamma \), through the substitution
\(t \rightarrow (1-i\gamma)t\).
In all simulations considered, we choose \(\gamma = 1e-2\).
We perform our simulations on a 3D of \(N_s^3=128^3\) points, with side lengths
\(L = 20\ell \), where \({(\ell =\hbar/M\omega)}^{1/2}\) is the (isotropic)
harmonic oscillator length.
We choose parameters that correspond to a \(^{87}\)Rb
condensate~\cite{Klausen2001} with \(c_0n=1.32\times10^4\hbar\omega\ell^3\),
\(c_0/c_1=90.7\), and \(c_0/c_2=-102\), where the ground state is predicted to
be nematic.
In each simulation, we perform a small spin rotation to the initial state to
avoid components that are identically zero.
Additionally, when constructing states with defects, the position of each defect
is perturbed slightly to avoid artificial stability when placed at exactly the
centre of the trap.


\subsection{Uniaxial nematic to biaxial nematic interface}
The first interface we consider is that between the UN and BN phases, considered
in Sec.~\ref{subsec: UN-BN-defects}.
Since the UN and BN phases are energetically degenerate in the absence of a
magnetic field, we introduce a spatially-dependent quadratic Zeeman shift
\(q(z)\) such that \(q(z) > 0\) on the UN side and \(q(z) < 0\) on the BN side,
to lift the degeneracy.
The quadratic Zeeman shift linearly interpolates over a small transition region,
which we take to be small compared to the spin-dependent healing lengths.

Our investigation begins with that of the SQV-SQV connection, where the initial
state is constructed as in Eq.~\eqref{eq: UN-BN-SQV-SQV} with \(\beta_0=0\).
To imprint the vortices, we perform a short imaginary time propagation, then
proceed to numerically evolve the spin-2 GPEs.

The dynamics of this connection is split into two distinct parts.
Firstly, upon evolution, the two overlapping SQVs spatially separate due to an
instability occurring at the interface \(z \approx 0\).
Each SQV then connects to a vortex-free state on the other side of the
interface shown in Fig.~\ref{fig: UN-BN-SQV-SQV-singlets}a.

After the initial separation, the cores of the vortices fill with atoms
occupying different ground states, drastically altering the order parameter
symmetry within the cores (see Fig.~\ref{fig: UN-BN-SQV-SQV-singlets}).
\begin{figure}
    \centering
    \begin{tikzpicture}
        \node at (0, 0) {\includegraphics[width=0.33\textwidth]
            {gfx/ch-spin2/UN-BN_SQV-SQV_longitudinal_a30.pdf}};

        \node at (4.8, 2.2) {\includegraphics[width=0.33\textwidth]
            {gfx/ch-spin2/UN-BN_SQV-SQV_a30_UN.pdf}};

        \node at (4.8, -2.7) {\includegraphics[width=0.33\textwidth]
            {gfx/ch-spin2/UN-BN_SQV-SQV_a30_BN.pdf}};

        \node at (10, 2.35) {\includegraphics[width=0.26\textwidth]
            {gfx/ch-spin2/UN-BN_SQV-SQV_singletTrio_UN.png}};
        \node[rectangle, draw=black, minimum width=0.55cm,
            minimum height=0.55cm]
        (UN_small_rec) at (5.05, 2.45) {};
        \node[rectangle, draw=black, minimum width=3.8cm,
            minimum height=3.8cm]
        (UN_big_rec) at (10, 2.33) {};
        \draw[-, dashed] (UN_big_rec.north west) -- (UN_small_rec.north west){};
        \draw[-, dashed] (UN_big_rec.south west) -- (UN_small_rec.south west){};

        \node at (10, -2.58) {\includegraphics[width=0.26\textwidth]
            {gfx/ch-spin2/UN-BN_SQV-SQV_singletTrio_BN.png}};
        \node[rectangle, draw=black, minimum width=0.55cm,
            minimum height=0.55cm]
        (BN_small_rec) at (5.1, -2.52) {};
        \node[rectangle, draw=black, minimum width=3.8cm,
            minimum height=3.8cm]
        (BN_big_rec) at (10, -2.57) {};
        \draw[-, dashed] (BN_big_rec.north west) -- (BN_small_rec.north west){};
        \draw[-, dashed] (BN_big_rec.south west) -- (BN_small_rec.south west){};

        \node at (0, -2.5) {(a)};
        \node at (5, -0.3) {(b)};
        \node at (5, -5.2) {(c)};
        \node at (10, -0.3) {(d)};
        \node at (10, -5.2) {(e)};
    \end{tikzpicture}
    \caption[Dynamics of a singly quantised vortex connection across a uniaxial
        nematic to biaxial nematic interface]
    {\label{fig: UN-BN-SQV-SQV-singlets}Spin singlet-trio amplitude for
        the UN-BN SQV-SQV connection given in Eq.~\eqref{eq: UN-BN-SQV-SQV} at
        \(\bar{t} = 300\).
        (a): Longitudinal cut showing the spatial separation of the two vortex
        lines.
        (b) and (c): Transverse cut on the UN (\(z/\ell = 3.125\)) and BN
        (\(z/\ell = -3.125\)) sides, respectively, showing the SQVs and their
        composite core structure.
        (d) and (e): Magnified transverse cuts with an overlay of the spherical
        harmonics, showing the non-trivial change of symmetry of the order
        parameter within the vortex cores.}
\end{figure}
In the UN case, the initially empty core fills with atoms occupying both the
cyclic and BN phases, generating a topological interface within the core itself.
This likely arises due to the differing phases factors between the wave function
components, as seen in Fig.~\ref{fig: UN-BN-duo-trio}.

The SQV on the BN undergoes a similar filling of the empty vortex core.
This time, the core fills with atoms in the UN, FM, and cyclic phases, also
generating an interface within the core.
The outer UN core is occupied by two small FM regions with
\(|\langle \hat{F} \rangle| = 2\), in addition to a larger, central cyclic
region \textcolor{red}{Plots highlighting the FM regions, hard to see from
    current plot}.
Spherical harmonics constructed about the core reveal that the phase winds by
\(\pi \) about these FM regions, accompanied by a \(\pi/2\) rotation of the
condensate spin vector.
The development of this composite core structure signifies the start of a
splitting process, whereby the SQV is expected to split into two HQVs.
Since our system has no rotation to stabilise the vortices, the timescales
considered here reveal that both vortices eventually leave the condensate
cloud.
Additionally, the SQV on the BN side of the interface will leave the condensate
before the splitting into two HQVs has occurred.
However, Fig.~\textcolor{red}{relevant fig} shows a purely imaginary time
simulation of the initial state in Eq.~\eqref{eq: UN-BN-SQV-SQV} with
\(\beta_0 = 0\) showing the resulting HQVs after the initial SQV has split.

We next investigate the spin vortex connection, using the initial state in
Eq.~\eqref{eq: UN-BN-SV-SV-spinor} with \(\beta_0=0\).
\textcolor{red}{Need to do simulation of this. May or may not include,
    depending on result.}

We can also investigate a vortex connection that terminates on the interface.
We start with the initial state in Eq.~\eqref{eq: UN-BN-SQV-SQV} with
\(\beta_0=0\), and artificially remove the winding from the middle component.
The result is a SQV in the BN phase that smoothly connects to a vortex-free
state in the UN phase.
The resulting spin magnitude and singlet-trio amplitude after purely
imaginary-time relaxation are plotted in Fig.~\ref{fig: UN-BN-VF-SQV}.
\begin{figure}
    \centering
    \begin{subfigure}{0.45\textwidth}
        \includegraphics[width=\textwidth]
        {gfx/ch-spin2/UN-BN_VF-SQV_spin_mag.pdf}
        \caption{}
    \end{subfigure}
    \begin{subfigure}{0.45\textwidth}
        \includegraphics[width=\textwidth]{gfx/ch-spin2/UN-BN_VF-SQV_a30.pdf}
        \caption{}
    \end{subfigure}
    \caption[Dynamics of a singly quantised vortex to vortex-free connection in
    a uniaxial nematic to biaxial nematic interface]
    {\label{fig: UN-BN-VF-SQV} Vortex-free to SQV connection defined
    by Eq.~\eqref{eq: UN-BN-SQV-SQV} with \(\beta=0\) and no winding in the
    middle component.
    (a): Spin magnitude at \(\bar{t}=5\). The HQV cores are identified where
    \(\spinmag = 2\).
    (b): Spin-singlet trio amplitude at \(\bar{t} = 5\). A cyclic region is
    revealed where \(|A_{30}|^2 = 2\), which arises due to phase differences
    (see Fig~\ref{fig: UN-BN-duo-trio}).}
\end{figure}
The dynamics of this connection closely resembles what the later dynamics of the
BN side of the SQV-SQV connection would look like, provided that the vortices
were stabilised against leaving the condensate.
We see that the initial SQV on the BN side has undergone a splitting process
into two HQVs, each of which terminate at the interface.
The cores of the HQVs are easily identified from the
\(\spinmag = 2\) regions.
Similar splitting of an SQV into HQVs has been observed in the polar phase of
spin-1 condensates~\cite{Seo2015, Xiao2021}.
For the timescales considered, the resulting HQVs remain stable against
leaving the condensate.

\textcolor{red}{Monopole?}

\subsection{Cyclic to ferromagnetic interface}
We numerically investigate vortex connections across a topological interface
between the cyclic and FM-2 phases, given by
Eq.~\eqref{eq: C-FM-interpolating-spinor}.
Since our numerical simulations use parameters that predict a nematic ground
state, we introduce a spatially-dependent \(c_1\) term such that \(c0/c_1=90.7\)
on the cyclic side but \(c_0/c_1=-90.7\) on the FM side, effectively changing
the sign of the \(c_1\) term.
Now, in the FM region, the parameters ensure that the FM region remains stable.
Despite the cyclic state not being the predicted ground state, the interface
is observed to remain stable for the timescales considered in our simulations.

We firstly investigate the connection of a third vortex in the cyclic phase
connecting to a singular SQV in the FM-2 phase using
Eq.~\eqref{eq: C-FM-general-spinor} with \(\alpha + \gamma = \varphi/3\) and
\(\tau=\varphi/3\) as the initial state.
The initial state is then propagated using a short imaginary-time evolution to
imprint the vortex cores.
From here we switch to complex-time using a damping coefficient of
\(\gamma=10^{-2}\).

The resulting spin magnitude and spin-singlet duo amplitude are plotted in
Fig.~\ref{fig: C-FM-third-SQV}.
\begin{figure}[htb!]
    \centering
    \begin{tikzpicture}

        \node[inner sep=0pt] (plot1) at (-5,0)
        {\includegraphics[width=0.33\textwidth]
            {gfx/ch-spin2/C-FM=2_third-SQV_spin_mag.pdf}};
        \node[inner sep=0pt] (plot2) at (-0.6, 0.25)
        {\includegraphics[width=0.25\textwidth, height=0.25\textwidth]
            {gfx/ch-spin2/C-FM=2_third-SQV_spinMag_spherical.pdf}};
        \node[rectangle, draw=black, minimum width=0.5cm, minimum height=0.5cm]
        (rec) at (-4.75, 0.75) {};
        \draw[-, dashed] (rec.north west) -- (plot2.north west) {};
        \draw[-, dashed] (rec.south west) -- (plot2.south west) {};

        \draw[-, thick] (plot2.south east) -- (plot2.north east) {};
        \draw[-, thick] (plot2.south east) -- (plot2.south west) {};
        \draw[-, thick] (plot2.south west) -- (plot2.north west) {};
        \draw[-, thick] (plot2.north west) -- (plot2.north east) {};

        \node[inner sep=0pt] (plot0) at (3.8,0)
        {\includegraphics[width=0.33\textwidth]
            {gfx/ch-spin2/C-FM=2_third-SQV_singlet_trio.pdf}};

        % Labels
        \node at (-5.15, -2.5) {(a)};
        \node at (-0.5, -2.5) {(b)};
        \node at (3.95, -2.5) {(c)};
    \end{tikzpicture}
    \caption[Dynamics of a one-third vortex to singly quantised vortex
    connection in a cyclic to ferromagnetic interface]
    {\label{fig: C-FM-third-SQV}Third vortex to SQV connection in an
    interface between the cyclic and FM-2 phases given by
    Eq.~\eqref{eq: C-FM-general-spinor} with \(\alpha + \gamma = -\varphi/3\)
    and \(\tau = \varphi/3\) at \(\bar{t} = 50\).
    (a): Longitudinal cut of \(\spinmag \) at \(y/\ell=0\).
    The third vortex on the cyclic side (\(z/\ell < 0\)) is evident from the
    \(\spinmag = 1\) core which extends throughout the FM region.
    (b): Zoomed transverse cut of \(\spinmag \) inside the core in the FM
    region.
    Overlaid are the spherical harmonics showing the non-trivial change
    of order parameter symmetry inside the core.
    (c): Longitudinal cut of \(|A_{30}|^2\) at \(y/\ell=0\).
    Cyclic regions are identified from \(|A_{30}|^2=2\).}
\end{figure}
Here, \(\spinmag \) reveals non-trivial core structures emerging.
Clearly one can see the third vortex on the cyclic side of the interface
(\(z/\ell < 0\)) evidenced by the \(\spinmag = 1\) core.
However, this \(\spinmag = 1\) region then extends throughout the longitudinal
extent of the condensate, and penetrates into the FM region.
The initial SQV of the FM side has developed a composite core structure.
Inside is the \(\spinmag = 1\) region, which is encased in a cyclic shell as
seen from the \(|A_{30}|^2=2\) regions.
The spherical harmonics reveal the non-trivial change of order parameter
symmetry within the composite core.
Far away from the vortex core, the system then interpolates back to the FM-2
phase.
Note that the composite core structure was predicted analytically in
Fig.~\ref{fig: C-FM-analytical-spin-singlet}.

Instead of considering only singular vortices, we can also investigate the
coreless vortex connection given in Eq.~\eqref{eq: C-FM-coreless-general}.
We choose this as the initial state, with
\(\beta = \pi\left(1 + \tanh(\rho-1)\right)/2\) to model the required
monotonically increasing function.
Here we focus only on the cyclic to FM-2 limit, but equivalently the cyclic to
FM-1 limit can be chosen by an appropriate choice of \(p\) and \(q\) that
interpolates \(f_z\) between \(-1\) and \(0\).
We perform purely imaginary-time simulations to simulate energy relaxation.
\begin{figure}
    \centering
    \begin{tikzpicture}
        % DQV
        \node[inner sep=0pt] (initial_dqv) at (0.4, 0)
        {\includegraphics[width=0.3\textwidth]
            {gfx/ch-spin2/C-FM=2_coreless_cyclic_init_spherical.pdf}};
        \node[cylinder, draw, fill=black, opacity=0.5, minimum height=3cm,
            rotate=90] (c_dqv) at (0.4, 0) {};

        % Colour bar
        \node[rotate=90] at (-2.15, 0.2)
        {\includegraphics[width=0.15\textwidth, height=0.015\textheight]
            {gfx/colourbars/hsv_colourbar.pdf}};
        \node[rotate=90] at (-2.15, -1.8) {Arg(\(Z\)) = 0};
        \node[rotate=90] at (-2.15, 1.7) {\(2\pi \)};

        % Spin mag
        \node[inner sep=0pt] (spinmag_dqv) at (5, 0)
        {\includegraphics[width=0.3\textwidth]
            {gfx/ch-spin2/C-FM=2_coreless_cyclic_spin_mag.pdf}};

        % Third spherical
        \node[inner sep=0pt] (third_spherical) at (10, 1.5)
        {\includegraphics[width=0.3\textwidth]
            {gfx/ch-spin2/C-FM=2_coreless_cyclic_third_spherical.pdf}};
        \node[cylinder, draw, fill=green, opacity=0.5, minimum height=2.8cm,
            rotate=90] (c_third) at (10, 1.5) {};
        \node[rectangle, draw, minimum width=0.5cm, minimum height=0.5cm,
            dashed, green]
        (third_small_rec) at (5.65, 0.78) {};
        \node[rectangle, draw, minimum width=4.4cm, minimum height=2.9cm,
            green] (third_big_rec) at (10, 1.6) {};
        \draw[-, dashed, green]
        (third_big_rec.north west) -- (third_small_rec.north west) {};
        \draw[-, dashed, green]
        (third_big_rec.south west) -- (third_small_rec.south east) {};

        % Two-third spherical
        \node[inner sep=0pt] (third_spherical) at (10, -1.5)
        {\includegraphics[width=0.3\textwidth]
            {gfx/ch-spin2/C-FM=2_coreless_cyclic_2third_spherical.pdf}};
        \node[cylinder, draw, fill=red, opacity=0.5, minimum height=2.8cm,
            rotate=90] (c_2third) at (10, -1.5) {};
        \node[rectangle, draw, dashed, minimum width=0.9cm,
            minimum height=0.9cm, rotate=45, opacity=0.5, red]
        (2third_small_rec) at (5.05, 0.15) {};
        \node[rectangle, draw, minimum width=4.4cm, minimum height=2.9cm,
            red] (2third_big_rec) at (10, -1.4) {};
        \draw[-, dashed, red]
        (2third_big_rec.north west) -- (2third_small_rec.south east) {};
        \draw[-, dashed, red]
        (2third_big_rec.south west) -- (2third_small_rec.south west) {};

        % Labels
        \node at (0.4, -2.3) {(a)};
        \node at (5, -2.3) {(b)};
        \node at (8.2, 2.7) {(c)};
        \node at (8.2, -0.3) {(d)};
    \end{tikzpicture}
    \caption[Dynamics of the doubly quantised vortex connection in a cyclic to a
        ferromagnetic interface]
    {\label{fig: C-FM-coreless-cyclic}Schematic representation of the
        splitting process occurring on the cyclic side of the interface
        (\(z/\ell < 0\)) given by the state in
        Eq.~\eqref{eq: C-FM-coreless-general}.
        (a): Spherical harmonic representation of the initial doubly quantised
        vortex line. By traversing a point about the vortex line, the condensate
        phase is seen to wind by \(4\pi \).
        (b) Transverse cut of \(\spinmag \) at \(z/\ell \approx -3\) after
        imaginary-time evolution at \(\bar{t} = 1.5\) showing the splitting of
        the initial doubly quantised vortex into fractional vortices.
        The one-third and two-third vortices are clearly identified from the
        \(\spinmag = 1\) and \(\spinmag = 2\) regions, respectively.
        (c) and (d): Spherical harmonic representations about the one-third and
        two-third vortices, respectively.
        The spherical harmonics shows the non-trivial change of the order
        parameter symmetry as we move away from the vortex cores.}
\end{figure}
We start by discussing the dynamics of the doubly quantised vortex on the cyclic
side of the interface, shown in Fig.~\ref{fig: C-FM-coreless-cyclic}.
As expected for a doubly quantised vortex line, it very rapidly undergoes a
splitting process.
In this case, it splits into four one-third vortices, evidenced by the
\(\spinmag = 1\) regions.
In addition, the large \(\spinmag = 2\) region indicates the core of a two-third
vortex.
\textcolor{red}{Why different sizes? Set by different healing lengths?}
Analysis of the spherical harmonics in Fig.~\ref{fig: C-FM-coreless-cyclic}c,d
shows the non-trivial order parameter symmetry both within and outside the
vortex cores.
By following the spherical harmonics about the vortex cores, the condensate
phase changes by \(2\pi/3\) and \(4\pi/3\) confirming that these structures are
one-third and two-third vortices.
Due to the energy relaxation, the one-third vortices quickly leave the
condensate.
However, the two-third vortex first undergoes a further splitting process into
two one-third vortices, which then proceed to exit the condensate cloud.

The initial coreless vortex on the FM side of the interface also undergoes a
complex splitting process, shown in Fig.~\ref{fig: C-FM-coreless-FM}.
\begin{figure}
    \centering
    \begin{tikzpicture}
        % DQV
        \node[inner sep=0pt] (initial_coreless) at (0.4, 0)
        {\includegraphics[width=0.3\textwidth]
            {gfx/ch-spin2/C-FM=2_coreless_FM_init_spherical.pdf}};

        % Colour bar
        \node[rotate=90] at (-2.15, 0.2)
        {\includegraphics[width=0.15\textwidth, height=0.015\textheight]
            {gfx/colourbars/hsv_colourbar.pdf}};
        \node[rotate=90] at (-2.15, -1.8) {Arg(\(Z\)) = 0};
        \node[rotate=90] at (-2.15, 1.7) {\(2\pi \)};

        % Spin mag
        \node[inner sep=0pt] (spinmag_coreless) at (5, 0)
        {\includegraphics[width=0.3\textwidth]
            {gfx/ch-spin2/C-FM=2_coreless_FM_spin_mag.pdf}};

        % After splitting
        \node[inner sep=0pt] (initial_dqv) at (10, 0)
        {\includegraphics[width=0.3\textwidth]
            {gfx/ch-spin2/C-FM=2_coreless_FM_after_spherical.pdf}};

        \node[rectangle, draw, minimum width=0.5cm, minimum height=0.5cm]
        (small_rec) at (5.35, -0.13) {};
        \node[rectangle, draw, minimum width=4.4cm, minimum height=2.3cm]
        (big_rec) at (10, 0) {};
        \draw[-, dashed] (big_rec.north west) -- (small_rec.north east) {};
        \draw[-, dashed] (big_rec.south west) -- (small_rec.south east) {};
    \end{tikzpicture}
    \caption[Dynamics of the coreless vortex in a cyclic to ferromagnetic
        interface]
    {\label{fig: C-FM-coreless-FM}Schematic representation of the
        splitting process occurring on the FM side of the interface
        (\(z/\ell > 0\)) of the state in Eq.~\eqref{eq: C-FM-coreless-general}.
        (a): Spherical harmonic representation of the initial coreless vortex,
        defined explicitly in Eq.~\eqref{eq: C-FM-coreless-FM-limits}.
        (b) Transverse cut of \(\spinmag \) at \(z/\ell \approx 3\) after
        complex-time evolution at \(\bar{t} = 1.5\).
        The singular vortex structures can be seen from the
        \(\spinmag \approx 1\) cores.
        (c): Spherical harmonic representation of the internal structure of
        the singular vortex, showing the non-trivial symmetry within the core.}
\end{figure}
The coreless structure is observed to split into four singular vortices,
observed from transverse cuts of \(\spinmag \).
Analysis of the spherical harmonics reveal the non-trivial symmetry within
the singular vortex cores.
As in the case on the cyclic side, these vortices rapidly leave the condensate
due to the energy relaxation.

On both sides of the interface the vortex structures are observed to terminate
at the interface itself, and do not connect in a way that is observed in the
SQV to one-third vortex connection (see Fig.~\ref{fig: C-FM-third-SQV}).
