\chapter{Vortex connections across topological interfaces}\label{chap: spin-2}
In this chapter, we both analytically and numerically investigate the physics
of topological defects when connected across topological interfaces in spin-2
Bose-Einstein condensates.
We demonstrate that a particularly rich phenomenology of topological defects
at the coherent
interface between regions of different broken symmetries can be realised in
spin-2 Bose-Einstein condensates.
In particular, we propose that an interface between uniaxial and biaxial nematic
phases exhibiting continuous and discrete order-parameter symmetry,
respectively, can be realised using existing experimental techniques.
We construct spinor wave functions that represent topological defects connecting
across the interface.
By numerical energy relaxation as well as simulations of dynamics, we
characterise the emergence of non-trivial defect core structures.
We further demonstrate the emergence of composite vortex-core structures and
continuous connection of fractional vortices representing non-Abelian charges in
interfaces involving the cyclic and ferromagnetic phases, which could be
realised through manipulation of the inter-atomic interactions.
Our results suggest the spin-2 Bose-Einstein condensates as experimentally
accessible test beds for interface physics with all combinations of continuous
and discrete symmetries, as well as phases supporting non-Abelian defects.

\section{Introduction to topological interfaces}
When topologically distinct phases, described by different order parameters,
coexist in a continuous and coherent ordered medium, a topological interface
must form at the phase boundary, where the different broken symmetries connect
smoothly.
Such interfaces are ubiquitous across many areas of physics, from the
\(A\)--\(B\) phase boundary in superfluid
helium~\cite{Salomaa1987, Volovik2009, Finne2006}, to appearing as the
termination points of cosmic strings in theories of the early
universe~\cite{Kibble1976, Vilenkin1994}, and even in brane-inflation
models in superstring theory~\cite{Dvali1999, Sarangi2002}.
They may also play an important role for superconducting materials in
solid-state physics~\cite{Bert2011}.
The different order-parameter symmetries imply that the bulk medium on either
side of the interface supports different families of topological defects and
textures, which therefore cannot cross the interface unchanged, but must either
terminate, or connect continuously and non-trivially to an object representing
the different topology on the other side.
Due to the ubiquitous nature of topological interfaces, their study in
controlled experiments becomes of general importance, inspiring the use of
laboratory systems as emulators for interface physics in contexts otherwise not
amenable to experimental observations, for example the simulation of
brane-collision processes using superfluid $^3$He~\cite{Bradley2007}.

Topological-interface physics becomes especially intriguing when the medium on
either or both sides of the interface exhibits point-group order-parameter
symmetry~\cite{Xiao2022}, leading to the existence of non-Abelian defects,
whose charges depend on the presence of other defects in the system and whose
dynamics is highly constrained~\cite{Poenaru1977,Mermin1979,Kobayashi2009}.
Such fully discrete order-parameters symmetries readily arise in particular
phases of spin-2 and -3 Bose-Einstein condensates
(BECs)~\cite{Barnett2006,Barnett2007,Semenoff2007,Makela2007,Yip2007,
    Kobayashi2009,Borgh2016b,Xiao2022}, which have consequently been proposed, e.g.,
as candidates for quantum-computation applications~\cite{Mawson2019} and for
realisation of non-Abelian quantum turbulence~\cite{Mawson2015}.
Spinor BECs~\cite{Kawaguchi2012,StamperKurn2013} are constructed in all-optical
traps such that the internal spin-degrees of freedom are not frozen out by
strong magnetic fields~\cite{StamperKurn1998} and provide an ideal
testing ground for investigating interface
physics~\cite{Borgh2012,Borgh2013,Borgh2014}.
Their rich phase diagrams~\cite{Kawaguchi2012} exhibit a rich variety of
phases with different order-parameter symmetries, supporting radically different
families of topological defects, such as singular
vortices~\cite{Yip1999,Isoshima2002,
    Zhou2003,Ji2008,Takahashi2009,Lovegrove2012,Weiss2019,Xiao2021}, including
vortices carrying fractional circulation~\cite{Leonhardt2000,Ji2008,Seo2015,
    Semenoff2007,Borgh2016b,Xiao2022}, and non-singular
textures~\cite{MizushimaPRL2002, MizushimaPRA2002,
    Martikainen2002,Choi2012,Lovegrove2014},
monopoles~\cite{Pietila2009,Ollikainen2017,Ruostekoski2003,Ray2014,Ray2015,
    Mithun2022}, and even Skyrmions~\cite{Tiurev2018,Lee2018} and
knots~\cite{Hall2016}.

With spin-2 BECs being experimentally readily realisable~\cite{Schmaljohann2004}
and  experimental techniques for controlled creation of vortices with internal
point-group symmetries having recently been developed~\cite{Xiao2022}, these
systems are poised as immediate candidates for the realisation of topological
interfaces with non-Abelian defect physics. Here we analytically construct spinor
wave functions representing continuous defect connections across all
permutations of topological interfaces between the different ground-state
phases of spin-2 BECs and demonstrate their energy relaxation by numerical
simulation for illustrative examples. We have previously suggested that
topological interface could be realised in spin-1 BECs through the spatial
engineering of induced Zeeman shifts~\cite{Borgh2014} (which can also control
the defect-core symmetry properties~\cite{Borgh2016a,Underwood2020}) or, less
straightforwardly, using optical or microwave Feshbach resonances to manipulate
atomic scattering lengths~\cite{Borgh2012,Borgh2013}. Here we show how the
former technique readily lends itself to the formation of a topological
interface between uniaxial (UN) and biaxial nematic (BN) phases with commonly
used atomic species such as $^{87}$Rb. We construct solutions corresponding to
connecting singly quantized and spin vortices, as well as terminating vortices.
\textcolor{red}{[Something more here]}

\section{Interface crossing solutions in a spin-2 BEC}
\subsection{Dynamical stability}

\section{Numerical investigations of defect crossing physics}
\subsection{Uniaxial nematic to biaxial nematic interface}
\subsection{Cyclic to ferromagnetic interface}
\subsection{Cyclic to biaxial nematic interface}
