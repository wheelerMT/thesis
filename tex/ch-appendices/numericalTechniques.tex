\chapter{Numerical techniques}\label{appendix: numerical-techniques}

\section{\label{sec: two-comp-dimensionless}Dimensionless two-component
Gross-Pitaevskii equations}

As mentioned in Sec~\ref{sec: dimensionless-equations}, dimensionless equations
offer numerous benefits, including increased numerical stability and the ease
of reformulating calculations into a desired scale and parameter regime.
Here we derive the dimensionless two-component GPEs which are stated in
Eq.~\eqref{eq: dimensionless-two-comp-GPEs}.

To begin we start with the full 3D dimensional GPEs for a two-component system:
\begin{align} \label{eq: two-comp-GPE-psi1}
    i\hbar\pdv{\psi_1}{t} &= \left(-\frac{\hbar^2\nabla^2}{2m_1} + g_1|\psi_1|^2
    +g_{12}|\psi_2|^2\right)\psi_1, \\
    i\hbar\pdv{\psi_2}{t} &= \left(-\frac{\hbar^2\nabla^2}{2m_2} + g_2|\psi_2|^2
    +g_{12}|\psi_1|^2\right)\psi_2. \label{eq: two-comp-GPE-psi2}
\end{align}
Our simulations are performed on space-time grid lattices with a spatial side
length of \(L = N_s a_s\), where \(a_s\) is the lattice spacing and the total
number of grid points is given as \(N_s^d\) where \(d\) is the dimensionality of
the system.
With these, we make use of the following dimensionless variables
\begin{align}
    \tilde{\vb{r}} = \frac{\vb{r}}{a_s}, \quad
    \tilde{g}_j = \frac{2mg_j{a_s}^{2-d}}{\hbar^2}, \quad
    \tilde{t} = \frac{t\hbar}{2ma_s^2}, \quad
    \tilde{\psi}_j = \sqrt{a_s^d}e^{2i\tilde{t}}\psi_j,
\end{align}
where \(m = m_1 = m_2\).
Using the above dimensionless quantities, we can rescale the dimensional
variables in Eqs~\eqref{eq: two-comp-GPE-psi1} -~\eqref{eq: two-comp-GPE-psi2},
which leads to the dimensionless equations
\begin{align}
    i\pdv{\tilde{\psi}_1}{t} &= \left(-\tilde{\nabla}^2
    + \tilde{g}_1|\tilde{\psi}_1|^2 + \tilde{g}_{12}|\tilde{\psi}_2|^2\right)
    \tilde{\psi}_1, \\
    i\pdv{\tilde{\psi}_2}{t} &= \left(-\tilde{\nabla}^2
    + \tilde{g}_2|\tilde{\psi}_2|^2 + \tilde{g}_{12}|\tilde{\psi}_1|^2\right)
    \tilde{\psi}_2.
\end{align}
Now, if we have atomic species where \(g_1=g_2=g\), then the above equations
simplify to (dropping the tildes for notational convenience)
\begin{align}
    i\pdv{{\psi}_1}{t} &= \left(-{\nabla}^2
    + {g}|{\psi}_1|^2 + {\gamma}|{\psi}_2|^2\right)
    {\psi}_1, \\
    i\pdv{{\psi}_2}{t} &= \left(-{\nabla}^2
    + {g}|{\psi}_2|^2 + \gamma|{\psi}_1|^2\right)
    {\psi}_2,
\end{align}
where \(\gamma = g_{12}/g\) is the ratio of inter- to intra-species interaction.

\section{\label{sec: symplectic-integrators}Symplectic integrators for spinor
Bose-Einstein condensates}
The numerics of Chapters~\ref{chap: spin-1} and~\ref{chap: spin-2} are based on
second-order symplectic integrators: a type of numerical scheme for Hamiltonian
systems.
In particular, we make use of the work of Symes \textit{et al.} for both our
spin-1~\cite{Symes2016} and spin-2~\cite{Symes2017a} systems.
Symplectic integration schemes provide highly accurate solutions for Hamiltonian
systems over extended periods of time, making them ideal for studying the
long-time dynamics in Chapter~\ref{chap: spin-1}.
In addition, the second-order numerical schemes devised by Symes \textit{et al.}
have been shown to be both simpler and more efficient than other second-order
composition methods~\cite{Wang2007}, which is critical for compute-heavy 3D
simulations as in Chapter~\ref{chap: spin-2}.

The spinor symplectic integration schemes work by splitting the spinor GPEs into
two subsystems.
The first is a single-particle-like subsystem, which includes the kinetic energy
and the Zeeman terms of the Hamiltonian, which is solved exactly in Fourier
Space.
The second system comprises the non-linear subsystem, which includes the
remaining non-linear terms in the spinor GPEs.
Due to the nature of the splitting, this remaining subsystem becomes exactly
solvable in position space.
Note that the Zeeman terms are assumed to be spatially-uniform (a good
approximation in experiments), which allows them to be included in the
single-particle-like subsystem.
However, if the Zeeman shifts are not spatially-uniform they must instead be
included with the non-linear subsystem, which still leads to analytical
solutions in the form of Jacobi elliptic functions (see
Refs.~\cite{Symes2016, Symes2017a} for details).
Finally, though not used in this thesis, the two-way splitting implies that
the scheme can be extended using the composition method of Blanes and
Moan~\cite{Blanes2002} to fourth-order, should a higher-order symplectic scheme
be required.

The numerical methods throughout this thesis have been implemented using the
personally-developed Python package
\href{https://github.com/wheelerMT/pygpe}{
    \textit{PyGPE}
}~\cite{Wheeler_PyGPE_2024}: a CUDA-accelerated library for numerically solving
the GPEs for scalar, two-component, spin-1, and spin-2 BEC systems.
It supports 1D, 2D, and 3D grid lattices, as well as possessing the ability to
run on both CPUs or Nvidia-native GPUs.
In addition, it provides tools for data handling, as well as a suite of
diagnostics functions useful for handling data from BEC simulations.
