\chapter{Numerical techniques}\label{appendix: numerical-techniques}

\section{\label{sec: two-comp-dimensionless}Dimensionless two-component
Gross-Pitaevskii equations}

As mentioned in Sec~\ref{sec: dimensionless-equations}, dimensionless equations
offer numerous benefits, including increased numerical stability and the ease
of reformulating calculations into a desired scale and parameter regime.
Here we derive the dimensionless two-component GPEs which are stated in
Eq.~\eqref{eq: dimensionless-two-comp-GPEs}.

To begin we start with the full 3D dimensional GPEs for a two-component system:
\begin{align} \label{eq: two-comp-GPE-psi1}
    i\hbar\pdv{\psi_1}{t} &= \left(-\frac{\hbar^2\nabla^2}{2m_1} + g_1|\psi_1|^2
    +g_{12}|\psi_2|^2\right)\psi_1, \\
    i\hbar\pdv{\psi_2}{t} &= \left(-\frac{\hbar^2\nabla^2}{2m_2} + g_2|\psi_2|^2
    +g_{12}|\psi_1|^2\right)\psi_2. \label{eq: two-comp-GPE-psi2}
\end{align}
Our simulations are performed on space-time grid lattices with a spatial side
length of \(L = N_s a_s\), where \(a_s\) is the lattice spacing and the total
number of grid points is given as \(N_s^d\) where \(d\) is the dimensionality of
the system.
With these, we make use of the following dimensionless variables
\begin{align}
    \tilde{\vb{r}} = \frac{\vb{r}}{a_s}, \quad
    \tilde{g}_j = \frac{2mg_j{a_s}^{2-d}}{\hbar^2}, \quad
    \tilde{t} = \frac{t\hbar}{2ma_s^2}, \quad
    \tilde{\psi}_j = \sqrt{a_s^d}e^{2i\tilde{t}}\psi_j,
\end{align}
where \(m = m_1 = m_2\).
Using the above dimensionless quantities, we can rescale the dimensional
variables in Eqs~\eqref{eq: two-comp-GPE-psi1} -~\eqref{eq: two-comp-GPE-psi2},
which leads to the dimensionless equations
\begin{align}
    i\pdv{\tilde{\psi}_1}{t} &= \left(-\tilde{\nabla}^2
    + \tilde{g}_1|\tilde{\psi}_1|^2 + \tilde{g}_{12}|\tilde{\psi}_2|^2\right)
    \tilde{\psi}_1, \\
    i\pdv{\tilde{\psi}_2}{t} &= \left(-\tilde{\nabla}^2
    + \tilde{g}_2|\tilde{\psi}_2|^2 + \tilde{g}_{12}|\tilde{\psi}_1|^2\right)
    \tilde{\psi}_2.
\end{align}
Now, if we have atomic species where \(g_1=g_2=g\), then the above equations
simplify to (dropping the tildes for notational convenience)
\begin{align}
    i\pdv{{\psi}_1}{t} &= \left(-{\nabla}^2
    + {g}|{\psi}_1|^2 + {\gamma}|{\psi}_2|^2\right)
    {\psi}_1, \\
    i\pdv{{\psi}_2}{t} &= \left(-{\nabla}^2
    + {g}|{\psi}_2|^2 + \gamma|{\psi}_1|^2\right)
    {\psi}_2,
\end{align}
where \(\gamma = g_{12}/g\) is the ratio of inter- to intra-species interaction.