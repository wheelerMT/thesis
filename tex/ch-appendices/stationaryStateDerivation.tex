\chapter{Derivation of stationary solutions in a spin-2 Bose-Einstein
  condensate}\label{appendix: stationary}
\section{Time-independent spin-2 Gross-Pitaevskii equations}
Here we derive stationary solutions to the spin-2 GPEs that provide the
interpolating spinor wave functions that are used throughout
Chapter~\ref{chap: spin-2}.
Firstly, recall that the stationary solutions of the spin-2 GPEs are obtained by
substituting \(\psi_m = \sqrt{n}\zeta_m e^{-i\mu t/\hbar}\) into
Eqs.~\eqref{eq: spin-2-GPEs-pm2}-\eqref{eq: spin-2-GPEs-0}.
Assuming a uniform system with \(V(\vb{r}) = 0\), this results in
\begin{align}
    \mu\zeta_2 & = \left(-2p + 4q + c_0n
    + 2c_1n\langle\hat{F}_z\rangle\right)\zeta_2
    + c_1n\langle\hat{F}_-\rangle\zeta_1
    + \frac{c_2}{\sqrt{5}}nA_{00}\zeta^*_{-2}, \\
    \mu\zeta_1 & = \left(-p + q + c_0n
    + c_1n\langle\hat{F}_z\rangle\right)\zeta_1
    + c_1\left(\frac{\sqrt{6}}{2}n\langle\hat{F}_-\rangle\zeta_0
    +n\langle\hat{F}_+\rangle\zeta_2\right)
    - \frac{c_2}{\sqrt{5}}nA_{00}\zeta^*_{-1}, \\
    \mu\zeta_0 & = c_0n\zeta_0 + \frac{\sqrt{6}}{2}c_1\left(
        n\langle\hat{F}_+\rangle\zeta_1 + n\langle\hat{F}_-\rangle\zeta_{-1}
    \right) + \frac{c_2}{\sqrt{5}}nA_{00}\zeta_0^*,        \\
    \mu\zeta_{-1} & = \left(p + q + c_0n
    - c_1n\langle\hat{F}_z\rangle\right)\zeta_{-1}
    + c_1\left(\frac{\sqrt{6}}{2}n\langle\hat{F}_+\rangle\zeta_0
    +n\langle\hat{F}_-\rangle\zeta_{-2}\right)
    - \frac{c_2}{\sqrt{5}}nA_{00}\zeta^*_{1}, \\
    \mu\zeta_{-2} & = \left(2p + 4q + c_0n
    - 2c_1n\langle\hat{F}_z\rangle\right)\zeta_{-2}
    + c_1n\langle\hat{F}_+\rangle\zeta_{-1}
    + \frac{c_2}{\sqrt{5}}nA_{00}\zeta^*_{2}.
\end{align}
Follow the literature of Ref.~\cite{Kawaguchi2012}, we make the following
assumptions.
We can choose the overall phase such that \(\zeta_0\) is real and, since the
system has \(\text{SO}(2)\) symmetry about the direction of the applied magnetic
field (which we take to be the \(z\)-axis), we choose the coordinate system such
that \(\langle\hat{F}_y\rangle=0\) without loss of generality, implying
\(\langle\hat{F}_+\rangle = \langle\hat{F}_-\rangle\).
In addition, to simplify further, we consider the specific case where the
transverse magnetisation is zero \(\langle\hat{F}_{\pm}\rangle = 0\) which is
valid in a system where \(q < 0\) such that the system favours atoms in the
outer components [see Eq.~\eqref{eq: spin-2-spin-vectors}].
Assuming the above, the stationary equations can be transformed into the
following simplified set of equations:
\begin{align}
    0 & = (-2p + 4q + c_0n + 2c_1n\langle\hat{F}_z\rangle - \mu)\zeta_2
    + \frac{c_2}{\sqrt{5}}nA_{00}\zeta^*_{-2},
    \label{eq: spin-2-stationary-zeta2}                                 \\
    0 & = (2p + 4q + c_0n - 2c_1n\langle\hat{F}_z\rangle - \mu)\zeta_{-2}
    + \frac{c_2}{\sqrt{5}}nA^*_{00}\zeta_{2},
    \label{eq: spin-2-stationary-zetam2}                                \\
    0 & = (-p + q + c_0n + c_1n\langle\hat{F}_z\rangle - \mu)\zeta_1
    + \frac{c_2}{\sqrt{5}}nA_{00}\zeta^*_{-1},
    \label{eq: spin-2-stationary-zeta1}                                 \\
    0 & = (p + q + c_0n - c_1n\langle\hat{F}_z\rangle - \mu)\zeta_{-1}
    + \frac{c_2}{\sqrt{5}}nA^*_{00}\zeta_1,
    \label{eq: spin-2-stationary-zetam1}                                \\
    0 & = \left(c_0n + \frac{c_2}{\sqrt{5}}nA_{00} - \mu\right)\zeta_0.
    \label{eq: spin-2-stationary-zeta0}
\end{align}
Noting that the above equations are decoupled in three parts, we can construct
the following matrix equations relating to
Eqs.~\eqref{eq: spin-2-stationary-zeta2}-\eqref{eq: spin-2-stationary-zetam2}
and Eqs.~\eqref{eq: spin-2-stationary-zeta1}
-~\eqref{eq: spin-2-stationary-zetam1}, respectively, as
\begin{align}
    \mqty(4q + 2\tilde{\beta} -\tilde{\mu} & \tilde{\alpha} \\
    \tilde{\alpha}^*               & 4q - 2\tilde{\beta} -\tilde{\mu})
    \mqty(\zeta_2                                                      \\
    \zeta_{-2}^*)                  & = 0, \label{eq: zeta-pm2-matrix}  \\
    \mqty(q + \tilde{\beta} -\tilde{\mu}   & -\tilde{\alpha} \\
    -\tilde{\alpha}^*              & q - \tilde{\beta} -\tilde{\mu})
    \mqty(\zeta_1                                                      \\
    \zeta_{-1}^*)                  & = 0, \label{eq: zeta-pm1-matrix}
\end{align}
with Eq.~\eqref{eq: spin-2-stationary-zeta0} being recast as
\begin{equation}\label{eq: spin-2-stationary-zeta0-recast}
    (\tilde{\alpha} - \tilde{\mu})\zeta_0 = 0,
\end{equation}
where \(\tilde{\mu} = \mu - c_0n\), \(\tilde{\alpha} = c_2nA_{00}/\sqrt{5}\) and
\(\tilde{\beta} = c_1n\langle\hat{F}_z\rangle - p\).
The stationary solutions are then classified according to the determinant of
the coefficient matrices of the above equations.
Explicitly, these are
\begin{align}
    D_2 & = {(4q-\tilde{\mu})}^2 -4\tilde{\beta}^2 - |\tilde{\alpha}|^2,
    \label{eq: D2}                                                        \\
    D_1 & = {(q - \tilde{\mu})}^2 - \tilde{\beta}^2 - |\tilde{\alpha}|^2.
    \label{eq: D1}
\end{align}
From these determinants and Eq.~\eqref{eq: spin-2-stationary-zeta0-recast},
we can derive stationary solutions that interpolate between different ground
states of the spin-2 system.

\subsection{Uniaxial nematic, biaxial nematic, and cyclic limits}
First consider the case \(D_1 \neq 0\) and \(D_2 = 0\).
Then, Eq.~\eqref{eq: zeta-pm1-matrix} and \(D_1 \neq 0\) implies that
\(\zeta_1=\zeta_{-1}=0\).
Here, consider the case that \(\tilde{\mu} = \tilde{\alpha}\).
Then all three of the \(\zeta_{\pm 2}, \zeta_0\) components can be non-zero.
From the definition of the longitudinal magnetisation given in
Eq.~\eqref{eq: spin-2-spin-vectors}, we have
\begin{align}
    |\zeta_2| = \sqrt{|\zeta_{-2}|^2 + \frac{\langle\hat{F}_z\rangle}{2}}.
\end{align}
The normalisation condition states \(|\zeta_2|^2 + |\zeta_0|^2 +
|\zeta_{-2}|^2=1\), which leads to
\begin{align}
    |\zeta_{-2}| &= \sqrt{\frac{1 - \zeta_0^2 -\langle\hat{F}_z\rangle/2}{2}},\\
    |\zeta_2| &= \sqrt{\frac{1 - \zeta_0^2 + \langle\hat{F}_z\rangle/2}{2}}.
\end{align}
Thus, the total spinor now reads
\begin{align}\label{eq: intermediate-UN-BN}
    \zeta = \mqty(
        e^{i\chi_2}\sqrt{\frac{1 - \zeta_0^2 + \langle\hat{F}_z\rangle/2}{2}} \\
        0 \\
        \zeta_0 \\
        0 \\
        e^{-i\chi_{-2}}\sqrt{\frac{1 -\zeta_0^2 - \langle\hat{F}_z\rangle/2}{2}}
    ).
\end{align}
The substitution of the above spinor into Eq.~\eqref{eq: zeta-pm2-matrix} leads
to the relations \(\tilde{\mu} = 2q-\tilde{\beta}^2/(2q)\) and
\(\langle\hat{F}_z\rangle = (\tilde{\beta} + p)/(c_1n)\), where
\(\tilde{\beta}\) can be calculated from the following
equation~\cite{Kawaguchi2012}:
\begin{align}
    \tilde{\beta}^3 +p\tilde{\beta}^2 + 4q[q + 2c_1n(1-\zeta_0^2)]\tilde{\beta}
    + 4pq^2 = 0.
\end{align}

Let us consider the case \(p=0\), then the above equation transforms to
\begin{align}
    \tilde{\beta}^2 = -4q\left[q + 2c_1n(1-\zeta_0^2)\right].
\end{align}
Then, since the left-hand side is positive, we derive the condition
\(|q| > |2c_1n(1-\zeta_0^2)|\).
Under this condition \(\tilde{\beta}\) is determined to be
\(\tilde{\beta}=0\)~\cite{Kawaguchi2012}, and hence \(\langle\hat{F}_z\rangle
=0\).
To determine \(\zeta_0\) we minimise the energy per
particle~\cite{Kawaguchi2012}
\begin{align}
    \epsilon = \sum_{m=-2}^{2}(-pm+qm^2)|\zeta_m|^2 + \frac{1}{2}c_0n
    + \frac{1}{2}c_1n\spinmag^2 + \frac{1}{2}c_2n|A_{00}|^2.
\end{align}
Substituting Eq.~\eqref{eq: intermediate-UN-BN} and
\(\langle\hat{F}_z\rangle=0\) into the above expression yields
\begin{align}
    \epsilon = 4q(1-\zeta_0^2) + \frac{1}{2}c_0n + \frac{c_2n}{10}\left|
        e^{i(\chi_2+\chi_{-2})}(1-\zeta_0^2) + \zeta_0^2
    \right|^2.
\end{align}
Note that since
\begin{align}
    \tilde{\alpha}=\tilde{\mu}=\frac{c_2n}{5}\left(
        e^{i(\chi_2+\chi_{-2})}\sqrt{(1-\zeta_0^2)
        -\frac{\langle\hat{F}_z\rangle}{4}}
        + \zeta_0^2\right),
\end{align}
which must be real, we require \(\chi_2 + \chi_{-2} = 0\) or \(\pi\).
If \(c_2 < 0\), then the energy is minimised by \(\chi_2+\chi_{-2}=\pi\), which
leads to the expression for \(\zeta_0\):
\begin{align}
    \zeta_0 = \sqrt{\frac{1 + 10q/(c_2n)}{2}}.    
\end{align}
Substituting \(\zeta_0\) back into Eq.~\eqref{eq: intermediate-UN-BN} leads to
the final interpolating spinor
\begin{align}
    \zeta^\text{C-N} = \mqty(
        ie^{i\chi}\frac{\sqrt{1 - 10q/(c_2n)}}{2} \\
        0 \\
        \sqrt{\frac{1 + 10q/(c_2n)}{2}} \\
        0 \\
        ie^{-i\chi}\frac{\sqrt{1 - 10q/(c_2n)}}{2}
    ),
\end{align}
where we have chosen \(\chi_{\pm 2} = \pi/2 + \chi\) to satisfy
\(\chi_2+\chi_{-2} = \pi\) and \(\chi\) is an arbitrary phase.
This solution continuously becomes the three-component cyclic state when
\(q = 0\), and the BN (UN) when \(q = -c_2n/10\) (\(q = c_2n/10\)).

Alternatively, when \(c_2 < 0\) the energy is minimised by having \(\chi_2
+ \chi_{-2} = 0\), which instead leads to the interpolating spinor
\begin{align}
    \zeta^\text{UN-BN} = \mqty(
        e^{i\chi}\frac{\sqrt{1 - 10q/(c_2n)}}{2} \\
        0 \\
        \sqrt{\frac{1 + 10q/(c_2n)}{2}} \\
        0 \\
        e^{-i\chi}\frac{\sqrt{1 - 10q/(c_2n)}}{2}
    ),
\end{align}
which now interpolates between the UN phase at \(q = |c_2|n/10\) and the BN
phase at \(q = -|c_2|n/10\).

\subsection{Cyclic to ferromagnetic}
Consider the case that \(D_2=D_1=0\).
Then, we have the following system of equations for the determinants
\begin{align}
    0 &= {(4q - \tilde{\mu})}^2 - 4\tilde{\beta}^2-|\tilde{\alpha}|^2, \\
    0 &= {(q - \tilde{\mu})}^2 - \tilde{\beta}^2 - |\tilde{\alpha}|^2,
\end{align}
which can be solved to find \(\tilde{\mu}\):
\begin{align}\label{eq: mu-C-FM}
    \tilde{\mu} = \frac{5q^2-\tilde{\beta}^2}{2q}.
\end{align}
Then, for the case that \(\tilde{\mu} \neq \tilde{\alpha}\) and \(q \neq 0\),
Eq.~\eqref{eq: spin-2-stationary-zeta0-recast} implies \(\zeta_0=0\).
Now, since we assumed zero transverse magnetisation,
Eq.~\eqref{eq: spin-2-spin-vectors} leads to
\begin{align}
    2c_1n(\zeta_2^*\zeta_1 + \zeta_{-1}^*\zeta_{-2}) = 0.
\end{align}
We can use the matrix Eqs~\eqref{eq: zeta-pm2-matrix}
and~\eqref{eq: zeta-pm1-matrix} to find expressions for \(\zeta_{-1}^*\) and
\(\zeta_{-2}\) or \(\zeta_{2}^*\) and \(\zeta_{1}\) which when substituted into
the above equation leads, respectively,
to the following equations
\begin{align}
    2c_1n\zeta_2^*\zeta_1\left(
        1 - \frac{\tilde{\beta} + 3q}{\tilde{\beta} - 3q}
    \right) &= 0, \\
    2c_1n\zeta_{-1}^*\zeta_{-2}\left(
        1 - \frac{\tilde{\beta} - 3q}{\tilde{\beta} + 3q}
    \right) &= 0, 
\end{align}
where we have substituted \(\tilde{\mu}\) according to Eq.~\eqref{eq: mu-C-FM}.
This implies that, generally, \(\zeta_2^*\zeta_1 = \zeta_{-1}^*\zeta_{-2} = 0\).
To be consistent with Eqs.~\eqref{eq: zeta-pm2-matrix}
and~\eqref{eq: zeta-pm1-matrix} either \(\zeta_1=\zeta_{-2}=0\) or
\(\zeta_2=\zeta_{-1} = 0\).
Here we focus on the former case, since it relates to our discussion in
Sec.~\ref{sec: spin-2-C-FM}.
The longitudinal magnetisation in Eq.~\eqref{eq: spin-2-spin-vectors} implies we
now have
\begin{align}
    |\zeta_2| = \sqrt{\frac{\langle\hat{F}_z\rangle + |\zeta_{-1}|^2}{2}}.    
\end{align}
Then, using the normalisation condition \(|\zeta_{2}|^2 + |\zeta_{-1}|^2 = 1\)
leads to
\begin{align}
    |\zeta_{-1}| &= \sqrt{\frac{2 - \langle\hat{F}_z\rangle}{3}}, \\
    |\zeta_2| &= \sqrt{\frac{1 + \langle\hat{F}_z\rangle}{3}},
\end{align}
and thus the final interpolating spinor now reads
\begin{align}
    \zeta^\text{C-FM} = \frac{1}{\sqrt{3}}\mqty(
        e^{i\chi_2}\sqrt{1 + \langle\hat{F}_z\rangle} \\
        0 \\
        0 \\
        e^{i\chi_{-1}}\sqrt{2 - \langle\hat{F}_z\rangle} \\
        0
    ).
\end{align}
Substituting this back into Eqs.~\eqref{eq: zeta-pm2-matrix}
and~\eqref{eq: zeta-pm1-matrix} leads to \(\langle\hat{F}_z\rangle =
(p - q)/(c_1n)\).
This state now continuously becomes the two-component cyclic state at
\(\langle\hat{F}_z\rangle = 0\) and the \(\text{FM}_2^+\) state at
\(\langle\hat{F}_z\rangle = 2\).

\subsection{Ferromagnetic to biaxial nematic}
Consider the case \(D_2 \neq 0\) and \(D_1 = 0\).
Then, \(D_2 \neq 0\) implies that \(\zeta_2=\zeta_{-2}=0\).
Consider also the case that \(\tilde{\mu} \neq \tilde{\alpha}\), then
Eq.~\eqref{eq: spin-2-stationary-zeta0-recast} implies that \(\zeta_0=0\).
Now, from the definition of the longitudinal magnetisation given in
Eq.~\eqref{eq: spin-2-spin-vectors}, we have
\begin{align}
    |\zeta_2| = \sqrt{|\zeta_{-2}|^2 + \frac{\langle\hat{F}_z\rangle}{2}}.
\end{align}
Using the normalisation condition \(|\zeta_2|^2 + |\zeta_{-2}|^2 = 1\) then
leads to
\begin{align}
    |\zeta_{-2}| &= \sqrt{\frac{1 - \langle\hat{F}_z\rangle / 2}{2}}, \\
    |\zeta_2| &= \sqrt{\frac{1 + \langle\hat{F}_z\rangle / 2}{2}}.
\end{align}
Thus, the final interpolating spinor now reads
\begin{align}
    \zeta^\text{FM-BN} = \mqty(
        e^{i\chi_2}\sqrt{\frac{1 + \langle\hat{F}_z\rangle / 2}{2}} \\
        0 \\
        0 \\
        0 \\
        e^{i\chi_{-2}}\sqrt{\frac{1 - \langle\hat{F}_z\rangle / 2}{2}}
    ).
\end{align}
Substituting the above spinor into Eq.~\eqref{eq: zeta-pm2-matrix} and using
\(D_2 = 0\) leads to \(\langle \hat{F}_z \rangle = p / [(c_1-c_2/20)n]\).
Note that the above spinor becomes the BN phase at \(p=0\) and
\(\text{FM}_2^\pm\) for \(p = \pm(2c_1-c_2/10)n\).