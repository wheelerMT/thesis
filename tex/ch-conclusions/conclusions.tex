\chapter{Conclusions \& future work}\label{chap: conclusions}
In this Chapter we give an overview of the overarching conclusions of the
thesis, in addition to suggesting some avenues for future work.

In Part I we introduced the necessary mathematical framework needed to
understand these systems in detail.
We started with the scalar system, constructing the full Hamiltonian using a
quantum treatment before introducing the mean-field theory and resulting GPEs.
Then, we generalised the theory to two-component systems, and derived the
miscibility criterion.
The theory necessary to understand spinor BECs was introduced, where we
constructed the single-particle and interaction Hamiltonians using a quantum
treatment, then introduced the mean-field equations along with their reduction
to lower dimensions and dimensionless versions.
Finally, an overview of the ground states, symmetries, and topologically stable
defects that arise in spin-1 and spin-2 systems was introduced.
For each ground state we presented two different graphical representations, and
then discussed the dynamical properties of certain vortex states arising in
these systems.

\section{The versatility of spinor Bose-Einstein condensates}
In this thesis we have shown the robustness of spinor and pseudospin-1/2 systems
for investigating a variety of different physics.
We showed how the relaxation dynamics of HQVs in pseudospin-1/2 systems exhibit
similarities in its spatial properties compared to similar studies concerning
scalar vortices in scalar BECs~\cite{Nowak2012,Karl2017}, despite being a
topologically distinct vortex.
However, there was a multitude of differences found in the temporal aspects of
the scaling.
It was shown that the decay rate of the vortices at early times was strongly
correlated to the ratio of the inter- and intra-species interactions, leading to
wildly different dynamics at these early times, before the decay rate at later
times tended to a universal scaling which has been observed in scalar
systems~\cite{Karl2017}.

In addition, we showed how spinor condensates can be used as a test bed to
discern more about the scaling laws associated with discontinuous, first-order
quantum phase transitions.
We presented the broken-axisymmetry to ferromagnetic phase transition as an
example, and showed that it did indeed fit the requirements of a first-order
transition.
We generalised the KZM, and showed how scaling laws that govern the density of
defects can still be found despite having a gapless spectrum.
Independent of the KZM, we also derived scaling laws for the onset of the decay
of the metastable state after the transition point is crossed, which aligned
with the predictions of our generalised KZM, further justifying the robustness
of our theory.

Finally, we showed that spinor BECs are excellent candidate for investigating
topological interface physics due to their rich phase diagrams.
In particular, we presented stationary solutions derived from the spinor GPEs
that offered interpolating solutions between two distinct ground states
dependent on an interpolating parameter.
It was also shown how a myriad of different defect states can be connected
across the interface, ranging from simple quantised phase vortices to more
complicated structures representing generalisations of the Dirac monopole.
Our work concluded with mean-field numerical simulations of select examples,
and showed vastly different results between the cases, ranging from the
formation of composite core structures and splitting of phase vortices in an
interface between the uniaxial nematic and biaxial nematic phases, to complex
splitting process of nonsingular vortices in a cyclic to ferromagnetic
interface.

\section{Future work}

\subsection{Understanding more about half-quantum vortex dynamics}
Our numerical investigation of the relaxation dynamics of HQVs investigated
in Chapter~\ref{chap: two-comp} revealed interesting behaviour unseen in similar
relaxation studies involving scalar condensates~\cite{Karl2017}.
In particular, a clear attractive force between the vortices dominates at
sufficiently high \(\gamma \), which simple point-vortex models have been unable
to account for~\cite{Eto2011,Kasamatsu2016}, so discerning more about the
dynamics of these objects is an active topic.
In fact, recent theoretical work has already shed more light on the
topic~\cite{Richaud2023}.
They showed that the atoms filling the large cores of the HQVs act as a pinning
potential, driving the HQVs to collide.
This work therefore explains previously observed numerical simulations of HQV
dynamics in these systems~\cite{Eto2011, Kasamatsu2016}, and provides a
better insight into our observed decay laws.

In addition, experimental work has already been conducted investigating the
relaxation dynamics of HQVs in a spin-1 BEC~\cite{Seo2016}.
However, corresponding numerical investigations of the decay rate of the
vortices has not been carried out, and it would indeed be interesting to see
whether there is overlap with our pseudospin-1/2 work.

\subsection{Quantum phase transitions and metastability}
Recently, there has been new interest in understanding quantum phase transitions
where metastability plays a crucial role such as in false-vacuum
decay~\cite{Billam2022,Song2022, Zenesini2023, Lagnese2023}.
However, a lack of theoretical understanding of first-order quantum phase
transitions leads to confusion to exactly how the metastable state decays.
Our work presented in Chapter~\ref{chap: spin-1} presents a theoretical
framework for extracting scaling laws associated with first-order quantum phase
transitions both for the onset of the decay of the metastable state itself and
the density of defects far past the transition point.
In addition, the work carried out is applicable in experimentally-relevant
parameter regimes, which opens up the avenue of spin-1 BECs being used as
emulators to understand the recent interest in false-vacuum decay.

\subsection{Experimental realisations of topological interfaces}
Our work in Chapter~\ref{chap: spin-2} already extends the work of
Refs.~\cite{Borgh2012, Borgh2013, Borgh2014} by considering topological
interfaces between the ground state phases of spin-2 BECs, but can be used as
a further stepping stone to understand more about topological interface physics.
Interfaces formed within vortex cores has recently been achieved in both
spin-1~\cite{Weiss2019,Xiao2021} and spin-2 BECs~\cite{Xiao2022}, but current
experimental realisations of interfaces formed between bulk regions has remained
elusive due to lack of proper experimental techniques.
Once an experimental realisation of such an interface is achieved, our work
can be used to understand the resulting core structure of defects connected
across the interface.
Naturally, further numerical studies of more complicated examples, e.g.,
monopoles, present an avenue for studying the dynamics of such objects when
constrained to topological interfaces.

