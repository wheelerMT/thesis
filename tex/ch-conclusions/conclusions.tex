\chapter{Conclusions \& future work}
In this thesis we have investigated a variety of topics in spinor and
pseudospinor condensates.
In particular, we investigated the relaxation dynamics of half-quantum vortices
in a pseudospin-1/2 system, a generalised Kibble-Zurek scaling in a spin-1
system, and finally topological interface dynamics in a spin-2 system.

Chapters 2 and 3 introduced the necessary mathematical framework needed to
understand these systems in detail.
We provided detailed constructions of the resulting interaction Hamiltonians for
each of these systems, with a focus on the differences arising.
Furthermore, chapter 3 introduced novel vortex structures encountered in spinor
BECs and how to construct them using spin rotations.

Chapter 4 investigated the relaxation dynamics of half-quantum vortices (HQVs)
in a pseudospin-1/2 system.
We presented a method of constructing an initially turbulent system of \(48^2\)
HQVs using previous numerical work for a scalar system~\cite{Billam2014}.
We then described our method of investigating the relaxation process of the
system by varying the ratio of inter-species and intra-species interactions,
\(\gamma \).
A minor result of the study showed that, despite carrying half the circulation,
the kinetic energy spectrum revealed similarities to that of scalar vortices.
The major result of the study was identifying anomalous scaling laws of the
vortex number that occurred when \(\gamma \gtrsim 0.6\).
In particular, for values of \(\gamma < 0.6\), there was a uniform \(t^{-2/5}\)
scaling of the vortex number throughout the simulations.
However, as \(\gamma \) was increased past this point, a shoulder started to
develop, separating the relaxation dynamics into two distinct parts.
A larger \(\gamma \) led to a steeper decay rate of the vortices at early times,
with the largest \(\gamma \) \((\gamma = 0.9)\) achieving a \(t^{-3/2}\) decay
rate.
At later times the anomalous scaling reverted to the observed \(t^{-2/5}\)
scaling.
We identified the cause of the anomalous scaling by referring to previous works
investigating the inter-vortex forces~\cite{Eto2011, Kasamatsu2016}.
The larger \(\gamma \) increased the core sizes of the vortices, which in turn
meant that the inter-vortex separation was smaller.
This then meant that a new, attractive force otherwise unseen in scalar
vortices dominated, and caused the HQVs to move together and annihilate,
leading to the anomalous scaling.

In chapter 5 we investigated the Kibble-Zurek mechanism across a first-order
phase transition in a spin-1 BEC\@.
In particular, we started from the broken-axisymmetry phase of a condensate with
ferromagnetic interactions, then proceeded to quench the quadratic Zeeman shift,
\(q\), which caused the ground state of the condensate to become ferromagnetic.
Firstly, we analytically constructed scaling laws using a modified Kibble-Zurek
mechanism that accounted for our discontinuous phase transition, using theory
developed for a discontinuous critical point~\cite{Suzuki2015}.
This led to a prediction of the scaling of the density of defects as
\(\tau_Q^{-1/4}\), where \(\tau_Q\) is the quench rate, which is noticeably
different from the traditional scaling of \(\tau_Q^{-1/3}\) seen across
second-order transitions~\cite{Damski2007, Anquez2016}.
In addition, we analytically derived the scaling properties near the critical
point, which came out to be \(\tau_Q^{-1/2}\).
We then carried out numerical investigations of this phase transition.
The numerics confirmed the formation of ferromagnetic domains in the system
after the critical point of \(q_c=0\) was crossed.
The size of these domains was indeed confirmed to depend on the quench rate,
\(\tau_Q\).
Using the simulations, we then calculated the total number of ferromagnetic
domains in the system at the end of the simulation for each quench rate tested.
We found that for sufficiently slow quenches (\(\tau_Q \lesssim 2\times 10^3\))
the scaling of the defect density obeyed the predicted \(\tau_Q^{-1/4}\)
scaling.
However, at faster quench rates this scaling broke down, and instead approached
a scaling close to \(\tau_Q^{-1/5}\).
\textcolor{red}{We revealed that this phenomenon is due to\ldots}
The scaling near the critical point was also calculated, and agreed perfectly
with our predicted scaling of \(\tau_Q^{-1/2}\).
Finally, we also tested our predicted scaling laws in a system that crossed
two phase transitions, i.e., going through the polar to broken-axisymmetry to
the FM phases.
Despite passing through two phase transitions, with one of them being
second-order, the scaling of the defect density at late times was qualitatively
similar to that observed just crossing one phase transition.

Chapter 6 was concerned with investigating topological interfaces in spin-2
BECs.
We derive stationary solutions of the spin-2 Gross-Pitaevskii equations that
provided interpolating solutions between different ground state manifolds.
We then discussed the energetic stability of each of the interfaces, and how
they can be manipulated with both linear and quadratic Zeeman shifts.
From here, we progressed into a discussion regarding the connection of
topological defects across such interfaces.
Numerous types of defects were considered ranging from singly quantised
vortices (SQVs), spin vortices, fractional vortices, coreless vortices, and even
monopole point defects.
The remainder of the chapter presented numerical work investigating different
interfaces and interesting vortex connections across them.
In a uniaxial-nematic to biaxial-nematic (UN-BN) interface we showed that SQVs
initially connected across the interface spatially separate.
After doing so, composite core structures develop which are filled with a
variety of different phases.
In addition, we showed that a SQV in the BN phase connected to a vortex free
state in the UN phase would split into HQVs.
A monopole point defect was also constructed on the UN side of the interface,
which smoothly terminated on a singly quantised spin vortex line in the BN
phase.
Finally, we investigated both fractional and coreless vortices across a cyclic
to ferromagnetic (C-FM) interface.
In particular, we observed a SQV on the FM side smoothly connecting to a
third vortex on the cyclic side of the interface.
The \(\spinmag=1\) core of the third vortex extended through the FM region,
which was topologically protected by having a cyclic shell encapsulating the
vortex core.
The coreless vortex on the FM side was revealed to connect to a doubly quantised
vortex (DQV) line in the cyclic region.
Numerical results showed complex splitting behaviour from this connection.
In particular, the initial DQV on the cyclic side split into four third
vortices alongside a central two-third vortex.
On the FM side, the coreless vortex is observed to split into four singular
vortex structures.

\section{Future work}

\subsection{Two-component system}
Our numerical investigation of the relaxation dynamics of HQVs revealed
interesting behaviour unseen in similar relaxation studies involving scalar
condensates~\cite{Karl2017}.
In particular, a clear attractive force between the vortices dominates at
sufficiently high \(\gamma \), so a logical next step is to determine the cause
of this attractive force.
In fact, recently theoretical work has shed more light on the
topic~\cite{Richaud2022}.
They showed that the atoms filling the large cores of the HQVs act as a pinning
potential, driving the HQVs to collide.
This work therefore explains previously observed numerical simulations of HQV
dynamics in these systems~\cite{Eto2011, Kasamatsu2016}, and provides a
better insight into our observed decay laws.

In addition, experimental work has already been conducted investigating the
relaxation dynamics of HQVs in a spin-1 BEC~\cite{Seo2016}.
However, corresponding numerical investigations of the decay rate of the
vortices has not been carried out, and it would indeed be interesting to see
whether there is overlap with our pseudospin-1/2 work.

\subsection{Spin-1 Kibble-Zurek}
A natural extension of this work would be to a spin-2 system, which contains
further second-order and first-order phase transitions.
For example, a prime candidate for a first-order transition would be between the
nematic and ferromagnetic regimes, which could be achieved similarly to our
current work by quenching the magnetic field.

\textcolor{red}{What else?}

\subsection{Spin-2 interface problems}
Naturally, there are numerous other topological defect connections one can
extend from our current work.
However, a more beneficial direction may be in trying to recreate the
topological interfaces created in both our work and previous spin-1
work~\cite{Borgh2012, Borgh2013, Borgh2014}, which so far has remained elusive.

\textcolor{red}{What else?}
