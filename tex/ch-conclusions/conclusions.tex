\chapter{Conclusions \& future work}
In this thesis we have investigated a variety of topics in spinor and
pseudospinor condensates.
In particular, we investigated the relaxation dynamics of half-quantum vortices
in a pseudospin-1/2 system, a generalised Kibble-Zurek scaling in a spin-1
system, and finally topological interface dynamics in a spin-2 system.

Chapters 2 and 3 introduced the necessary mathematical framework needed to
understand these systems in detail.
We provided detailed constructions of the resulting interaction Hamiltonians for
each of these systems, with a focus on the differences arising.
Furthermore, chapter 3 introduced novel vortex structures encountered in spinor
BECs and how to construct them using spin rotations.

Chapter 4 investigated the relaxation dynamics of half-quantum vortices (HQVs)
in a pseudospin-1/2 system.
We presented a method of constructing an initially turbulent system of \(48^2\)
HQVs using previous numerical work for a scalar system~\cite{Billam2014}.
We then described our method of investigating the relaxation process of the
system by varying the ratio of inter-species and intra-species interactions,
\(\gamma \).
A minor result of the study showed that, despite carrying half the circulation,
the kinetic energy spectrum revealed similarities to that of scalar vortices.
The major result of the study was identifying anomalous scaling laws of the
vortex number that occurred when \(\gamma \gtrsim 0.6\).
In particular, for values of \(\gamma < 0.6\), there was a uniform \(t^{-2/5}\)
scaling of the vortex number throughout the simulations.
However, as \(\gamma \) was increased past this point, a shoulder started to
develop, separating the relaxation dynamics into two distinct parts.
A larger \(\gamma \) led to a steeper decay rate of the vortices at early times,
with the largest \(\gamma \) \((\gamma = 0.9)\) achieving a \(t^{-3/2}\) decay
rate.
At later times the anomalous scaling reverted to the observed \(t^{-2/5}\)
scaling.
We identified the cause of the anomalous scaling by referring to previous works
investigating the inter-vortex forces~\cite{Eto2011, Kasamatsu2016}.
The larger \(\gamma \) increased the core sizes of the vortices, which in turn
meant that the inter-vortex separation was smaller.
This then meant that a new, attractive force otherwise unseen in scalar
vortices dominated, and caused the HQVs to move together and annihilate.


\section{Future work}

\subsection{Two-component system}

\subsection{Spin-1 Kibble-Zurek}

\subsection{Spin-2 interface problems}

